% !TEX root = ./gruppi.tex
% !TEX program = lualatex

\documentclass[ structure      = article
              , maketitlestyle = standard
              , twocolcontents = toc
              , secstyle       = center
              , secfont        = roman
              , liststyle      = aligned
              , quotesize      = normalsize
              , headerstyle    = center
              , footnotestyle  = dotted
              ]{suftesi}

\usepackage[no-math]{fontspec}
\usepackage[rm,tt=false]{libertine}
\usepackage{polyglossia}
\setmainlanguage{italian}
\setotherlanguage{english}
\setotherlanguage{german}
\usepackage{microtype}

\usepackage{hyperref}
\hypersetup{ breaklinks
           , colorlinks
           , allcolors = blue!50!black
           }

\usepackage[autostyle,italian=quotes]{csquotes}
\usepackage[ bibstyle = alphabetic
           , citestyle = alphabetic
           , pluralothers=true
           , autolang=langname
           ]{biblatex}
\addbibresource{biblio.bib}
\nocite{*}

\usepackage{enumitem}
\setlist[enumerate]{ref=(\arabic*)}

%\usepackage{graphicx}

\usepackage{libertinust1math}
\usepackage{MnSymbol}
\usepackage[bb=ams]{mathalfa}
\usepackage{mathtools}
\let\underbrace\LaTeXunderbrace
\let\overbrace\LaTeXoverbrace

\usepackage{amsthm}
\newcounter{thmcounter}
\counterwithin{thmcounter}{section}

\theoremstyle{definition}

\newtheorem{teor}[thmcounter]{Teorema}
\newtheorem{prop}[thmcounter]{Proposizione}
\newtheorem{lemm}[thmcounter]{Lemma}
\newtheorem{coro}[thmcounter]{Corollario}
\newtheorem{defi}[thmcounter]{Definizione}
\newtheorem{osse}[thmcounter]{Osservazione}
\newtheorem{rich}[thmcounter]{Richiamo}
\newtheorem{esem}[thmcounter]{Esempio}
\newtheorem{eser}[thmcounter]{Esercizio}

\numberwithin{equation}{section}

\usepackage{tikz}
\usetikzlibrary{ calc
               , babel
               , cd
               }
\tikzset{>={To[length=3pt,width=3pt]}}
\tikzcdset{ row sep/normal=.7cm
          , column sep/normal=.7cm
          , arrow style=tikz
          , diagrams={>={To[length=3.5pt,width=4pt]}}
          , shorten=-2pt
          }

% Grafo di Cayley dei gruppi diedrali.
\newcommand\CayleyDihedral[2][1cm]{
  \begin{tikzpicture}[point/.style={circle, inner sep=2pt}]
  \pgfmathsetmacro\rotang{360/#2}
  \foreach \i in {1, ..., #2} {
    \pgfmathsetmacro\j{int(\i-1)}
    \pgfmathsetmacro\k{int(#2-\j)}
    \pgfmathsetmacro\angle{90-\rotang*\j}
    \node (X\i) at (\angle:#1) {\ifnum \j=0 $1$ \else $r^\j$ \fi};
    \node (Y\i) at (\angle:{2*#1}) {\ifnum \k=#2 $s$ \else $r^\j s = s r^\k$ \fi};
  }
  \foreach \i in {1, ..., #2} {
    \pgfmathsetmacro\j{int(mod(\i,#2)+1)}
    \draw [semithick, blue, ->] (X\i) -- (X\j);
    \draw [semithick, blue, ->] (Y\j) -- (Y\i);
    \draw [semithick, red]  (X\i) -- (Y\i);
  }
  \end{tikzpicture}
}

\newcommand\im{\operatorname{im}}
\newcommand\iso\cong
\newcommand\niso\ncong
\newcommand\id{\mathrm{id}}
\newcommand\mor[1]{\xrightarrow{#1}}
\newcommand\card[1]{\left\lvert#1\right\rvert}
\newcommand\rest[1]{{|}_{#1}}
\newcommand\N{\mathbb{N}}
\newcommand\Z{\mathbb{Z}}
\newcommand\Q{\mathbb{Q}}
\newcommand\R{\mathbb{R}}
\newcommand\C{\mathbb{C}}
\newcommand\F{{\mathbb F}}
\newcommand\Hom{\mathrm{Hom}}
\newcommand\Iso{\mathrm{Isom}}
\newcommand\Aut{\mathrm{Aut}}
\newcommand\Ann{\mathrm{Ann}}
\newcommand\mcd{\operatorname{mcd}}
\newcommand\ord{\operatorname{ord}}
\newcommand\cl[1]{\overline{#1}}
\newcommand\gen[1]{\left\langle#1\right\rangle}
\newcommand\conj\Gamma
\newcommand\Int{\mathrm{Int}}
\newcommand\parti{\mathcal{P}}
\newcommand\normal{\triangleleft}
\newcommand\cc[1]{\left[#1\right]}
\newcommand\Stab{\operatorname{Stab}}

%%%
\usepackage{xcolor}
\newcommand\nota[1]{\textcolor{red}{[#1]}}

\renewcommand\implies\Rightarrow
\renewcommand\tilde\widetilde
\renewcommand\hat\widehat
\renewcommand\bar\overline
\renewcommand\epsilon\varepsilon