% !TEX root       = ../gruppi.tex
% !TEX program    = lualatex
% !TEX spellcheck = it_IT

\section{Applicazioni}

Nel {\scshape Teorema di Sylow II} (Teorema~\ref{teor:SylowII}), se $m$ è primo, il vincolo su $s_p$ è molto forte: è uguale a $1$ oppure a $m$. Le applicazioni qui sotto si basano proprio su questo.

%\section{Gruppi di ordine $pq$}
\begin{prop}[Gruppi di ordine $pq$]\label{prop:GruppiPQ}
Se $G$ è un gruppo di ordine $pq$, con $p, q$ primi e $p < q$, allora
\begin{enumerate}
\item $s_q = 1$.
\item Se $q \not\equiv 1 \mod p$, allora $s_p=1$ e $G\iso C_p\times C_q\iso C_{pq}$.
\end{enumerate}
In particolare, $G$ non è semplice.
\end{prop}

\begin{proof}
Per il Teorema di Sylow II, 
\[
\begin{cases}
s_p \in \{1+kp : k \in \N\} \\ 
s_p \in \{1, q\}
\end{cases}
\quad
\begin{cases}
s_q \in \{1+hq : h \in \N\} \\
s_q \in \{1, p\}
\end{cases}
\]
Qui, poiché $p < q$, è sicuro che $s_q = 1$. Se in più assumiamo che $q \notin \{1 + kp : k \in \N\}$, allora possiamo concludere che anche $s_p = 1$.
\end{proof}

%\begin{osse}
%Si vedrà che $q\equiv1\mod p$ $\implies$ $\exists!$ (a meno di isomorfismo) un gruppo non abeliano di ordine $pq$. Un esempio (per $p=2$) è il gruppo diedrale $D_q$.
%\end{osse}

%\section{Gruppi di ordine $p^2q$}

\begin{prop}[Gruppi di ordine $p^2q$]\label{prop:GruppiP2Q}
Sia $G$ un gruppo di ordine $p^2q$, dove $p$ e $q$ sono numeri primi distinti. Allora $G$ ha un sottogruppo {\em normale} di ordine $p^2$ oppure di ordine $q$. In particolare, $G$ non è semplice.% $s_p=1$ o $s_q=1$ (quindi $G$ non è semplice).
\end{prop}

\begin{proof}
Possiamo equivalentemente ridurci a dimostrare che $s_p = 1$ oppure $s_q = 1$. Per il {\scshape Teorema di Sylow II}, 
\[\begin{aligned}
& s_p \in \{1, 1+p, 1+2p, \dots{}\} \cap \{1, q\} \\
& s_q \in \{1, 1+q, 1+2q, \dots{}\} \cap \{1, p, p^2\} .
\end{aligned}\]
Se $p > q$, allora $s_p = 1$. Vediamo cosa succede se $p < q$. Dimostriamo che
\begin{quotation}
se $s_q \ne 1$, allora $s_p = 1$.
\end{quotation}
Se $s_q \ne 1$, allora l'unica possibilità è che $s_q = p^2$. Ricordiamo che $s_q$ è il numero dei $q$-Sylow ed osserviamo che questi sottogruppi sono tutti ciclici perché hanno ordine primo, e necessariamente si intersecano a due a due banalmente. Allora in $G$ ci sono $s_q(q-1) = p^2(q-1)$ elementi di ordine $q$. Pertanto il numero di elementi di ordine diverso da $q$ è $p^2q - p^2(q-1) = p^2$. Tutti i $p$-Sylow devono avere $p^2$ elementi: eccoli questi $p^2$ elementi. C'è quindi un unico $p$-Sylow.
\end{proof}

\begin{osse}
In realtà, nel caso $p<q$ e $s_q>1$, possiamo concludere che $p=2$ e $q=3$. Infatti $s_q = p^2 \equiv 1 \mod q$, cioè $q$ divide $(p^2-1)=(p-1)(p+1)$. Sicuramente $q$ non divide $p-1$, quindi divide $p+1 \le q$. La conclusione è $q=p+1$: gli unici primi a fare ciò appunto sono $2$ e $3$ rispettivamente. Puoi pure rifare la seconda parte della dimostrazione con questi due numeri esplicitamente. Un esempio di questo tipo è $G=A_4$ (esercizio!).
\end{osse}


%\section{Gruppi di ordine $pqr$}
\begin{prop}[Gruppi di ordine $pqr$]\label{prop:GruppiPQR}
Sia $G$ un gruppo finito di ordine $pqr$, dove $p, q, r$ primi e $p < q < r$. Allora $G$ ha un un sottogruppo normale di ordine $q$ oppure $r$. In particolare, $G$ non è semplice.
\end{prop}

\begin{proof}
Analogamente a prima, dimostriamo
\begin{quotation}
se $s_r \ne 1$, allora $s_q = 1$.
\end{quotation}
Per il {\scshape Teorema di Sylow II}, 
\[\begin{aligned}
& s_q \in \{1, 1+q, 1+2q, \dots{}\} \cap \{1, p, r, pr\} \\
& s_r \in \{1, 1+r, 1+2r, \dots{}\} \cap \{1, p, q, pq\} .
\end{aligned}\]
Andiamo per esclusione prima.
Sicuramente $s_r$ non può essere $p$ oppure $q$, e quindi rimane solo $s_r = pq$. Per quanto riguarda $s_q$, non può essere $p$ e rimangono $1$, $r$ e $pr$. Ragioneremo fino ad escludere anche $r$ e $pr$ . Come prima, i sottogruppi di $r$-Sylow sono $pq$ e sono tutti ciclici. In $G$ ci sono $pqr - pq(r-1) = pq$ elementi di ordine diverso da $r$. Quanti sono gli elementi di ordine $q$? Ce ne sono $s_q (q-1)$, il ragionamento è lo stesso. Quindi necessariamente $s_q(q-1) \le pq$. Per ipotesi, si ha anche $s_qp \le s_q(q-1)$ e quindi $s_q \le q$. Chi sopravvive è $s_q = 1$.
\end{proof}

%\section{I gruppi semplici non hanno sottogruppi di indice piccolo}

\begin{prop}\label{prop:IndiceTra2e4}
Sia $G$ un gruppo finito e $H$ un suo sottogruppo non banale. Se $2\le[G:H]\le4$, allora $G$ non è semplice.
\end{prop}

\begin{proof}
Supponiamo che $G$ sia semplice e $n$ sia il suo ordine. In tal caso $\ker L$, vedi Esempio~\ref{esem:AzioneTraslazioneClassiLaterali}, ha due sole possibilità: è banale oppure tutto $G$. La seconda chiaramente non è possibile perché questo vorrebbe dire che $G=H$, mentre $[G:H] \ne 1$. Quindi $L$ è iniettivo, e il {\scshape Primo Teorema di Isomorfismo} implica che $G$ ha una copia isomorfa $G'$ dentro $S(G/H)$. Poiché Le classi laterali sono solo $m \in {2, 3, 4}$, possiamo identificare $S(G/H)$ con $S_m$. Osserviamo che così stando le cose, $m \mid n \mid m!$ e $m \ne n$. Ci sono quindi questi scenari, al variare di $m$:
\begin{itemize}
%\item Per assurdo $G$ semplice $\implies$ $L : G\to S(G/H)$ è un omomorfismo iniettivo $\implies$ $G':=\im(L)<S(G/H)\iso S_m$ (con $m:=[G:H]$) tale che $G'\iso G$ e $n:=\card{G}=\card{G'}$ soddisfa $m\mid n\mid m!$ (per il teorema di Lagrange) e $n>m$ (perché $H\ne\{1\}$).
\item Se $m=2$, allora $n = m$, il che è impossibile.
\item Se $m=3$, allora $n = 6 = 2 \cdot 3$. Per la Proposizione~\ref{prop:GruppiPQ}, segue che $G$ non è semplice, assurdo anche questo.
\item Se $m=4$, allora $n = 8 = 2^3$ o $n = 12 =2^2 \cdot 3$ oppure $n = 24 = 2^3 \cdot 3$. %($\implies$ $G\iso S_4$).
In ogni caso, $G$ non è semplice.\qedhere
\end{itemize}
\end{proof}

%\begin{osse}
%Segue che $G$ non è semplice se $\exists p$ primo tale che $p\mid\card{G}$ e \\
%$s_p=3$ ($\implies$ $p=2$) o $s_p=4$ ($\implies$ $p=3$).
%\end{osse}


%\section{Esercizio}

\begin{prop}
I gruppi non abeliani di ordine $< 60$ non sono semplici.
\end{prop}

\begin{proof} Sia $G$ un gruppo non abeliano di ordine $n < 60$. Un prima scrematura: i gruppi di ordini $p$ o $p^2$ per $p$ primo sono abeliani. Quindi gli ordini sotto $60$ da esaminare restano: $6$, $8$, $10$, $12$, $14$, $15$, $16$, $18$, $20$, $21$, $22$, $24$, $26$, $27$, $28$, $30$, $32$, $33$, $34$, $35$, $36$, $38$, $39$, $40$, $42$, $44$, $45$, $46$, $48$, $50$, $51$, $52$, $54$, $55$, $56$, $57$ e $58$.
\begin{itemize}
\item Se $n=p^k$ con $p$ primo e $k>2$, allora $G$ non è semplice. Gli ordini della forma $p^k$ con $k > 2$ sotto il $60$ sono: $2^3 = 8$, $2^4 = 16$, $2^5 = 32$ e $3^3 = 27$. 
\item Se $n$ è il prodotto di tre primi distinti, allora $G$ non semplice (Proposizione~\ref{prop:GruppiPQR}). Gli ordini $< 60$ di questa forma sono: $2 \cdot 3 \cdot 5 = 30$, $2 \cdot 3 \cdot 7 = 42$.
\item Se $n$ è prodotto di due primi, allora $G$ non è semplice (Proposizione~\ref{prop:GruppiPQ}). Questa volta abbiamo coperto molti casi: $2 \cdot 3 = 6$, $2 \cdot 5 = 10$, $2 \cdot 7 = 14$, $2 \cdot 11 = 22$, $2 \cdot 13 = 26$, $2 \cdot 17 = 34$, $2 \cdot 19 = 38$, $2 \cdot 23 = 46$, $2 \cdot 29 = 58$, $3 \cdot 5 = 15$, $3 \cdot 7 = 21$, $3 \cdot 11 = 33$, $3 \cdot 13 = 39$, $3 \cdot 17 = 51$, $3 \cdot 19 = 57$, $5 \cdot 7 = 35$ e $5 \cdot 11 = 55$.
\item Se $n$ è della forma $p^2q$ con $p, q$ primi, allora $G$ non è semplice (Proposizione~\ref{prop:GruppiP2Q}). Gli ordini sono: $2^2 \cdot 2 = 8$, $2^2 \cdot 3 = 12$, $2^2 \cdot 5 = 20$, $2^2 \cdot 7 = 28$, $2^2 \cdot 11 = 44$, $2^2 \cdot 13 = 52$, $3^2 \cdot 2 = 18$, $3^2 \cdot 3 = 27$, $3^2 \cdot 5 = 45$ e $5^2 \cdot 2 = 50$.
\end{itemize}
%
Interrompiamo per elencare gli ordini sopravvissuti: $24$, $36$, $40$, $48$, $54$ e $56$. %Osserviamo che, a parte il $36$, gli altri sono della forma $p^k q$ con $p, q$ primi distinti e $k \ge 3$. 
%
\begin{itemize}
%\item Resta il caso $n=p^iq^j$ con $p$, $q$ primi distinti e $i,j>0$.
\item Vediamo i gruppi di ordine $36 = 2^2 \cdot 3^2$. %$i,j\ge2$ $\implies$ $i=j=2$ (se no $n\ge2p^2q^2\ge2\cdot2^2\cdot3^2=72$) $\implies$ $n=2^2\cdot3^2=36$ (se no $n\ge2^2\cdot5^2=100$) $\implies$ 
Questi hanno dei $3$-Sylow di indice $4$: per la Proposizione~\ref{prop:IndiceTra2e4}, non sono semplici.
%\item Posso supporre $j=1$.
%\item $i=1$ o $i=2$ $\implies$ $G$ non semplice (già visto).
\item Nei casi $24 = 2^3 \cdot 3$, $48 = 2^4 \cdot 3$ e $54 = 3^3 \cdot 2$, $G$ ha un $p$-Sylow ha indice $2$ o $3$ in $G$, e quindi $G$ non semplice grazie alla Proposizione~\ref{prop:IndiceTra2e4}.
%\item Resta il caso $n=p^iq$ con $i>2$ e $q\ge5$ $\implies$ $p^i<60/5=12$ $\implies$ $p=2$, $i=3$ $\implies$ $q<60/2^3<8$ $\implies$ $q=5$ o $q=7$ $\implies$ $n=2^3\cdot5=40$ o $n=2^3\cdot7=56$.
\item Se $n= 40 = 3^2 \cdot 4$, allora per il {\scshape Teorema di Sylow II} si ha $s_5=1$ e neanche in questo caso $G$ è semplice.
\item Rimane solo $n = 56 = 2^3 \cdot 7$ a questo punto. Il ragionamento non è diverso da quello per dimostrare le Proposizioni~\ref{prop:GruppiPQ}, \ref{prop:GruppiP2Q} e ~\ref{prop:GruppiPQR}. Esercizio! \qedhere
\end{itemize}
\end{proof}

%\section{Gruppi di ordine 56}
%
%$\card{G}=56$ $\implies$ $s_7\equiv1\mod7$ e $s_7\mid8$ $\implies$ $s_7=1$ ($\implies$ $G$ non semplice) o $s_7=8$ $\implies$
%\[
%  T:=\{a\in G : \ord(a)=7\}
%\]
%tale che $\card{T}=s_7(7-1)=8\cdot6=48$. \\
%$H$ $2$-Sylow di $G$ $\implies$ $H\subseteq G\setminus T$, $\card{H}=8=\card{(G\setminus T)}$ $\implies$ $H=G\setminus T$ $\implies$ $s_2=1$ $\implies$ $G$ non semplice.