% !TEX program = lualatex
% !TEX spellcheck = it_IT

% !TEX root = ./campi.tex
% !TEX program = lualatex

\documentclass[ structure      = article
              , maketitlestyle = standard
              , twocolcontents = toc
              , secstyle       = center
              , secfont        = roman
              , liststyle      = aligned
              , quotesize      = normalsize
              , headerstyle    = center
              , footnotestyle  = dotted
              ]{suftesi}

\usepackage[no-math]{fontspec}
\usepackage[rm,tt=false]{libertine}
\usepackage{polyglossia}
\setmainlanguage{italian}
\setotherlanguage{english}
\setotherlanguage{german}
\usepackage{microtype}

\usepackage{hyperref}
\hypersetup{ breaklinks
           , colorlinks
           , allcolors = blue!50!black
           }

\usepackage[autostyle,italian=quotes]{csquotes}
\usepackage[ bibstyle     = alphabetic
           , citestyle    = alphabetic
           , pluralothers = true
           , autolang     = langname
           ]{biblatex}
\addbibresource{biblio.bib}
\nocite{*}

\usepackage{enumitem}
\setlist[enumerate]{ref=(\arabic*)}

\usepackage{graphicx}
\usepackage{booktabs}

\usepackage{libertinust1math}
\usepackage{MnSymbol}
\usepackage[bb=ams]{mathalfa}
\usepackage{mathtools}
\let\underbrace\LaTeXunderbrace
\let\overbrace\LaTeXoverbrace

\usepackage{amsthm}
\newcounter{thmcounter}
\counterwithin{thmcounter}{section}

\theoremstyle{definition}
\newtheorem{defi}[thmcounter]{Definizione}
\newtheorem{cons}[thmcounter]{Costruzione}
\newtheorem{lemm}[thmcounter]{Lemma}
\newtheorem{prop}[thmcounter]{Proposizione}
\newtheorem{teor}[thmcounter]{Teorema}
\newtheorem{coro}[thmcounter]{Corollario}
\newtheorem{osse}[thmcounter]{Osservazione}
\newtheorem{rich}[thmcounter]{Richiamo}
\newtheorem{esem}[thmcounter]{Esempio}
\newtheorem{eser}[thmcounter]{Esercizio}


\numberwithin{equation}{section}

\usepackage{tikz}
\usetikzlibrary{ calc
               , babel
               , cd
               , patterns
               , decorations.pathreplacing
               }
\tikzset{>={To[length=3pt,width=3pt]}}
\tikzcdset{ row sep/normal=.7cm
          , column sep/normal=.7cm
          , arrow style=tikz
          , diagrams={>={To[length=3.5pt,width=4pt]}}
          , shorten=-2pt
          }

\newcommand\im{\operatorname{im}}
\newcommand\iso{\cong}
%\newcommand\niso{\ncong}
\newcommand\all{\forall\,}
\newcommand\exi{\exists\,}
\newcommand\exiun{\exists!\,}
\newcommand\id{\mathrm{id}}
\newcommand\st{\,:\,}
\newcommand\comp{\circ}
\newcommand\mor[1]{\xrightarrow{\,#1\,}}
\newcommand\rat{\dashrightarrow}
\newcommand\card[1]{\left\lvert#1\right\rvert}
\newcommand\rest[1]{|_{#1}}
\newcommand\abs[1]{\lvert#1\rvert}
\newcommand\N{\mathbb{N}}
\newcommand\Z{\mathbb{Z}}
\newcommand\Q{\mathbb{Q}}
\newcommand\R{\mathbb{R}}
\newcommand\C{\mathbb{C}}
\newcommand\F{\mathbb{F}}
\newcommand\Aut{\operatorname{Aut}}
\newcommand\mcd{\operatorname{mcd}}
\newcommand\mcm{\operatorname{mcm}}
\renewcommand\gcd{\operatorname{mcd}}
\newcommand\ord{\operatorname{ord}}
\newcommand\cl[1]{\mspace{1mu}\overline{\mspace{-1mu}#1\mspace{-1mu}}\mspace{1mu}}
\newcommand\dvd{\mid}
\newcommand\ndvd{\nmid}
\newcommand\primi{\mathcal{P}}
\newcommand\gen[1]{\left\langle#1\right\rangle}
\renewcommand\bar\overline
\newcommand\normal{\triangleleft}
\newcommand\car{\operatorname{char}}
\newcommand\polmin{\mathrm{m}}
\newcommand\calg[2][]{\overline{#2}^{_{#1}}}
\newcommand\Gal[2][]{\operatorname{Gal}_{#1}\!\left(#2\right)}
\newcommand\Fro{\mathcal{F}}
\newcommand\notimplies{\not\implies}

%%%

\renewcommand\implies\Rightarrow
\renewcommand\impliedby\Leftarrow
\renewcommand\iff\Leftrightarrow
\newcommand\sgn{\operatorname{sgn}}

\newcommand\tsumura[1]{%
  Problema #1 di~\cite{tsumura:exercises}%
}

%\usepackage{xcolor}
\newcommand\nota[1]{\textcolor{red}{[#1]}}


\title{Algebra 2 --- Teoria dei Campi}
\author{}
\date{}


\begin{document}

\maketitle

\begin{abstract}
Queste pagine partono dalle \textenglish{slides} del corso di {\scshape Algebra 2} tenuto negli anni 2020/2021 dal prof. Alberto Canonaco (\url{alberto.canonaco@unipv.it}), che si possono ancora reperire all'indirizzo \url{https://www-dimat.unipv.it/canonaco/2020-2021/alg2.html}. Tuttavia è bene tenere conto che:
\begin{itemize}[leftmargin=*]
\item Ci sono integrazioni con note e corsi degli anni successivi. Per questo motivo, la presentazione del materiale ha subito dei cambiamenti e delle dimostrazioni sono state cambiate.
\item Sono stati inseriti alcuni richiami più o meno estesi ad argomenti di {\scshape Algebra 1}.
\item La bibliografia è estesa.
\end{itemize}
Importante: qua e là c'è ancora qualche retaggio delle vecchie slides e a volte il discorso può essere troncato nel mezzo. Spesso le notazioni sono incoerenti. Eventuali errori sono da attribuire a chi sta mantenendo queste note e sta facendo integrazioni.
\end{abstract}

\printbibliography[title=Testi di riferimento]

\clearpage

\tableofcontents

\clearpage


\section{Caratteristica di un anello}

In queste pagine gli anelli sono tutti dotati di identità moltiplicativa e gli omomorfismi di anelli preservano questi elementi. %Ricordiamo anche un fatto apparentemente innocuo come il seguente.

\begin{lemm}\label{lemm:AnelloIniziale}
Per ogni anello $R$ esiste uno e un solo omomorfismo $\Z \to R$.
\end{lemm}

\begin{proof}
Scriviamo esplicitamente questo omomorfismo:
\[\phi : \Z \to R \,,\ \phi(n) := \begin{cases} \underbrace{\phi(1) + \dots{} + \phi(1)}_{n \text{ volte}} & \text{se } n \ge 0 \\ -\phi(-n) & \text{altrimenti} \end{cases}.\]
Che questo sia effettivamente un omomorfismo e che sia l'unico è facilmente verificabile.
\end{proof}

\begin{defi}
La {\em caratteristica} di un anello $R$ è il numero naturale $\car(R)$ per cui, indicato con $\phi : \Z \to R$ indica l'unico omomorfismo di anelli, si ha
\[\car(R)\Z = \ker \phi .\]
\(\Z\) è un dominio ad ideali principali: per questo, si può definire la caratteristica di $R$ come il generatore $\ge 0$ dell'ideale $\ker \phi$. In alcuni libri potreste trovare definita la caratteristica di $R$ come proprio l'ideale $\ker \phi$.
\end{defi} 

\begin{esem}
Alcuni esempi:
\begin{itemize}
\item $\car(\Z) = 0$. Infatti un omomorfismo $\Z \to \Z$ è l'identità: a causa del Lemma~\ref{lemm:AnelloIniziale}, questo è l'unico che può esserci. Così stando le cose, il nucleo è banale.
\item $\car(\Z/n\Z)=n$ con $n \ge 1$. La proiezione al quoziente $\Z \to \Z/n\Z$: di nuovo a causa del Lemma~\ref{lemm:AnelloIniziale}, è l'unico che può esserci. Il nucleo di questo omomorfismo è $n\Z$.
\end{itemize}
\end{esem}

%\begin{osse}
Per il {\scshape Primo Teorema di Isomorfismo} si ha che 
\[\im\left(\Z \mor \phi R\right) \iso \frac{\Z}{\car(R)\Z} .\]
Questo vuol dire, ad esempio, che gli anelli a caratteristica $n$ contengono al loro interno una copia isomorfa di $\Z/n\Z$. In particolare, la caratteristica è $0$, l'anello contiene una copia di $\Z$ e quindi è necessariamente infinito.

%\begin{itemize}
%\item Gli anelli a caratteristica $n \ne 0$ contengono al loro interno una copia isomorfa di $\Z/n\Z$. Questo è il caso per esempio degli anelli di cardinalità finita.
%\item Gli anelli a caratteristica $0$ contengono una copia di $\Z$ al loro interno invece. In realtà, scopriremo breve che per certi anelli vale di più.
%\end{itemize}
%Quindi se $A$ è un dominio di integrità, allora $\car(A)$ è $0$ oppure un numero primo.

%Ecco qualche prima conseguenza:
%\begin{itemize}
%\item Poiché $\im(f)=\{n_A \st n \in \Z\}=\gen{1_A}$,
%\[
%\car(A)=
%\begin{cases}
%\ord(1_A) & \text{se $\ord(1_A)<\infty$} \\
%0 & \text{se $\ord(1_A)=\infty$}.
%\end{cases}
%\]
%\item $A$ dominio (in particolare campo) $\implies$ $\im(f)$ dominio $\implies$ $\car(A)\Z$ ideale primo $\implies$ $\car(A)$ è $0$ o un numero primo.
%\end{itemize}
%\end{osse}

Vale anche il viceversa: un anello che contiene una copia isomorfa a $\Z/n\Z$ ha caratteristica $n$. E per rendersi conto ciò abbiamo bisogno di un semplice teorema.

%Sotto alcune condizioni, la caratteristica di un anello è davvero qualcosa di \enquote{caratteristico}: se un anello $R$ è contenuto in uno più grande $S$, cioè esiste un omomorfismo iniettivo $R \hookrightarrow S$, allora hanno caratteristiche uguali.

\begin{prop}\label{prop:ConservazioneCaratteristica}
Se $R$ e $S$ sono due anelli e se esiste un omomorfismo iniettvo $i : R \to S$, allora $\car(R)=\car(S)$.
\end{prop}

\begin{proof}
Scriviamo $\phi_R : \Z \to R$ e $\phi_S : \Z \to S$ gli unici omomorfismi che ci possono essere. Ne segue quindi che $\phi_S = i \phi_R$. Se riusciamo a mostrare che i due omomorfismi hanno lo stesso nucleo, allora possiamo concludere. Viceversa, se $x \in \ker \phi_S$, allora $0 = \phi_S(x) = i\left(\phi_R(x)\right)$, da cui $\phi_R(x) = 0$ perché $i$ è iniettiva.
\end{proof}

Osserviamo che la caratteristica non è esattamente un affare di cardinalità. Certo, gli anelli finiti, hanno caratteristica non nulla e gli anelli a caratteristica $0$ sono infiniti. Tuttavia, possiamo farci un semplice esempio in cui la caratteristica un anello sia non nulla e la sua cardinalità infinita.

\begin{esem}
L'anello $\Z/2\Z$ ha caratteristica $2$. Ma anche $\Z/2\Z[X]$ ha caratteristica $2$ grazie all'inclusione $\Z/2\Z \hookrightarrow \Z/2\Z[X]$, e certamente non ha cardinalità finita.
\end{esem}

Noi ci interesseremo di campi da un certo punto in poi: è utile richiamare una proprietà fondamentale allora.

\begin{prop}\label{prop:OmomorfismiCampiSonoIniettivi}
Sia $R$ un anello con divisione e $S$ un anello non banale. Allora ogni omomorfismo $f : R \to S$ è iniettivo.
\end{prop}

\begin{proof}
$\ker f$ è banale. Infatti gli unici ideali di $R$ sono quello banale e $R$ stesso. Poiché $S$ non è banale, $0 \ne 1$ e quindi $1 \notin \ker f$. Pertanto il nucleo non può essere $R$.
\end{proof}

\begin{coro}
Gli omomorfismi di campi sono tutti iniettivi. Se esiste un omomorfismo di campi $K \to L$, allora $K$ e $L$ hanno la stessa caratteristica. Equivalentemente, se due campi hanno caratteristica diversa, non possono esserci omomorfismi tra loro.
\end{coro}


%\section{Campo dei quozienti di un dominio}

Rimandiamo al corso di {\scshape Algebra 1}, la costruzione del {\em campo delle frazioni} $Q(R)$ a partire da un dominio di integrità $R$. A causa dell'omomorfismo iniettivo $R \to Q(R)$ che manda $a$ in $a/1$, possiamo scrivere $a$ al posto di $a/1$. Ricordiamo in particolare come $a/b$ è una classe di equivalenza sotto una certa relazione di equivalenza su $R \times (R \setminus \{0\})$. Questo campo ha una notevole proprietà universale.

%\begin{defi}
%Indichiamo con $Q(A)$ il campo dei quozienti (o delle frazioni) di un dominio $A$. Vediamo $A$ come sottoanello di $Q(A)$ identificando $a\in A$ con $a/1\in Q(A)$.
%\end{defi}

\begin{lemm}\label{lemm:CampoFrazioni}
Siano $R$ un dominio di integrità, $K$ un campo qualsiasi e $f : R \to K$ omomorfismo iniettivo. Allora esiste uno e un solo omomorfismo iniettivo $\tilde f : Q(R) \to K$ per cui commuta
\[\begin{tikzcd}[column sep=tiny]
R \ar[hookrightarrow, dr, swap] \ar["f", rr] & & K \\
 & Q(R) \ar["{\tilde f}", ur, swap]
\end{tikzcd}\]
\end{lemm}

\begin{proof}
Introduciamo immediatamente $\tilde f$:
\[\tilde f(a/b) := f(a)f(b)^{-1} .\]
È un omomorfismo: per ogni $a, c \in R$ e $b, d \in R \setminus \{0\}$ si ha
\begin{align*}
\tilde{f}((a/b)+(c/d)) &= \tilde{f}((ad+bc)/(bd))=f(ad+bc)f(bd)^{-1}= \\
                       &= f(a)f(b)^{-1}+f(c)f(d)^{-1} =\tilde{f}(a/b)+\tilde{f}(c/d) \\
\tilde{f}((a/b)(c/d))  &=\tilde{f}((ac)/(bd))=f(ac)f(bd)^{-1}= \\
                       &= f(a)f(b)^{-1}f(c)f(d)^{-1}=\tilde{f}(a/b)\tilde{f}(c/d) \\
\tilde{f}(1)    &= f(1)=1 .        
\end{align*}
Poiché l'omomorfismo di inclusione è iniettivo, allora i nuclei di $f$ e $\tilde f$ sono uguali: quindi $\tilde f$ è pure iniettivo. L'unicità è praticamente contenuta nella definizione di $\tilde f$.
\end{proof}

Questo lemma è interessante. L'iniezione $f : R \to K$ individua all'interno di $K$ una copia isomorfa a $R$: il lemma dice che se $K$ ha il suo interno una copia di $R$, allora contiene tutto $Q(R)$. L'ovvia applicazione riguarda $\Z$ e $\Q$ e la nozione di caratteristica di anello.

\begin{coro}
Se $K$ è un campo di caratteristica $0$, allora contiene al suo interno una e una sola copia isomorfa a $\Q$. Vale a dire: esiste ed un solo omomorfismo iniettivo $\Q \to K$.
\end{coro}

\begin{proof}
Dal Lemma~\ref{lemm:AnelloIniziale} sappiamo che c'è un unico omomorfismo $\Z \to K$. Da ipotesi questo omomorfismo è iniettivo e per il Lemma~\ref{lemm:CampoFrazioni} esiste esattamente un omomorfismo iniettivo $\Q \to K$. 
%$\exiun f : \Z\to K$ omomorfismo di anelli, e $f$ è iniettivo perché $\car(K)=0$. Per il Lemma $\exiun\tilde{f} : Q(\Z)=\Q\to K$ omomorfismo (iniettivo) di anelli (tale che $\tilde{f}\rest{\Z}=f$).
\end{proof}

Questo teorema è a riepilogo delle considerazioni fatte fino ad ora.

\begin{teor}
Un campo $K$ ha al suo interno una copia isomorfa a $\Z/p\Z$ con $p$ primo (nel qual caso, $p = \car(K)$) oppure a $\Q$ (nel qual caso $0 = \car(K)$).
\end{teor}

\begin{proof}
Per il Lemma~\ref{lemm:AnelloIniziale}, c'è un unico omomorfismo $\phi : \Z \to K$. Se è iniettivo, allora $K$ ha caratteristica $0$. Altrimenti, ha una caratteristica finita e $\im \phi \iso \Z/p\Z$ per qualche $p \ge 1$. Essendo $\Z$ un dominio ad ideali principali e $K$ un campo, necessariamente la $p$ è primo.
\end{proof}

\begin{eser}
Riesci a trovare un campo infinito ma di caratteristica $\ne 0$?
\end{eser}

%Questo teorema è molto interessante: se di un campo $K$ riesco a trovare una copia di $\Z/p\Z$ con $p$ primo, allora ho determinato la caratteristica dell'intero campo, non importa quanto grande sia. Facciamo qualche passo indietro.
%
%%\begin{eser}
%Gli anelli di caratteristica $0$ sono necessariamente infiniti. Gli anelli finiti hanno caratteristica non nulla. Esistono anelli infiniti ma di caratteristica $\ne 0$?
%%\end{eser}
%
%\begin{esem}
%Prendiamo l'anello dei polinomi su $R := \Z/2\Z$, cioè l'anello dei polinomi a coefficienti $0$ oppure $1$. Ora, a causa del Lemma~\ref{prop:ConservazioneCaratteristica}, $R[X]$ ha la stessa caratteristica di $R$. Anche campo delle frazioni $K := Q(R[X])$ ha caratteristica finita, $2$, come abbiamo visto. Tuttavia $K$ non è finito, visto che sicuramente contiene tutti i termini $X^r$ con $r \in \Z$.
%\end{esem}



%\begin{defi}
%Se $K$ è un campo, $F\subseteq K$ è un {\em sottocampo} di $K$ se $F$ è un sottoanello di $K$ e come anello è un campo. \\
%\end{defi}
%Chiaramente $F\subseteq K$ è un sottocampo di $K$ $\iff$
%\begin{itemize}
%\item $1\in F$;
%\item $a,b\in F$ $\implies$ $a-b,ab\in F$;
%\item $a\in F\setminus\{0\}$ $\implies$ $a^{-1}\in F$.
%\end{itemize}
%\begin{esem}
%Le seguenti inclusioni sono sottocampi:
%\begin{itemize}
%\item $\Q\subset\R\subset\C$;
%\item $K\subset K(X):=Q(K[X])$ $\all K$ campo.
%\end{itemize}
%\end{esem}
%\begin{osse}
%$K$ campo, $F_{\lambda}\subseteq K$ sottocampi (con $\lambda\in\Lambda$) $\implies$ $\bigcap_{\lambda\in\Lambda}F_{\lambda}\subseteq K$ sottocampo ({\em esercizio}).
%\end{osse}


%\begin{defi}
%Il {\em sottocampo primo} di un campo $K$ è il pi\`u piccolo sottocampo di $K$, cioè l'intersezione di tutti i sottocampi di $K$.
%\end{defi}


\section{Polinomi e radici}

Se $R$ è un anello, indichiamo con $R[X]$ l'anello dei polinomi nell'indeterminata $X$, dove $X$ è solo un mero simbolo. Indichiamo gli elementi di questo anello come somme formali
\[\sum_{k \in \N} a_k X^k\]
dove $a : \N \to R$ è una successione in cui solo un numero finito di termini $a_k$ è diverso da zero. I polinomi $\sum_{k \in \N} a_k X^k$ in cui $a_k = 0$ per $k \ge 1$ sono identificati con $a_0 \in R$: quindi si potrebbe pensare $R$ come sottoinsieme di $R[X]$.

Richiamiamo anche come sono definite la somma e il prodotto di polinomi di un qualsiasi anello $R[X]$:% Rimandiamo al corso di {\scshape Algebra 1} per i dettagli.
\begin{align*}
& \left(\sum_{k \in \N} a_k X^k\right) + \left(\sum_{k \in \N} b_k X^k\right) := \sum_{k \in \N} \left(a_k+b_k\right) X^k \\
& \left(\sum_{k \in \N} a_k X^k\right) \left(\sum_{k \in \N} b_k X^k\right) := \sum_{k \in \N} \left(\sum_{h=0}^k a_h b_{k-h}\right) X^k
\end{align*}

Il {\em grado} di un polinomio $p := \sum_{k \in \N} a_k X^k \in R[X]$ non nullo è il numero
\[\deg p := \max \left\{k \in \N \mid a_k \ne 0 \right\} .\]
Il grado del polinomio nullo è definito come $-\infty$, anche se non è una convenzione universalmente accettata. È facile verificare che
\[\deg(p + q) = \max\left\{ \deg p, \deg q \right\}\]
e se $R$ è un dominio di integrità allora anche
\[\deg(pq) = \deg p + \deg q .\]

Ora parliamo di anelli commutativi e di anelli di polinomi su anelli commutativi, visto che poi andremo piuttosto rapidamente verso i campi.

\begin{prop}\label{prop:OmomorfismiDaAnelliPolinomi}
Siano $R$ e $S$ due anelli commutativi e $f : R \to S$ un omomorfismo. Allora per ogni $\alpha \in S$ esiste uno e un solo omomorfismo $\tilde f : R[X] \to S$ tale che
\[\begin{tikzcd}
R \ar["f", dr, swap] \ar[hookrightarrow, r] & R[X] \ar["{\tilde f}", d] \\
                                            & S
\end{tikzcd}\]
commuta e $\tilde f (X) = \alpha$.
\end{prop}

\begin{proof}
Il diagramma commutativo già suggerisce come è fatto $\tilde f$:
\[\tilde f \left(\sum_{k \in \N} a_kX^k\right) := \sum_{k \in \N} f\left(a_k\right)\alpha^k\]
(In $\sum_{k \in \N} a_k X^k$ solo un numero finito di $a_k$ è $\ne 0$, quindi $\sum_{k \in \N} f\left(a_k\right) \alpha^k$ è una somma certamente finita.) Questa funzione è un omomorfismo, vediamo.
\begin{align*}
\tilde f \left(\sum_{k \in \N} a_kX^k + \sum_{k \in \N} b_kX^k\right) &= \tilde f \left(\sum_{k \in \N} \left(a_k+b_k\right)X^k\right) =\\
&= \sum_{k \in \N} f\left(a_k+b_k\right)\alpha^k = \\
&= \sum_{k \in \N} f\left(a_k\right)\alpha^k + \sum_{k \in \N} f\left(b_k\right)\alpha^k = \\
&= \tilde f \left(\sum_{k \in \N} a_kX^k\right) + \tilde f \left(\sum_{k \in \N} b_kX^k\right)
\end{align*}
Per vedere che preserva i prodotti, abbiamo bisogno dell'assunzione della commutatività.
\begin{align*}
\tilde f \left(\left(\sum_{k \in \N} a_kX^k\right) \left(\sum_{k \in \N} b_kX^k \right)\right) &= \tilde f \left(\sum_{k \in \N} \left(\sum_{h=0}^k a_hb_{k-h}\right)X^k\right) = \\
&= \sum_{k \in \N} f\left(\sum_{h=0}^k a_hb_{k-h}\right)\alpha^k = \\
&= \sum_{k \in \N} \sum_{h=0}^k f\left(a_h\right) f\left(b_{k-h}\right) \alpha^k = \\
&= \sum_{k \in \N} \sum_{h=0}^k f\left(a_h\right)\alpha^j f\left(b_{k-h}\right)\alpha^{k-h} = \\
&= \left(\sum_{k \in \N} f\left(a_k\right)\alpha^k\right) \left(\sum_{k \in \N} f\left(b_k\right)\alpha^k\right) = \\
&= \tilde f \left(\sum_{k \in \N} a_kX^k\right) \tilde f \left(\sum_{k \in \N} b_kX^k\right)
\end{align*}
Infine, il fatto che preserva l'identità è immediato.
\end{proof}

\begin{defi}[Valutazione di polinomi]\label{prop:ValutazionePolinomi}
Sia $R$ un anello commutativo e $\alpha \in R$. Chiamiamo {\em valutazione} in $\alpha$ l'omomorfismo $R[X] \to R$ di anelli indotto dall'identità $\id_R : R \to R$ nel senso della Proposizione~\ref{prop:OmomorfismiDaAnelliPolinomi}. In tal caso, scriviamo $p(\alpha)$ l'immagine di $p \in R[X]$ sotto l'omomorfismo di valutazione in $\alpha$: cioè se 
\[p = \sum_{j \in \N} a_j X^j ,\]
allora
\[p(\alpha) = \sum_{j \in \N} a_j \alpha^j .\]
\end{defi}

\begin{coro}
Siano $R$ e $S$ due anelli commutativi e $f : R \to S$ un omomorfismo. Allora esiste uno e un solo omomorfismo $f_\ast : R[X] \to S[X]$ tale che commuta
\[\begin{tikzcd}
R \ar[hookrightarrow, d] \ar["f", r] & S \ar[hookrightarrow,d]   \\ %R[X] \ar["{\tilde f}", d] \\
R[X] \ar["{f_\ast}", r, swap] &             S[X]
\end{tikzcd}\]
e $f_\ast(X) = X$. Esplicitamente, se $f : R \to S$ è un omomorfismo di anelli, allora %e $p = \sum_{j \in \N} a_j X^j$, allora
%\[f_\ast (p) = \sum_{j \in \N} f\left(a_j\right) X^j .\]
\[f_\ast \left( \sum_{j \in \N} a_j X^j \right) = \sum_{j \in \N} f\left(a_j\right) X^j .\]
\nota{Scrivere del funtore di $(\phantom\square)_\ast : \mathbf{CRing} \to \mathbf{CRing}$.}
\end{coro}

\begin{coro}
Siano $R$ e $S$ anelli commutativi, $f : R \to S$ un omomorfismo e $\alpha \in S$. Allora l'omomorfismo $\tilde f : R[X] \to S$ della Proposizione~\ref{prop:OmomorfismiDaAnelliPolinomi} è
\[R[X] \to S\,,\ p \mapsto f_\ast(p)(\alpha) .\]
\end{coro}

Da un certo punto in poi parleremo di campi, quindi vediamo subito come si applicano queste cose. Abbiamo già visto che tutti che gli omomorfismi di campi sono iniettivi: quindi, se abbiamo un omomorfismo di campi $i : K \to L$, allora l'immagine di $K$ in $L$ è una copia di $K$. In questo senso, diciamo che $K$ è contenuto in $L$ anche se non è letteralmente un sottoinsieme di $L$. Confondere $K$ con la sua immagine dentro $L$ è un abuso di cui ci gioveremo molto spesso, cercando di essere il più chiari e trasparenti possibile. Inoltre, se $r \in K$, allora indichiamo con $r$ anche l'elemento $i(r)$ di $L$ che corrisponde a $r$. L'abuso si propaga anche sui polinomi: un elemento $p$ di $K[X]$ viene identificato all'elemento $i_\ast(p)$ di $L[X]$, e quindi per evitare troppe parentesi spesso ci riferiremo a quest'ultimo come \enquote{al polinomio $p$ visto come elemento di $L[X]$} o in modi simili. 

\begin{defi}[Radice di un polinomio]
Sia $i : K \to L$ un omomorfismo di campi, $\alpha \in L$ e $p \in K[X]$. Diciamo che $\alpha$ è {\em radice} di $p$ in $L$ qualora $i_\ast (p)(\alpha) = 0$. Cioè, impiegando l'abuso di linguaggio appena spiegato, la radice di un polinomio $p \in K[X]$ in $L$ è un $\alpha \in L$ tale che vedendo $p$ come un elemento di $L[X]$ si ha che sia annulla valutato in $\alpha$.
\end{defi}

\begin{esem}
Consideriamo il polinomio $X^2+1 \in \R[X]$: non ha radici in $\R$, ma li ha in $\C$. Se consideriamo l'inclusione $i : \R \hookrightarrow \C$, allora abbiamo
\[i_\ast \left(X^2+1\right) = X^2+1 .\]
Le radici complesse sono due: $i$ e $-i$.
\end{esem}

\begin{esem}\label{esem:AlgebristaC}
La \enquote{definizione da algebrista} di $\C$ è un'altra però:
\[\C := \frac{\R[X]}{\left(X^2+1\right)}\]
in cui
\[i := X + \left(X^2+1\right) .\]
Vediamo come si inquadrano le cose nella forma delle definizione data. Ora l'omomorfismo
\[i : \R \to \C\,,\ i(r) := r + \left(X^2+1\right) \]
induce l'omomorfismo $i_\ast : \R[X] \to \C[X]$. Usiamo $X^2+1$ per indicare l'immagine di $X^2+1$ sotto $i_\ast$, identificando i coefficienti $a_j$ con le rispettive immagini $a_j + \left(X^2+1\right)$. Verifichiamo che $i$ è radice di $X^2+1$:
\[\left(X + \left(X^2+1\right)\right)^2 +1 = X^2 + 1 + \left(X^2+1\right) = \underbrace{0 + \left(X^2+1\right)}_{\text{lo zero di } \C} .\]
\end{esem}

\begin{defi}[Elementi algebrici e trascendenti]
Sia $i : K \to L$ un'estensione e $\alpha \in L$. Diciamo che $\alpha$ è {\em algebrico} qualora esista qualche $p \in K[X]$ non nullo tale che $i_\ast(p)(\alpha) = 0$. Equivalentemente, $\alpha$ è algebrico qualora l'omomorfismo
\[K[X] \to L\,,\ p \mapsto i_\ast(p)(\alpha)\]
non è iniettivo. Invece diremo che $\alpha$ è {\em trascendente} quando $\alpha$ non è trascendente.
\end{defi}

\nota{Inserire esempi.}

\begin{prop}
Sia $i : K \to L$ un'estensione e $\alpha \in L$ algebrico. Allora esiste uno e un solo $m \in K[X]$ monico e irriducibile tale che sia un generatore nucleo dell'omomorfismo
\[K[X] \to L\,,\ p \mapsto i_\ast(p)(\alpha) .\]
\end{prop}

\begin{proof}
Poiché $K$ è un campo, $K[X]$ è un dominio ad ideali principali. Quindi il nucleo dell'omomorfismo in questione è generato da un certo $m \in K[X]$. Possiamo assumere che sia monico, essendo $K$ un campo. Inoltre l'omomorfismo di valutazione in $\alpha$ induce un omomorfismo iniettivo
\[\frac{K[X]}{\gen m} \to L\]
verso un altro campo: $m$ è pure irriducibile perché $\frac{K[X]}{\gen m}$ è un campo e $K[X]$ è un dominio ad ideali principali. Infine se $m_1, m_2 \in K[X]$ sono generatori monici del nucleo di questo omomorfismo, allora $m_1 = a m_2$ per qualche $a \in K$ invertibile. Essendo entrambi monici, concludiamo che $a = 1$.
\end{proof}

\begin{defi}[Polinomio minimo]
Sia $i : K \to L$ un'estensione di campi e $\alpha \in L$ algebrico. Il {\em polinomio minimo} di $\alpha$ è il generatore monico del nucleo dell'omomorfismo
\[K[X] \to L\,,\ p \mapsto i_\ast(p)(\alpha) .\]
\end{defi}

\begin{prop}\label{prop:EquivalentiPolinomioMinimo}
Sia $i : K \to L$ un omomorfismo di campi, $\alpha \in L$ e $m \in K[X]$ non nullo e monico. Allora sono equivalenti:
\begin{enumerate}
\item $m$ è il polinomio minimo di $\alpha$ su $K$.
\item $m$ è irriducibile su $K$ e $i_\ast (m)(\alpha) = 0$.
\end{enumerate}
\end{prop}

\begin{proof}
Implicazione $(1) \Rightarrow (2)$. Da definizione, $m$ è il generatore monico dell'omomorfismo
\[K[X] \to L\,,\ p \mapsto i_\ast (p)(\alpha) .\]
Quindi $m$ come polinomio in $L[X]$ si annulla in $\alpha$. Inoltre, essendo $K$ e $L$ campi, pure $\frac{K[X]}{\gen m}$ lo è: allora $m$ è irriducibile perché $K[X]$ dominio a ideali principali.\newline
Implicazione $(2) \Rightarrow (1)$. Scriviamo $m'$ il generatore monico del nucleo dell'omomorfismo
\[K[X] \to L\,,\ p \mapsto i_\ast (p) (\alpha) .\]
Quindi $m \in \gen{m'}$. Ma, essendo $m$ irriducibile, abbiamo $m = a m'$ con $a \in L$ invertibile. Trattandosi di polinomi monici, $a = 1$ necessariamente.
\end{proof}

La ricerca del polinomio minimo è quindi ridotta ad una questione di irriducibilità: il lettore è invitato a ripassare i criteri per l'irriducibilità di polinomi fatti ad {\scshape Alegbra 1}. Facciamo degli esempi.

\begin{esem}
Sia il solito omomorfismo $\R \hookrightarrow \C$. Il polinomio minimo di $i \in \C$ è $X^2+1 \in \R[X]$ perché si annulla in $i$ ed è irriducibile in $\R[X]$ (è un polinomio di grado due senza zeri nel campo $\R$). Allo stesso modo, si verifica che $-i$ ha lo stesso polinomio minimo.
\end{esem}

%\begin{esem}
%Consideriamo l'omomorfismo di inclusione $\Q \hookrightarrow \R$ e $\alpha := \frac{1+\sqrt 2}{2} \in \R$. Per trovare un polinomio in $\Q[X]$ che sia il polinomio minimo di $\alpha$ a volte serve un po' di inventiva. Ad esempio:
%\begin{align*}
%& \alpha = \frac{1+\sqrt 5}{2} \\
%& 2 \alpha = 1 + \sqrt 5 \\
%& 2\alpha - 1 = \sqrt 5 \\
%& 4\alpha^2 - 4\alpha + 1 = 5 \\
%& \alpha^2 - \alpha - 1 = 0 .
%\end{align*}
%Quindi un candidato a polinomio minimo è $X^2 - X - 1$. Resta da vedere se è irriducibile: è un polinomio di grado 2 a coefficienti in un campo senza zeri in quel campo.
%\end{esem}

\begin{esem}
Consideriamo l'omomorfismo di inclusione $\Q \hookrightarrow \R$ e $\alpha := \sqrt{\sqrt[3]{4} -1} \in \R$. Per trovare un polinomio in $\Q[X]$ che sia il polinomio minimo di $\alpha$ a volte serve un po' di inventiva. Ad esempio:
\begin{align*}
& \alpha = \sqrt{\sqrt[3]{4} -1} \\
& \alpha^2 + 1 = \sqrt[3]{4} \\
& \alpha^6 + 3\alpha^4 + 3\alpha^2 + 1 = 4 \\
& \alpha^6 + 3\alpha^4 + 3\alpha^2 - 3 = 0 .
\end{align*}
Quindi un candidato a polinomio minimo è $X^6 + 3X^4 + 3X^2 - 3$. Per vedere se è irriducibile possiamo usare il {\em Criterio di \textgerman{Eisenstein}}: $3$ non divide il coefficiente direttivo, divide tutti gli altri e $3^2$ non divide il termine noto.
\end{esem}

\begin{esem}
Sia $i : K \to L$ un omomorfismo di campi e $\alpha \in L$. Supponiamo che anche $\alpha \in K$. Tecnicamente parlando questo è un piccolo abuso: quello che vogliamo dire è che $\alpha$ appartiene all'immagine di $K$ in $L$ tramite $i$, ovvero $\alpha = i\left(\alpha'\right)$ per un unico $\alpha' \in K$. Calcoliamo il polinomio minimo di $\alpha$. Consideriamo l'omomorfismo
\[K[X] \to L \,,\ p \mapsto i_\ast (p)(\alpha)\]
e calcoliamone il nucleo. Se $p \in K[X]$ è tale che
\[0 = i_\ast(p)(\alpha) = i_\ast(p)\left(i\left(\alpha'\right)\right) = i\left(p\left(\alpha'\right)\right)\]
allora per l'iniettività di $i$ si ha
\[p\left(\alpha'\right) = 0 .\]
Concludiamo quindi che il polinomio minimo di $\alpha = i\left(\alpha'\right)$ è $X-\alpha'$. Con un abuso di notazione, possiamo dire che il polinomio minimo di $\alpha \in K$ è $X-\alpha$. È un abuso che nemmeno si nota nel caso in cui l'omomorfismo è una semplice inclusione insiemistica.
\end{esem}

\begin{esem}
Consideriamo l'omomorfismo di inclusione $\R \hookrightarrow \C$. Abbiamo da poco visto che il polinomio minimo di $\alpha \in \R$ è di primo grado, $X-\alpha$. Sia quindi $\alpha \in \C \setminus \R$. Chiaramente il polinomio minimo di $\alpha$ deve essere di grado $\ge 2$. Costruiremo il polinomio minimo di $\alpha$. Indicando con $\bar\alpha$ il coniugato di $\alpha$, si verifica immediatamente che $\alpha + \bar\alpha$ e $\alpha \bar\alpha$ sono reali. Il polinomio
\[X^2 - (\alpha + \bar\alpha)X + \alpha \bar\alpha\]
è a coefficienti reali ed ha come radici $\alpha$ e $\bar\alpha$. Trattandosi di un polinomio di grado $2$ che non ha zeri reali, il polinomio è anche irriducibile in $\R[X]$.
\end{esem}

Sotto questo punto di vista, lavorare con gli omomorfismi $\R \hookrightarrow \C$ è poco interessante: i polinomi minimi sono di grado $1$ oppure di grado $2$. Un po' più bizzarri sono gli omomorfismi di campo che partono da $\Q$. Vediamo qualche esempio.

\begin{figure}
\centering
\input{roots-of-one.qtikz}
\caption{Le radici quinte di 1 sul piano di Argand-Gauss}
\end{figure}

\begin{esem}[Radici dell'unità]
Il polinomio $X^n-1$ ha $n$ radici complesse, che possiamo scrivere in forma esponenziale
\[\xi_k := e^{i\frac{2\pi k}n} \quad \text{per } k \in \{0, \dots{}, n-1\}\]
di cui la prima è sicuramente è reale. Se $n$ è dispari, $1$ è l'unica radice reale. Possiamo fattorizzare questo polinomio come
\[X^n - 1 = (X-1) \left(X^{n-1} + \dots{} + X + 1\right)\]
e quindi le radici complesse sono di $X^{n-1} + \dots{} + X + 1$. Ricordiamo che
\begin{quotation}
Se $p \ge 3$ è primo, allora $X^{p-1} + \dots{} + X + 1$ è irriducibile in $\Q[X]$.
\end{quotation}
Quindi se consideriamo l'estensione $\Q \hookrightarrow \C$ data dalla composizione delle inclusioni $\Q \hookrightarrow \R$ e $\R \hookrightarrow \C$, e se $p \ge 3$, allora le radici complesse $\xi_1, \dots{}, \xi_{p-1}$ hanno tutte lo stesso polinomio minimo in $\Q[X]$, cioè $X^{p-1} + \dots{} + X + 1$. Osserviamo invece se l'omomorfismo scelto è $\R \hookrightarrow \C$, allora $X^{p-1} + \dots{} + X + 1$ come polinomio reale non è più irriducibile. Infatti, se $\alpha$ è una delle radici non reali, abbiamo visto che il polinomio minimo di $\alpha$ in $\R[X]$ è
\[X^2 - (\alpha + \bar\alpha)X + \alpha \bar\alpha .\]
e divide $X^{p-1} + \dots{} + X + 1$.
\end{esem}


\section{Estensioni di campi}

Abbiamo visto (Proposizione~\ref{prop:OmomorfismiCampiSonoIniettivi}) che gli omomorfismi di campi sono tutti iniettivi. Presi due campi $K$ e $L$, se esiste un omomorfismo $K \to L$, allora $L$ contiene al suo interno una copia isomorfa a $K$. Quindi, anche se $K$ non è propriamente un sottoinsieme di $L$, possiamo dire che $K$ è contenuto in $L$ oppure che $L$ contiene $K$. In ogni caso, si è scelta una nuova parola per indicare questa inclusione.

\begin{defi}
Un'{\em estensione} (di campi) è un omomorfismo di campi.
\end{defi}

Per abuso di notazione, spesso un'estensione di campi $i : K \to L$ viene indicata semplicemente con $K \subseteq L$, come in \cite{aluffi:algebra}, anche quando non è proprio un'inclusione insiemistica. Esistono altre notazioni: per esempio in \cite{milne:fields} si usa $L/K$ mentre in \cite{leinster:fields} viene impiegato $L:K$. Esiste anche $K \hookrightarrow L$ una combinazione di $\subset$ e $\to$.

Abbiamo già visto alcuni esempi banali di estensioni di campi. Un tipo di estensioni è ispirato all'Esempio~\ref{esem:AlgebristaC}.

\begin{cons}
Se $K$ è un campo, allora $K[X]$ è un dominio ad ideali principali. Se oltre a $K$ abbiamo un $p \in K[X]$ non nullo e irriducibile, allora $\frac{K[X]}{\gen p}$ è un campo. Un'estensione molto naturale quindi è
\[K \to \frac{K[X]}{\gen p}\,,\ r \mapsto r + \gen p .\]
Questa costruzione è molto interessante.
\end{cons}

\begin{prop}
Se $K$ è un campo e $p \in K[X]$ è non nullo e irriducibile, allora sotto l'estensione
\[K \to \frac{K[X]}{\gen p}\,,\ r \mapsto r + \gen p \]
$p$ visto come elemento di $\frac{K[X]}{\gen p}$ ha almeno uno zero.
\end{prop}

Si pensi per esempio a $\R \hookrightarrow \C$ con $X^2+1$: in $\R$ non ci sono radici, ma sicuramente ce n'è qualcuna in $\C = \frac{\R[X]}{\gen{X^2+1}}$. La dimostrazione non è niente di diverso dal conto fatto nell'Esempio~\ref{esem:AlgebristaC}.

\begin{cons}
Sia $i : K \to L$ una estensione di campi e $\alpha \in L$ con polinomio minimo $m \in K[X]$. Abbiamo quindi un'estensione
\[K \to \frac{K[X]}{\gen m}\]
come nell'esempio precedente.
%Come abbiamo visto nella Proposizione~\ref{prop:ValutazionePolinomi}, abbiamo l'omomorfismo indotto $\tilde i : K[X] \to L$ tale che $\tilde i (r) = r$ per ogni $r \in K$ e $\tilde i (X) = \alpha$. (Qui avremmo dovuto scrivere più correttamente $\tilde i(r) = i(r)$, ma abbiamo impiegato l'abuso di cui abbiamo parlato: $r$ è propriamente un elemento di $K \subseteq K[X]$ ma sotto l'omomorfismo iniettivo $i : K \to L$ useremo lo stesso simbolo per riferirci all'elemento di $L$ a cui $r$ viene identificato sotto $i$.) Ora, a causa del {\scshape Primo Teorema di Isomorfismo}, l'omomorfismo $\tilde i : K[X] \to L$ induce un certo omomorfismo iniettivo \[\frac{K[X]}{\ker \tilde i} \to L .\]
%Essendo $K$ un campo, $K[X]$ è un dominio a ideali principali, quindi il nucleo di $\ker \tilde i$ è generato da una qualche $p \in K[X]$. Supponiamo che questo nucleo non sia banale: questo significa che $p$ è un polinomio non nullo che valutato in $\alpha$ è uguale $0$. Inoltre $L$ è un campo, e quindi lo deve essere anche $\frac{K[X]}{\gen p}$. Ne deduciamo che $p$ deve essere anche irriducibile. Pertanto riepilogando:
%\begin{quotation}
%un'estensione $K \to L$ e un $\alpha \in L$ per cui esiste $p \in K[X]$ monico e irriducibile che si annulla in $\alpha$ inducono l'estensione 
%\[K \to \frac{K[X]}{\gen p}\,,\ r \mapsto r + \gen p .\]
%\nota{Discutere su questo.}
%\end{quotation}
\end{cons}

Un altro modo di avere estensioni di campi a partire da un'estensione $i : K \to L$ e da $\alpha \in L$ è il seguente.

\begin{cons}[Estensioni generate]
Sia $i : K \to L$ un'estensione di campi e $S$ un sottoinsieme qualunque di $L$. Definiamo $K(S)$ come il più piccolo sottocampo di $L$ che contiene sia $K$ che $S$. Un piccolo abuso qui: tecnicamente $K(\alpha)$ è il più piccolo sottocampo di $L$ contenente sia l'immagine di $K$ tramite $i$ che $S$. Nel caso in cui $S$ sia un singoletto $\{\alpha\}$, allora scriviamo $K(\alpha)$ al posto di $K(\{\alpha\})$. Quindi un'ovvia estensione è data da $i : K \to L$ stessa:
\[K \to K(S)\,,\ r \mapsto i(r) .\]
Una classe importante di estensioni, ovviamente, sono quelle in cui $S$ è un insieme finito. Hanno un un nome.
\end{cons}

\begin{defi}[Estensioni finitamente generate]
Un'estensione $i : K \to L$ è detta {\em finitamente generata}, qualora esiste $S \subseteq L$ finita tale che $L = K(S)$. Nel caso in cui $S = \{\alpha\}$, l'estensione $K \to L = K(\alpha)$ è detta {\em semplice}.
\end{defi}

Vediamo qualche esempio di estensione generata che sarà importante anche per il seguito.

\begin{esem}
Sia $K$ un campo e $m \in K[X]$ irriducibile. L'estensione $K \hookrightarrow \frac{K[X]}{\gen m}$ che abbiamo già menzionato è generata. Si vede abbastanza rapidamente. Indichiamo con $K\left(X + \gen m\right)$ il più piccolo sottocampo di $\frac{K[X]}{\gen m}$ contente $K$ (propriamente l'immagine di $i$) e $X + \gen m$. Ora, gli elementi di $\frac{K[X]}{\gen m}$ sono della forma $p + \gen m$ con $p \in K[X]$, cioè combinazioni lineari di $X^k +\gen m$: quindi possiamo concludere che
\[\frac{K[X]}{\gen m} = K\left(X + \gen m\right) .\]
In questo senso, l'estensione $K \hookrightarrow \frac{K[X]}{\gen m}$ è semplice.
\end{esem}

\begin{eser}
Considerando l'inclusione $\Q \subseteq \C$, sai dire se $\sqrt3 \in \Q\left(\sqrt2\right)$?
\end{eser}

\begin{prop}
Sia $i : K \to L$ un'estensione e $\alpha_1, \dots{}, \alpha_n \in L$, con $n \ge 2$. Allora
\[K\left(\alpha_1, \dots{}, \alpha_{n-1}, \alpha_n\right) = K\left(\alpha_1, \dots{}, \alpha_{n-1}\right)\left(\alpha_n\right). \]
\end{prop}

\begin{proof}
Esercizio.
\end{proof}

Cioè le estensioni finitamente generate possono essere introdotte iterativamente a partire dalla costruzione di estensione semplice.

Scopriremo molto presto l'importanza di questa costruzione, anche perché sotto certe ipotesi le estensioni generate hanno una descrizione esplicita molti semplice e maneggevole. 

\begin{defi}[Morfismi di estensioni]
Prendiamo sue estensioni di campo
\[\begin{tikzcd}[column sep=small]
L_1 & & L_2 \\
& K \ar["i", ul] \ar["j", ur, swap] &
\end{tikzcd}\]
Una morfismo di estensioni da $i$ a $j$ è un qualsiasi omomorfismo $f : L_1 \to L_2$ per cui commuta
\[\begin{tikzcd}[column sep=small]
L_1 \ar["f", rr] & & L_2 \\
& K \ar["i", ul] \ar["j", ur, swap] &
\end{tikzcd}\] 
\end{defi}

Per dire più concretamente come sono fatte un certo tipo di estensioni semplici serve un po' di lavoro preliminare.

\begin{prop}\label{prop:IsomorfismoEstensioneGenerataDaElementoAlgebrico}
Sia $i : K \to L$ una estensione e $\alpha \in L$ con polinomio minimo $m \in K[X]$. In precedenza abbiamo visto l'estensione di campi
\[K \hookrightarrow \frac{K[X]}{\gen m}\,,\ r \mapsto r + \gen m .\]
Allora esiste una e una sola omomorfismo $f : \frac{K[X]}{\gen m} \to L$ tale che
\[\begin{tikzcd}[column sep=small]
\frac{K[X]}{\gen m} \ar["f", rr] & & L \\
& K \ar[hookrightarrow, ul] \ar["i", ur, swap] &
\end{tikzcd}\]
commuta e $f(X+ \gen m) = \alpha$. In particolare, $f$ ha immagine $K(\alpha)$ e quindi
\[\frac{K[X]}{\gen m} \iso K(\alpha) .\]
\end{prop}

\begin{proof}
Grazie al {\scshape Primo Teorema di Isomorfismo}, l'omomorfismo di valutazione in $\alpha$
\[v_\alpha : K[X] \to L \,,\ p \mapsto i_\ast (p)(\alpha)\]
si fattorizza mediante la proiezione al quoziente in questo modo:
\[\begin{tikzcd}[column sep=small]
K[X] \ar["\pi", dr, swap] \ar["{v_\alpha}", rr] & & L \\
& \frac{K[X]}{\gen m} \ar["{\bar v_\alpha}", ur, swap]
\end{tikzcd}\]
Le estensioni di campi dell'enunciato si ottengono componendo $v_\alpha$ e $\pi$ con l'inclusione $K \hookrightarrow K[X]$: la $f$ dell'enunciato è proprio quella che abbiamo indicato qui con $\bar v_\alpha$. Con questa informazione è facile verificare che che $f = \bar v_\alpha$ è un morfismo di estensioni e che manda $X + \gen m$ in $\alpha$.\newline
Rimane da provare l'isomorfismo che coinvolge l'estensione generata da $\alpha$, e per farlo proveremo che $\im f = K(\alpha)$. L'immagine di $f : \frac{K[X]}{\gen m} \to L$ è un sottocampo di $L$ che contiene $K$ e $\alpha \in L$: quindi $K(\alpha) \subseteq \im f$, da definizione di estensione generata. D'altra parte, le immagini di $f$ sono polinomi di grado $< \deg m$ di $K[X]$ valutati in $\alpha$: quindi è anche vero che $\im f \subseteq K(\alpha)$.
\end{proof}

\begin{rich}
Sia $K$ un campo e $p \in K[X]$ non nullo. $K[X]$ è un dominio euclideo e questo significa che gli elementi di $\frac{K[X]}{\gen p}$ sono precisamente le classi laterali
\[g + \gen p \quad\text{con } g \in K[X] \text{ e } \deg g \le \deg p -1 .\]
\end{rich}

Ecco quindi come sono fatte concretamente le estensioni semplici $K \to K(\alpha)$ quando $\alpha$ è algebrico.

\begin{coro}\label{coro:KAlgebricoEsplicito}
Sia $i : K \to L$ una estensione e $\alpha \in L$ con polinomio minimo $m \in K[X]$. Allora
\[K(\alpha) = \left\{ p(\alpha) \mid p \in K[X], \deg p \le \deg m -1 \right\} .\]
\end{coro}

Rimaniamo ancora un po' su quanto detto nella Proposizione precedente.

\begin{coro}
Sia $i : K \to L$ una estensione e $m \in K[X]$ monico e irriducibile. Considera anche l'usuale estensione $K \hookrightarrow \frac{K[X]}{\gen m}$, $r \mapsto r + \gen m$. Allora esiste una biezione
\[\{\alpha \in L \mid \alpha \text{ radice di } m\} \leftrightarrow \left\{\text{omomorfismi di estensioni } \frac{K[X]}{\gen m} \to L\right\} .\]
Cioè: esistono tanti modi di incorporare $\frac{K[X]}{\gen m}$ all'interno di $L$ quante sono le radici di $p$ in $L$.
\end{coro}

\begin{esem}
Consideriamo l'inclusione $\Q \hookrightarrow \C$ e $X^2+1 \in \Q[X]$ che è un polinomio monico e irriducibile. Le radici sono due, $i$ e $-i$, e quindi il quoziente $\frac{\Q[X]}{\gen{X^2+1}}$ ha le seguenti copie all'interno di $\C$: $\Q(i)$ e \(\Q(-i)\). Osserviamo però che $\Q(i)$ e $\Q(-i)$ sono uguali (esercizio), ma questo non conta perché noi stiamo considerando il numero di estensioni $\frac{\Q[X]}{\gen{X^2+1}} \to \C$.
\end{esem}

Quindi quante estensioni $K(\alpha) \to L$ ci sono? Basta rimaneggiare sfruttare l'isomorfismo che abbiamo appena visto:

\begin{coro}\label{coro:NumeroMorfismiEstensioniDaKAlgebrico}
\nota{Riscrivere: introdurre un terzo campo $L'$ e conteggiare il numero di morfismi di estensioni $K(\alpha) \to L'$ invece.} Sia $i : K \to L$ una estensione e $\alpha \in L$ con polinomio minimo $m \in K[X]$.  Allora esiste una biezione
\[\{\text{radici di } m \text{ in } L\} \leftrightarrow \left\{\text{omomorfismi di estensioni } K(\alpha) \to L\right\} .\]
\end{coro}


\section{Grado di estensioni}

Presa un estensione $i : K \to L$, possiamo vedere $L$ come uno spazio vettoriale su $K$. L'operazione interna è l'operazione di addizione di $L$, mentre la moltiplicazione per scalare deve essere introdotta:
\begin{align*}
K \times L &\to L \\
(k, l) &\mapsto i(k)l
\end{align*}
Con un abuso, possiamo identificare $K$ con la sua immagine sotto $i$ in $L$ e quindi scrivere \enquote{$kl$} al posto di \enquote{$i(k)l$}, rendendo così la moltiplicazione per scalare un affare interno a $L$ stesso. È un abuso di notazione così radicato e comodo che anche noi faremo lo stesso facendo attenzione e cercando di essere il più chiari possibile.

%Quindi abbiamo fatto entrare in gioco l'Algebra Lineare. Le funzioni lineari in questo contesto sono molto interessanti. Siano
%\[\begin{tikzcd}[column sep=tiny]
%L_1  & & L_2 \\
%& K \ar["i", ul] \ar["j", ur, swap]  & 
%\end{tikzcd}\]
%due estensioni di uno stesso campo. Abbiamo quindi due spazi vettoriali $L_1$ e $L_2$. Una funzione lineare $f : L_1 \to L_2$ è una funzione tale che
%\[f(i(a)x+i(b)y) = j(a)f(x) + j(b)f(y) \quad\text{per ogni } a,b \in K \text{ e } x,y \in L .\]
%%Con l'abuso di linguaggio menzionato, possiamo scrivere
%%\[f(ax+by) = af(x) + bf(y) \quad\text{per ogni } a,b \in K \text{ e } x,y \in L .\]
%È facile verificare che $f : L_1 \to L_2$ è lineare se e solo se soddisfa
%\begin{align*}
%& f(x + y) = f(x) + f(y) \quad\text{per ogni } x,y \in L \\
%& f(i(a)) = j(a)         \quad\text{per ogni } a \in K
%\end{align*}
%%
%Se adottiamo l'abuso di linguaggio poco fa menzionato, possiamo riscrivere la seconda identità come
%\[f(a) = a \quad\text{per ogni } a \in K\]
%cioè $f$ fissa gli elementi di $K$. 

\begin{defi}
Il {\em grado} di un'estensione di campi $i : K \to L$ è la dimensione $L$ come spazio vettoriale su $K$ e si indica con $[L:K]$. L'estensione si dice {\em finita} qualora la dimensione di $L$ è finita.
\end{defi}

%Se $K\subseteq L$ non è finita, scriveremo semplicemente $[L:K]=\infty$.

Quindi se $i : K \to L$ è un'estensione di grado $n < \infty$, allora esistono degli elementi $\alpha_1, \dots{}, \alpha_n \in L$ che formano una base di $L$ e quindi $L$ come campo vettoriale è isomorfo a $K^n$. Vediamo qualche conseguenza di questo fatto.

\begin{prop}
Siano $F\subseteq K\subseteq L$ sue estensioni consecutive. 
\begin{enumerate}
\item Se $F\subseteq L$ è un'estensione finita, allora anche $F \subseteq K$ e $K\subseteq L$ lo sono
\item Se $\left\{\alpha_1, \dots{}, \alpha_m\right\}$ è una base di $K$ come spazio vettoriale su $F$ e $\left\{\beta_1, \dots{}, \beta_n\right\}$ è una base di $L$ come spazio vettoriale su $K$, allora
\[\left\{\alpha_i \beta_j \mid i = 1, \dots{}, m, j = 1,\dots{}, n \right\}\]
è una base di $L$ come spazio vettoriale su $F$. In particolare, se $F \subseteq K$ e $K\subseteq L$ sono entrambe finite, allora pure $F \subseteq L$ lo è. Inoltre
\begin{equation}
[L:F]=[L:K][K:F] .\label{form:GradoDiComposizioneEstensioni}
\end{equation}
\end{enumerate}
\end{prop}

La formula~\ref{form:GradoDiComposizioneEstensioni} ricorda una proprietà dell'indice dei sottogruppi in un gruppo: vedremo in seguito che è una importante coincidenza.

\begin{proof}
Sia $F \subseteq L$ un'estensione finita. L'estensione $F \subseteq K$ è ovviamente finita perché $K$ è un sottospazio vettoriale di $L$. Inoltre $L$ come spazio vettoriale su $F$ possiede una base $\left\{ \alpha_1, \dots{}, \alpha_n \right\} \subseteq L$ e quindi gli elementi di $L$ possono essere anche ottenute come combinazioni lineari con coefficienti in $K$. Pertanto anche l'estensione $K \subseteq L$ è finita.\newline
%$[K:F]\le[L:F]$ (perché $K$ è un $F$-sottospazio vettoriale di $L$) e $[L:K]\le[L:F]$ (perché ogni insieme di generatori di $L$ su $F$ lo è anche su $K$), dunque basta dimostrare l'ultima affermazione. \\
Viceversa \nota{forse è meglio riscriverla\dots{}} siano $[K:F]=m$ e $[L:K]=n$, cioè $K \iso F^m$ come spazi vettoriali su $F$ e $L \iso K^n$ come spazi vettoriali su $K$ e quindi anche su $F$. Allora 
\[L \iso \left(F^m\right)^n\iso F^{mn}\] 
come spazi vettoriali su $F$. Concludiamo quindi che $[L:F]=mn$.
\end{proof}

%\begin{osse}
%$\{\alpha_1,\dots,\alpha_m\}$ $F$-base di $K$ e $\{\beta_1,\dots,\beta_n\}$ $K$-base di $L$ $\implies$ $\{\alpha_i\beta_j\st i=1,\dots,m\text{ e } j=1,\dots,n\}$ $F$-base di $L$: basta dimostrare che è un insieme di generatori, il che è vero perché ogni elemento di $L$ è della forma $\sum_{j=1}^nb_j\beta_j$ con $b_j=\sum_{i=1}^ma_{i,j}\alpha_i$ e $a_{i,j}\in F$.
%\end{osse}

\begin{osse}
\begin{itemize}
\item $[L:K]$ non va confuso con l'indice di $K<L$.
\item $[L:K]>0$ e $[L:K]=1$ $\iff$ $K=L$.
\end{itemize}
\end{osse}
\begin{esem}
\begin{itemize}
\item $[\C:\R]=2$ perché $\{1,i\}$ è una $\R$-base di $\C$.
\item $[K(X):K]=\infty$ perché $\{X^n\st n\in\N\}\subset K[X]\subset K(X)$ è $K$-linearmente indipendente.
\end{itemize}
\end{esem}


%\section{Estensioni notevoli}

%\begin{itemize}
%\item $B$ anello commutativo, $A\subseteq B$ sottoanello, $U\subseteq B$ $\implies$ \\
%$A[U]$ indica il pi\`u piccolo sottoanello di $B$ contenente $A$ e $U$, cioè l'intersezione di tutti i sottoanelli di $B$ contenenti $A$ e $U$. Inoltre è facile vedere che
%\[
%A[U]=\{f(b_1,\dots,b_n)\st f\in A[X_1,\dots,X_n],\ b_1,\dots,b_n\in U\}.
%\]
%In particolare $A[b]:=A[\{b\}]=\{f(b)\st f\in A[X]\}$ $\all b\in B$.
%\item $K\subseteq L$ estensione, $U\subseteq L$ sottoinsieme $\implies$ \\
%$K(U)$ indica il pi\`u piccolo sottocampo di $L$ contenente $K$ e $U$, cioè l'intersezione di tutti i sottocampi di $L$ contenenti $K$ e $U$. Chiaramente $K[U]\subseteq K(U)$ e è facile vedere che
%\[
%K(U)=\left\{\alpha\beta^{-1}\st\alpha,\beta\in K[U],\ \beta\ne0\right\}\iso Q(K[U])
%\]
%(l'inclusione $K[U]\to K(U)$ si estende a un omomorfismo iniettivo $Q(K[U])\to K(U)$, che è anche suriettivo). \\
%Ovviamente $K\subseteq K(U)$ è un'estensione, detta {\em generata da $U$} (su $K$).
%\end{itemize}
%
%\begin{defi}
%Un'estensione $K\subseteq L$ è {\em finitamente generata} se $\exi U\subseteq L$ finito tale che $L=K(U)$. L'estensione è {\em semplice} se $\exi\alpha\in L$ tale che $L=K(\alpha):=K(\{\alpha\})$.
%\end{defi}
%
%\begin{esem}
%Un'estensione $K\subseteq L$ è semplice se $[L:K]=p$ primo: date estensioni $K\subseteq K'\subseteq L$, da
%\[
%p=[L:K]=[L:K'][K':K]
%\]
%segue $[L:K']=1$ e $[K':K]=p$ o $[L:K']=p$ e $[K':K]=1$, e quindi $K'=L$ o $K'=K$. \\
%Allora $L=K(\alpha)$ $\all\alpha\in L\setminus K$.
%\end{esem}



%\section{Elementi algebrici e elementi trascendenti}

%\begin{defi}
%$K\subseteq L$ estensione. Si dice che $\alpha\in L$ è {\em algebrico su $K$} se $\exi0\ne f\in K[X]$ tale che $f(\alpha)=0$. \\
%Altrimenti si dice che $\alpha$ è {\em trascendente su $K$}.
%\end{defi}
%
%\begin{prop}
%$K\subseteq L$ estensione, $\alpha\in L$.
%\begin{enumerate}
%\item $\alpha$ trascendente su $K$ $\implies$ $K[\alpha]\iso K[X]$ e $K(\alpha)\iso K(X)$ come $K$-algebre.
%\item $\alpha$ algebrico su $K$ $\implies$ $\exiun\polmin_{\alpha}=\polmin_{\alpha,K}\in K[X]$ monico (detto {\em polinomio minimo} di $\alpha$ su $K$) tale che
%\[
%\{f\in K[X]\st f(\alpha)=0\}=(\polmin_{\alpha}).
%\]
%Inoltre $\polmin_{\alpha}$ è irriducibile in $K[X]$ e $K[\alpha]=K(\alpha)\iso K[X]/(\polmin_{\alpha})$ come $K$-algebre.
%\end{enumerate}
%\end{prop}
%
%\begin{proof}
%$g : K[X]\to L$, $f\mapsto f(\alpha)$ è un omomorfismo di $K$-algebre (è l'unico tale che $X\mapsto \alpha$); inoltre $\im(g)=\{f(\alpha)\st f\in K[X]\}=K[\alpha]$ e $\ker(g)=\{f\in K[X]\st f(\alpha)=0\}$.
%\begin{enumerate}
%\item $\ker(g)=\{0\}$ $\implies$ $K[X]\iso\im(g)=K[\alpha]$ (come $K$-algebre), quindi anche $K(X)=Q(K[X])\iso Q(K[\alpha])\iso K(\alpha)$.
%\item $\ker(g)$ ideale non nullo di $K[X]$ dominio a ideali principali tale che $K[X]^*=K^*$ $\implies$ $\exiun\polmin_{\alpha}\in K[X]$ monico tale che $\ker(g)=(\polmin_{\alpha})$ $\implies$ per il primo teorema di isomorfismo
%\[
%K[\alpha]=\im(g)\iso K[X]/\ker(g)=K[X]/(\polmin_{\alpha})
%\]
%come anelli, ma è facile vedere che l'isomorfismo (essendo indotto da $g$) è anche di $K$-algebre. \\
%$K[\alpha]\subseteq L$ sottoanello $\implies$ $K[\alpha]\iso K[X]/(\polmin_{\alpha})$ dominio $\implies$ $(\polmin_{\alpha})$ ideale primo non nullo $\implies$ $\polmin_{\alpha}$ irriducibile e $(\polmin_{\alpha})$ ideale massimale $\implies$ $K[\alpha]\iso K[X]/(\polmin_{\alpha})$ campo $\implies$ $K[\alpha]=K(\alpha)$ (dato che in ogni caso $K[\alpha]\subseteq K(\alpha)$). \qedhere
%\end{enumerate}
%\end{proof}

Vediamo ora il grado delle estensioni che abbiamo fino ad ora introdotto.

\begin{esem}
Sia $K$ un campo e $p \in K[X]$ non nullo. Allora $\frac{K[X]}{\gen p}$ è uno spazio vettoriale su $K$ di grado $\deg p$ perché una sua base è
\[\left\{1 + \gen p, X + \gen p, \dots{}, X^{\deg p -1} + \gen p\right\} .\]
\end{esem}

Questo è interessante perché se $p$ è irriducibile, allora abbiamo il grado dell'estensione di campi $K \to \frac{K[X]}{\gen p}$, $r \mapsto r + \gen p$. Ecco come prosegue la cosa grazie alla Proposizione~\ref{prop:IsomorfismoEstensioneGenerataDaElementoAlgebrico}.

\begin{prop}\label{prop:GradoEstensioneKAlgebrico}
Sia $i : K \to L$ un'estensione e $\alpha \in L$ con polinomio minimo $m \in K[X]$. Allora il grado dell'inclusione $K \hookrightarrow K(\alpha)$ è uguale a $\deg m$.
\end{prop}

\begin{proof}
In realtà il Corollario~\ref{coro:KAlgebricoEsplicito} ha già fatto tutto il lavoro: una base di $K(\alpha)$ come spazio vettoriale su $K$ è $\left\{1, \alpha, \dots{}, \alpha^{\deg m -1}\right\}$.
\end{proof}

Abbiamo appena compreso che il calcolo del grado di una estensione $K \hookrightarrow K(\alpha)$ con $\alpha$ algebrico passa per il calcolo del polinomio minimo di $\alpha$. Quindi il lettore deve capire che è necessario una certa familiarità con i criteri di irriducibilità di polinomi.

Non è difficile ora formulare delle condizioni equivalenti all'essere elementi algebrici in termini del grado di un'opportuna estensione. 

\begin{prop}
Sia $i : K \to L$ un'estensione e $\alpha \in L$. Allora sono equivalenti:
\begin{enumerate}
\item $\alpha$ è algebrico su $K$.
\item $K \hookrightarrow K(\alpha)$ è finita. In questo caso $[K(\alpha):K]$ è il grado del polinomio minimo di $\alpha$ su $K[X]$.
\end{enumerate}
\end{prop}

\begin{proof}
Se $\alpha$ è algebrico, allora ammette un polinomio minimo $m \in K[X]$ e quindi siamo nelle ipotesi della Proposizione precedente. Il viceversa richiede un po' di Algebra Lineare. Se $[K(\alpha):K] = n < \infty$, allora sicuramente gli $n+1$ elementi
\[1, \alpha, \dots{}, \alpha^{n-1}, \alpha^n\]
sono linearmente dipendenti. Cioè esistono $a_0, \dots{}, a_n \in K$ non tutti nulli per cui
\[a_0 + a_1 \alpha + \dots{} + a_n \alpha^n = 0 .\]
Abbiamo quindi trovato un polinomio non nullo che sia annulla in $\alpha$.  
\end{proof}


%\begin{proof}
%\begin{itemize}
%\item[$1\implies2$] Per la parte 2 della Proposizione.
%\item[$2\implies3$] $K[\alpha]=K(\alpha)$ campo $\implies$ $K[\alpha]\niso K[X]$ $\implies$ \\
%$\alpha$ algebrico su $K$ per la parte 1 della Proposizione $\implies$ $K(\alpha)\iso K[X]/(\polmin_{\alpha})$ per la parte 2 della Proposizione $\implies$ $[K(\alpha):K]=\dim_K(K[X]/(\polmin_{\alpha}))=\deg(\polmin_{\alpha})$ per il Lemma.
%\item[$3\implies1$] Per la parte 1 della Proposizione, dato che $\dim_K(K(X))=\infty$.
%\end{itemize}
%\end{proof}
%
%
%
%\begin{proof}
%\begin{itemize}
%\item $d:=\deg(f)$, $K[X]_{<d}:=\{g\in K[X]\st g=0\text{ o }\deg(g)<d\}$ $K$-sottospazio vettoriale di $K[X]$ tale che $\dim_K(K[X]_{<d})=d$ (una base di $K[X]_{<d}$ è $\{X^i\st 0\le i<d\}$).
%\item La funzione
%\[
%\begin{split}
%\psi : K[X] & \to K[X] \\
%g & \mapsto r \text{ con $g=qf+r$, $q\in K[X]$ e $r\in K[X]_{<d}$}
%\end{split}
%\]
%è ben definita e $K$-lineare ({\em esercizio}).
%\item $\im(\psi)= K[X]_{<d}$ (perché $\psi(g)=g$ se $g\in K[X]_{<d}$) e $\ker(\psi)=\{qf\st q\in K[X]\}=(f)$ $\implies$
%\[
%K[X]/(f)=K[X]/\ker(\psi)\iso\im(\psi)=K[X]_{<d}
%\]
%come $K$-spazi vettoriali per il primo teorema di isomorfismo $\implies$ $\dim_K(K[X]/(f))=\dim_K(K[X]_{<d})=d$.\qedhere
%\end{itemize}
%\end{proof}
%
%\begin{osse}
%Una $K$-base di $K[X]/(f)$ è $\{X^i+(f)\st 0\le i<d\}$.
%\end{osse}

\begin{defi}
Un'estensione $K \subseteq L$ è detta {\em algebrica} quando ogni elemento di $L$ è algebrico su $K$.
\end{defi}

\begin{prop}\label{prop:EstensioneFinitaEquivalenti}
Sia $K \subseteq L$ un'estensione. Allora sono equivalenti:
\begin{enumerate}
\item $K \subseteq L$ è finita;
\item $K \subseteq L$ è algebrica e finitamente generata;
\item Esistono $\alpha_1, \dots{}, \alpha_n \in L$ algebrici su $K$ tali che $L = K(\alpha_1,\dots,\alpha_n)$.
\end{enumerate}
\end{prop}

\begin{proof}
Proviamo le implicazioni $1 \implies 2$, $2 \implies 3$ e $3 \implies 1$. 
\begin{itemize}[leftmargin=*]
\item[$1 \implies 2$] Sia $\alpha \in L$. Allora $[K(\alpha):K] \le [L:K(\alpha)] [K(\alpha):K] = [L:K] < \infty$. Ora, poiché $[L:K] = n < \infty$, allora $L$ come spazio vettoriale su $K$ è generato da $n$ elementi linearmente indipendenti $\alpha_1, \dots{}, \alpha_n \in L$. Pertanto $L=K(\alpha_1,\dots,\alpha_n)$ immediatamente dalla definizione di estensione generata.
\item[$2\implies3$] Ovvia.
\item[$3\implies1$] Se $\alpha_i$ è algebrico su $K$, allora $\alpha_i$ lo è anche su $K_i:=K(\alpha_1,\dots,\alpha_{i-1})$. Quindi per ogni $i=1,\dots,n$ si ha
\[[L:K]=\prod_{i=1}^n[K_{i+1}:K_i]<\infty .\qedhere\]
\end{itemize}
\end{proof}

\nota{Come cambia il polinomio minimo su al variare di $F$ in $K \subseteq F \subseteq L$?}
%\begin{osse}
%$K\subseteq K'\subseteq L$ estensioni, $\alpha\in L$ algebrico su $K$ $\implies$ \\
%$\alpha$ algebrico su $K'$ e $[K'(\alpha):K']\le[K(\alpha):K]<\infty$: \\
%$\polmin_{\alpha,K}\in K[X]\subseteq K'[X]$ tale che $\polmin_{\alpha,K}(\alpha)=0$ $\implies$ $\polmin_{\alpha,K'}\dvd\polmin_{\alpha,K}$ in $K'[X]$ $\implies$ $\deg(\polmin_{\alpha,K'})\le\deg(\polmin_{\alpha,K})$.
%\end{osse}

\begin{prop}
Siano $F\subseteq K\subseteq L$ estensioni. Allora $F\subseteq L$ è algebrica se e solo se $F \subseteq K$ e $K \subseteq L$ lo sono.
\end{prop}

\begin{proof}
Una implicazione dovrebbe essere semplice a questo punto. Viceversa, siano $F \subseteq K$ e $K \subseteq L$ algebriche e mostriamo che $[F(\alpha):F] < \infty$ per ogni $\alpha \in L$. Poiché $K \subseteq L$ è algebrica esiste un $p \in K[X]$ non nullo che abbia $\alpha$ come radice: indichiamo con $a_0, \dots{}, a_n \in K$ i coefficienti del polinomio. Quindi $\alpha$ è algebrico su $F\left(a_0, \dots{}, a_n\right)$. Per l'implicazione $3 \implies 1$ della Proposizione precedente, si ha $F \subseteq F\left(a_0, \dots{}, a_n\right)$ è finita. Anche $F\left(a_0, \dots{}, a_n\right) \subseteq F\left(a_0, \dots{}, a_n\right)(\alpha) = F\left(a_0, \dots{}, a_n, \alpha\right)$ è finita. Componendo le due estensioni finite, si ottiene l'estensione finita $F \subseteq F\left(a_0, \dots{}, a_n, \alpha\right)$. Possiamo a questo punto scrivere
\[\underbrace{[F\left(a_0, \dots{}, a_n, \alpha\right) : F]}_{< \infty} = [F\left(a_0, \dots{}, a_n, \alpha\right) : F(\alpha)] [F(\alpha) : F]\]
da cui si ha che $[F(\alpha):F] < \infty$.
\end{proof}


\section{Esercizi}

Un classico esercizio è quello di calcolare il grado di estensioni finitamente generate, cosa che spesso passa per la ricerca di polinomi minimi (quindi criteri di irriducibilità).

\begin{eser}
\begin{enumerate}
\item Mostrare che $[\Q(\sqrt{2},i):\Q]=4$.
\item Mostrare che $[\Q(\sqrt{2},\sqrt{3}):\Q]=4$.
\item Mostrare che $\Q(\sqrt{2}+\sqrt{3})=\Q(\sqrt{2},\sqrt{3})$.
\item Determinare il polinomio minimo di $\sqrt{2}+\sqrt{3}$ in $\Q[X]$.
\end{enumerate}
\end{eser}

\begin{proof}[Svolgimento]
\begin{enumerate}
\item Possiamo considerare le seguenti estensioni consecutive
\[\begin{tikzcd}
\Q \ar[r, hookrightarrow] & \Q\left(\sqrt 2\right) \ar[r, hookrightarrow] &  \Q\left(\sqrt 2\right)(i) = \Q\left(\sqrt 2, i\right)
\end{tikzcd}\]
ed usare la Proposizione~\ref{prop:GradoEstensioneKAlgebrico} per determinare il grado di ciascuna di si queste. Un polinomio razionale irriducibile e monico che ha come radice $\sqrt2$ è $X^2-2$: ecco il polinomio minimo di $\sqrt2$. Cerchiamo ora un $p \in \Q\left(\sqrt2\right) [X]$ monico e irriducibile che abbia $i$ come radice. Il polinomio $X^2-2 \in \Q[X]$ continua ad essere irriducibile pure in $\Q\left(\sqrt2\right)$: poiché $\sqrt 2$ è algebrico su $\Q$, sappiamo che
\[\Q\left(\sqrt2\right) = \left\{ a+b\sqrt2 \mid a, b \in \Q \right\}\]
e qui non si possono trovare le radici di $X^2-2$ visto come elemento di $\Q\left(\sqrt2\right)[X]$. Questo basta per l'irriducibilità, visto che si tratta di un polinomio di grado 2 e a coefficienti in un campo che non ha radici nello stesso campo. Quindi il grado delle estensioni è
\[\begin{tikzcd}
\Q \ar["2", r, hookrightarrow, swap] & \Q\left(\sqrt 2\right) \ar["2", r, hookrightarrow, swap] &  \Q\left(\sqrt 2\right)(i) = \Q\left(\sqrt 2, i\right)
\end{tikzcd}\]
e l'estensione $\Q \subseteq \Q\left(\sqrt 2, i\right)$ è di grado $4$.
Il lettore potrebbe provare invece a considerare le estensioni
\[\begin{tikzcd}
\Q \ar[r, hookrightarrow] & \Q(i) \ar[r, hookrightarrow] & \Q\left(\sqrt 2, i\right)
\end{tikzcd}\]
per risolvere l'esercizio.
%$\polmin_{\sqrt{2}}=X^2-2$ $\implies$ $[\Q(\sqrt{2}):\Q]=2$, $\polmin_i=X^2+1$ $\implies$ $[\Q(i):\Q]=2$. Dunque $l:=[\Q(\sqrt{2},i):\Q]$ tale che $2=\mcm(2,2)\dvd l\dvd2\cdot2=4$ $\implies$ $l=2$ o $4$. Per assurdo $l=2$ $\implies$ $[\Q(\sqrt{2},i):\Q(\sqrt{2})]=1$ $\implies$ $\Q(\sqrt{2},i)=\Q(\sqrt{2})$ $\implies$ $i\in\Q(\sqrt{2})\subset\R$, assurdo.
\item Possiamo considerare le estensioni consecutive
\[\begin{tikzcd}
\Q \ar[r, hookrightarrow] & \Q\left(\sqrt 2\right) \ar[r, hookrightarrow] &  \Q\left(\sqrt 2\right)\left(\sqrt 3\right) = \Q\left(\sqrt 2, \sqrt 3\right)
\end{tikzcd} \]
e provare a fare come nel punto precedente. Conosciamo già il grado della prima estensione, perciò concentriamoci sulla seconda. Un elemento di $\Q[X]$ che ha come radice $\sqrt3$ è $X^2-3$: vediamo se come elemento di $\Q\left(\sqrt2\right)[X]$ continua a essere irriducibile. È un polinomio a coefficienti nel campo $\Q\left(\sqrt2\right)$ di grado $2$, quindi controlliamo se le sue radici sono $\Q\left(\sqrt2\right)$. Ora, poiché $\sqrt2$ è algebrico su $\Q$, possiamo scrivere
\[\Q\left(\sqrt2\right) = \left\{a+b\sqrt2 \mid a, b \in \Q\right\} .\]
Vediamo allora se $\sqrt3 = a+b\sqrt2$ per qualche $a, b \in \Q$: non è il caso perché
\[\sqrt3 = a+b\sqrt2 \Rightarrow 3 = a^2+2b^2+2ab\sqrt2 \Rightarrow \underbrace{\frac{3-a^2-2b^2}{2ab}}_{\in \Q} = \underbrace{\sqrt2}_{\notin \Q} .\]
Possiamo quindi concludere che pure l'estensione $\Q\left(\sqrt 2\right) \subseteq \Q\left(\sqrt 2, \sqrt 3\right)$ in esame ha grado $2$.
%\nota{Continua\dots{}}
%Analogamente al punto 1, $[\Q(\sqrt{2},\sqrt{3}):\Q]=2$ o $4$. Per assurdo sia $2$ $\implies$ $\sqrt{3}\in\Q(\sqrt{2})$. Poiché $\Q(\sqrt{2})\iso\Q[X]/(X^2-2)$, che ha come $\Q$-base $\{\cl{1},\cl{X}\}$, $\Q(\sqrt{2})$ ha come $\Q$-base $\{1,\sqrt{2}\}$ $\implies$ $\exiun a,b\in\Q$ tali che $\sqrt{3}=a+b\sqrt{2}$ $\implies$ $3=(a+b\sqrt{2})^2=a^2+2b^2+2ab\sqrt{2}$ $\implies$ $a^2+2b^2=3$ e $2ab=0$ $\implies$ $a=0$ e $2b^2=3$ o $b=0$ e $a^2=3$, assurdo.
\item Ovviamente $\alpha := \sqrt2+\sqrt3 \in \Q\left(\sqrt2,\sqrt3\right)$, e quindi $\Q(\alpha)\subseteq\Q\left(\sqrt2,\sqrt3\right)$. Per dimostrare l'inclusione inversa, basta verificare che $\sqrt2,\sqrt3 \in \Q(\alpha)$. Possiamo per esempio ragionare così: l'inversa di $\alpha$ è $\alpha^{-1} = -\sqrt2 + \sqrt3$ e possiamo scrivere
\[\sqrt2 = \frac{\alpha - \alpha^{-1}}{2} \quad\text{e}\quad \sqrt3 = \frac{\alpha+\alpha^{-1}}{2} .\]
E questo basta per concludere che $\sqrt 2, \sqrt3 \in \Q(\alpha)$.
%Possiamo fare le seguenti manipolazioni algebriche senza uscire da $\Q(\alpha)$:
%\begin{align*}
%& \alpha^2=5+2\sqrt{6}\in\Q(\alpha) \\
%& \sqrt{6}=(\alpha^2-5)/2\in\Q(\alpha) \\
%& \alpha\sqrt{6}=2\sqrt{3}+3\sqrt{2}\in\Q(\alpha) \\
%& 2\sqrt{3}+3\sqrt{2}-2\alpha=\sqrt{2}\in\Q(\alpha) \\
%& \alpha-\sqrt{2}=\sqrt{3}\in\Q(\alpha) .
%\end{align*}
%\item Come visto nel punto 3, $2\sqrt{6}=\alpha^2-5$ $\implies$
%\[
%24=(\alpha^2-5)^2=\alpha^4-10\alpha^2+25
%\]
%$\implies$ $\alpha^4-10\alpha^2+1=0$ $\implies$ $\alpha$ è radice di $f:=X^4-10X^2+1\in\Q[X]$ $\implies$ $\polmin_{\alpha}\dvd f$. Per i punti 3 e 2
%\[
%\deg(\polmin_{\alpha})=[\Q(\alpha):\Q]=[\Q(\sqrt{2},\sqrt{3}):\Q]=4=\deg(f)
%\]
%$\implies$ $\polmin_{\alpha}$ e $f$ sono associati $\implies$ $\polmin_{\alpha}=f$ perché sono entrambi monici.
\item Come al solito cerchiamo prima di tutto un polinomio che abbia come radice $\alpha := \sqrt2 + \sqrt3$.
\begin{align*}
& \alpha^2=5+2\sqrt{6} \\
& (\alpha^2 - 5)^2 = 24 \\
& \alpha^4 -10\alpha^2+1 = 0 .
\end{align*}
Quindi ecco un possibile polinomio minimo: $X^4 -10X^2+1 = 0$. I possibili candidati a radici sono $1$ e $-1$, ma nessuno tra questi lo è. Quindi se $X^4 -10X^2+1 = 0$ è riducibile, allora deve essere fattorizzabile in due polinomi di grado $2$. Nemmeno questa è una possibilità perché $Y^2-10Y+1 = 0$ è un polinomio di grado $2$ a coefficienti in $\Q$ che non ha radici in $\Q$.\qedhere
\end{enumerate}
\end{proof}

\begin{eser}[Estensioni di $\Q$]
\begin{enumerate}
\item Mostrare che  per ogni $n\ge 1$ esiste $\alpha \in \R$ tale che $[\Q(\alpha):\Q] = n$.
%\item $[\calg{\Q}:\Q]=\infty$.
\item $[\R:\Q] = [\C:\Q]=\infty$.
%\item L'estensione $\Q\subseteq\calg{\Q}$ non è finitamente generata.
\end{enumerate}
\end{eser}

\begin{proof}[Svolgimento]
\begin{enumerate}
\item L'idea è di trovare degli $\alpha_n \in \R$ radici di polinomi irriducibili $p_n \in \Q[X]$ tali che $\deg p_n = n$. Infatti, grazie alla Proposizione~\ref{prop:GradoEstensioneKAlgebrico} si ha $[\Q\left(\alpha_n\right) : \Q] = \deg p_n$. Ad esempio, i numeri $\alpha_n := \sqrt[n]{2}$ hanno rispettivamente come polinomio minimo $X^n-2 \in \Q[X]$. La verifica che questi polinomi siano tutti irriducibili è lasciato come esercizio.
%\item $\all n>0$, dato $\alpha$ come nel punto 1, $\alpha\in\calg[\C]{\Q}=\calg{\Q}$ $\implies$ $\Q\subseteq\Q(\alpha)\subseteq\calg{\Q}$ estensioni $\implies$ $[\calg{Q}:\Q]\ge[\Q(\alpha):\Q]=n$.
\item Se l'estensione $\Q \to \R$ fosse di grado $n$, allora abbiamo visto che si può costruire una estensione $\Q \subseteq \Q(\alpha)$ di grado $n+1$, ma ciò non è possibile. Il fatto che pure $\Q \subseteq \C$ è di grado infinito segue immediatamente. \qedhere
%$\Q \subseteq\calg{\Q}\subseteq\C$ estensioni $\implies$ $[\C:\Q]\ge[\calg{Q}:\Q]=\infty$. \\
%$\Q\subseteq\R\subseteq\C$ estensioni, $[\C:\R]=2$, per assurdo $[\R:\Q]=n<\infty$ $\implies$ $[\C:\Q]=2n<\infty$, assurdo.
%\item è algebrica ma non finita (per il punto 2).
\end{enumerate}
\end{proof}

\begin{eser}[\href{https://yutsumura.com/cubic-polynomial-x3-2-is-irreducible-over-the-field-qi/}{Problema 399} di \cite{tsumura:exercises}]
Dimostra che $X^3-2$ è irriducibile sul campo $\Q(i)$.
\end{eser}

\begin{proof}[Svolgimento]
È un polinomio di grado $3$ a coefficienti in un campo: basta quindi far vedere che non ha radici in quel campo. Sia $\alpha \in \Q(i)$ una qualsiasi delle radici di $X^3-2$. In particolare si ha l'inclusione $\Q(\alpha) \subseteq \Q(i)$ e si può scrivere
\[[\Q(i):\Q(\alpha)][\Q(\alpha):\Q] = [\Q(i):\Q] .\]
Calcoliamo prima quello che siamo immediatamente in grado di fare. $[\Q(\alpha):\Q]$ perché $X^3-2 \in \Q[X]$ è il polinomio minimo di $\alpha$ e $[\Q(i):\Q] = 2$ perché $X^2+1 \in \Q[X]$ è polinomio minimo di $i$. E siamo caduti in un assurdo perché $3 \nmid 2$. 
\end{proof}

\begin{eser}
Siano $K\subseteq K'\subseteq L$ due estensioni e $\alpha \in L$. Sia anche $[K':K]=n$ e $[K(\alpha):K]=m$.
\[\begin{tikzcd}[column sep=small]
& K' \ar[dr, hookrightarrow] \\
K \ar["n", ur, hookrightarrow] \ar["m", hookrightarrow, dr, swap] & & L \\
& K(\alpha) \ar[ur, hookrightarrow]
\end{tikzcd}\]
\begin{enumerate}
\item Dimostrare che $\mcm(m,n) \mid [K'(\alpha):K] \le mn$. (Dunque $[K'(\alpha):K]=mn$ se $\mcd(m,n)=1$.)
\item Far vedere che $[K'(\alpha):K] \nmid mn$ se $K=\Q$, $L=\C$, $K'=\Q(\beta)$ con $\alpha$ e $\beta$ radici distinte di $X^3-2$.
\end{enumerate}
\end{eser}

\begin{proof}[Svolgimento]
\nota{Da riscrivere.}
\begin{enumerate}
\item $m':=[K'(\alpha):K']\le m$, $K\subseteq K'\subseteq K'(\alpha)$ estensioni $\implies$ $l:=[K'(\alpha):K]=[K'(\alpha):K'][K':K]=m'n\le mn$. $K\subseteq K(\alpha)\subseteq K'(\alpha)$ estensioni $\implies$ $m\dvd l$; $n\dvd l=m'n$ $\implies$ $\mcm(m,n)\dvd l$. 
\item $\polmin_{\alpha}=\polmin_{\beta}=X^3-2$ (perché monico e irriducibile in $\Q[X]$) $\implies$ $m=[\Q(\alpha):\Q]=\deg(\polmin_{\alpha})=3$ e analogamente $n=3$. \\
$\omega:=\alpha\beta^{-1}\in\C$ tale che $K'(\alpha)=\Q(\alpha,\beta)=\Q(\beta,\omega)$.\\
$\omega^3=\alpha^3\beta^{-3}=1$ $\implies$ $\omega$ radice di $X^3-1=(X-1)f$ con $f:=(X^2+X+1)$ monico e irriducibile in $\Q[X]$; $\omega\ne1$ $\implies$ $\omega$ radice di $f$ $\implies$ $\polmin_{\omega}=f$ $\implies$ $[\Q(\omega):\Q]=\deg(\polmin_{\omega})=2$. \\
$[K'(\alpha):\Q]=[\Q(\beta,\omega):\Q]=3\cdot2=6\ndvd mn=9$. \qedhere
\end{enumerate}
\end{proof}


\section{Chiusura algebrica}

\nota{Da riscrivere.}

\begin{lemm}
Sia $K\subseteq L$ un'estensione. Allora la {\em chiusura algebrica} di $K$ in $L$
\[\calg[L]{K}:=\{\alpha\in L\st\text{$\alpha$ algebrico su $K$}\}\]
è un sottocampo di $L$. Inoltre l'estensione $K\subseteq\calg[L]{K}$ è algebrica e $\calg[L]{\calg[L]{K}}=\calg[L]{K}$.
\end{lemm}

\begin{proof}
Chiaramente $K\subseteq\calg[L]{K}$ (in particolare $1\in\calg[L]{K}$). \\
$\alpha,\beta\in\calg[L]{K}$ $\implies$ per la Proposizione di prima $K\subseteq K(\alpha,\beta)$ è un'estensione algebrica $\implies$ $\alpha-\beta,\alpha\beta\in K(\alpha,\beta)$ sono algebrici su $K$ $\implies$ $\alpha-\beta,\alpha\beta\in\calg[L]{K}$. \\
Analogamente $0\ne\alpha\in\calg[L]{K}$ $\implies$ $K\subseteq K(\alpha)$ estensione algebrica $\implies$ $\alpha^{-1}\in K(\alpha)$ algebrico su $K$ $\implies$ $\alpha^{-1}\in\calg[L]{K}$. \\
Per definizione l'estensione $K\subseteq\calg[L]{K}$ è algebrica. Analogamente è algebrica l'estensione $\calg[L]{K}\subseteq\calg[L]{\calg[L]{K}}$, e quindi anche $K\subseteq\calg[L]{\calg[L]{K}}$ per la Proposizione precedente. Allora $\calg[L]{\calg[L]{K}}\subseteq\calg[L]{K}$, per cui $\calg[L]{\calg[L]{K}}=\calg[L]{K}$.
\end{proof}

\begin{defi}
Una {\em chiusura algebrica} di un campo $K$ è un'estensione algebrica $K \subseteq \calg{K}$ con $\calg{K}$ algebricamente chiuso.
\end{defi}

\begin{coro}
$K\subseteq L$ estensione con $L$ algebricamente chiuso $\implies$ $K\subseteq\calg[L]{K}$ è una chiusura algebrica di $K$.
\end{coro}

\begin{proof}
$K\subseteq\calg[L]{K}$ estensione algebrica per la Definizione-Proposizione. \\
$\calg[L]{K}$ algebricamente chiuso: $f\in\calg[L]{K}[X]\setminus\calg[L]{K}\subseteq L[X]\setminus L$ $\implies$ $\exi\alpha\in L$ tale che $f(\alpha)=0$ (perché $L$ algebricamente chiuso) $\implies$ $\alpha$ algebrico su $\calg[L]{K}$ $\implies$ $\alpha\in\calg[L]{\calg[L]{K}}=\calg[L]{K}$.  
\end{proof}

\begin{esem}
$\Q\subseteq\calg{\Q}:=\calg[\C]{\Q}$ è una chiusura algebrica di $\Q$. Si dice che $\alpha\in\C$ è {\em algebrico} (risp.\ {\em trascendente}) se $\alpha\in\calg{\Q}$ (risp.\ $\alpha\not\in\calg{\Q}$).
\end{esem}

\begin{lemm}
Sia $K$ un campo. Allora esiste un'estensione $K \to K'$ tale che ogni polinomio non costante a coefficienti in $K$ ha una radice in $K'$.
\end{lemm}

\begin{proof}
Consideriamo l'insieme 
\[U:=\{f \in K[X] \mid f \text{ irriducibile e monico}\}\]
e per ogni $f \in U$ assegniamo un simbolo $X_f$ che svolgerà il ruolo di indeterminata per certi polinomi. Consideriamo infatti l'anello dei polinomi a coefficienti in $K$ e nelle indeterminate $X_f$, con $f \in U$, 
\[A:=K\left[X_f \mid f \in U\right] .\]
\nota{Abbiamo parlato di polinomi in un un numero arbitrario di indeterminate?} L'insieme $I:= \left(f(X_f) \mid f\in U\right)$ è un ideale di $A$. Mostriamo che $I \subsetneq A$: Se fosse $I = A$, allora esisterebbero $f_1, \dots{}, f_n \in U$ distinti e $g_1,\dots,g_n\in A$ tali che
\[ h:=\sum_{l=1}^nf_i\left(X_{f_i}\right)g_i=1. \]
$K\subseteq L$ campo di spezzamento di $\prod_{i=1}^nf_i$ $\implies$ $\all i=1,\dots,n$ $\exi\alpha_i\in L$ tale che $f_i(\alpha_i)=0$. Valutando $h=1$ in
\[
X_f=
\begin{cases}
\alpha_i & \text{se $f=f_i$ per qualche $i=1,\dots,n$} \\
0 & \text{altrimenti}  
\end{cases}
\]
si ottiene $0=1$ in $L$, assurdo. \\
$\exi J\subset A$ ideale massimale tale che $I\subseteq J$ $\implies$ $K':=A/J$ campo e $\pi\rest{K} : K\to K'$ (con $\pi : A\to K'$ proiezione) estensione di campi con la propriet\`a richiesta: dato $f\in K[X]\setminus K$, posso supporre $f\in U$ $\implies$ $f(\pi(X_f))=\pi(f(X_f))=0$ perché $f(X_f)\in I\subseteq J=\ker(\pi)$.
\end{proof}

\begin{teor}
Ogni campo $K$ ha una chiusura algebrica $K\subseteq\calg{K}$.
\end{teor}

\begin{proof}
\begin{itemize}
\item Posto $K_0:=K$, per il Lemma induttivamente $\all n\in\N$ $\exi K_n\subseteq K_{n+1}$ estensione tale che $f\in K_n[X]\setminus K$ $\implies$ $f$ ha una radice in $K_{n+1}$.
\item $L:=\bigcup_{n\in\N}K_n$ campo ({\em esercizio}) tale che $K\subseteq L$ estensione con $L$ algebricamente chiuso: $f\in L[X]\setminus L$ $\implies$ $\exi n\in\N$ tale che $f\in K_n[X]$ $\implies$ $f$ ha una radice in $K_{n+1}\subseteq L$.
\item $\calg{K}:=\calg[L]{K}$ tale che $K\subseteq\calg{K}$ chiusura algebrica di $K$ (gi\`a visto). \qedhere
\end{itemize}
\end{proof}


\section{Campi di spezzamento}

%\section{Radice di un polinomio in un'estensione}
%Dato $f\in K[X]$ (che posso supporre irriducibile e monico), esiste un'estensione $K\subseteq L$ tale che $f$ abbia una radice $\alpha\in L$? \\
%Se la risposta è s\`\i, $f=\polmin_{\alpha,K}$; inoltre posso supporre $L=K(\alpha)$, e in questo caso $L\iso K[X]/(f)$.
%
%\begin{prop}
%$f\in K[X]$ irriducibile e monico, $\pi : K[X]\to L:=K[X]/(f)$ proiezione al quoziente, $\alpha:=\pi(X)\in L$ $\implies$ $\pi\rest{K} : K\to L$ estensione, $L=K(\alpha)$ e $f=\polmin_{\alpha,K}$ (quindi $f$ ha una radice in $L$).
%\end{prop}
%
%\begin{proof}
%$f$ irriducibile $\implies$ $(f)$ ideale massimale $\implies$ $L$ campo. \\
%$\pi\rest{K}$ estensione perché omomorfismo di anelli con $K$ e $L$ campi. \\
%$\pi$ omomorfismo di $K$ algebre tale che $\pi(X)=\alpha$ $\implies$ $\pi(g)=g(\alpha)$ $\all g\in K[X]$ $\implies$ $L=\{g(\alpha)\st g\in K[X]\}=K[\alpha]=K(\alpha)$. \\
%$f$ irriducibile e monico, $f(\alpha)=\pi(f)=0$ $\implies$ $f=\polmin_{\alpha,K}$.
%\end{proof}

In generale, un $f \in K[X]$ non nullo può non avere tutte le radici all'interno del campo $K$. Successivamente abbiamo visto che si può costruire una chiusura algebrica in cui ogni polinomio a coefficienti in quel campo ha radici. Coi campi di spezzamento facciamo un passo indietro: dato $f \in K[X]$, aggiungere al campo $K$ quanto basta per poter scrivere $f$ come prodotto di polinomi di grado $1$. Quindi è una costruzione che parte da un fissato polinomio.

\begin{defi}[Campo di spezzamento]
Sia $K$ un campo e $f \in K[X]$ non nullo. Un {\em campo di spezzamento} di $f$ è un'estensione $i : K \to K_f$ tale che:
\begin{enumerate}
\item $f$ si {\em spezza} su $K_f$, vale a dire esistono $c \in K^\ast$ e $\alpha_1, \dots{}, \alpha_n \in K_f$ tali che 
\[i_\ast (f) = c\prod_{k=1}^n(X-\alpha_k) .\]
Con il solito abuso possiamo pure scrivere $f$ invece di $i_\ast(f)$ a patto di ricordarsi che il polinomio così fattorizzato è visto come elemento di $K_f[X]$, anello in cui si può effettivamente scrivere questa fattorizzazione.
\item $K_f = K\left( \alpha_1, \dots{}, \alpha_n \right)$, ovvero $i : K \to K_f$ è una estensione generata dalle radici di $f$ in $K_f$.
\end{enumerate}
Tecnicamente, un campo di spezzamento è un'estensione $K \to K_f$, ma talvolta si chiama campo di spezzamento anche solo il campo $K_f$.
\end{defi}

%\begin{osse}
%$K\subseteq L$ estensione tale che $f=c\prod_{i=1}^n(X-\alpha_i)\in K[X]$ si spezza su $L$ $\implies$ $\exiun K\subseteq L_0\subseteq L$ estensione tale che $K\subseteq L_0$ è un campo di spezzamento di $f$; inoltre $L_0=K(\alpha_1,\dots,\alpha_n)$. \\
%Infatti $f$ si spezza su $K(\alpha_1,\dots,\alpha_n)$, e se $K\subseteq L'\subseteq L$ è un'estensione tale che $f$ si spezza su $L'$, allora $\alpha_1,\dots,\alpha_n\in L'$ (per l'unicit\`a della fattorizzazione in $L[X]$), per cui $K(\alpha_1,\dots,\alpha_n)\subseteq L'$.
%\end{osse}

%\section{Esistenza del campo di spezzamento}

\begin{esem}
Consideriamo il polinomio $X^2+1 \in \Q[X]$. Come polinomio a coefficienti complessi ha due radici, $i$ e $-i$. Il campo di spezzamento si costruisce aggiungendo le radici, cioè $\Q(i, -i)$. Certo, due generatori sono sovrabbondanti perché si verifica subito che $\Q(i) = \Q(i, -i)$. Sicuramente il polinomio in esame ha radici in $\C$, ma è il campo di spezzamento è definito come il più piccolo che si può ottenere aggiungendo le radici di quel polinomio.
\end{esem}

\nota{Fare altri esempi.}

Abbiamo già visto come dato un qualsiasi polinomio irriducibile $p \in K[X]$, allora in $\frac{K[X]}{\gen p}$ una radice di $p$ è $X + \gen p$. Questo è anche vero per ogni polinomio non nullo, visto che $K[X]$ è un dominio a fattorizzazione unica. Possiamo reiterare questo processo fino ad aggiungere tutte le radici e questo è il risultato del seguente teorema.

\begin{teor}[Esistenza campo di spezzamento]\label{teor:EsistenzaCampoSpezzamento}
Sia $K$ un campo e $f \in K[X]$ non nullo. Allora
\begin{enumerate}
\item Esiste un campo di spezzamento $K \hookrightarrow K_f$ di $f$.
\item $\left[K_f:K\right] \le (\deg f)!$.% \nota{??? $\to$} Inoltre $\deg f \dvd [L:K]$ se $f$ è pure irriducibile in $K[X]$.
\end{enumerate}
\end{teor}

\begin{proof}
Procediamo per induzione sul grado del polinomio $n := \deg f$. Se $n=0$, allora il polinomio è un elemento invertibile e quindi basta prendere $K_f = K$; è ovvio anche che $1 = \left[K_f:K\right] \le (\deg f)!$. Sia ora $n>0$. Scegliamo $g \in K[X]$ irriducibile che divide $f$, esiste poiché $K[X]$ è un dominio a fattorizzazione unica. Possiamo assumere senza perdere nulla anche che $f$ e $g$ siano monici. Sappiamo che un campo che ha sicuramente qualche radice di $g$ e quindi di $f$ è
\[E := \frac{K[X]}{\gen g} .\]
Abbiamo visto anche come $E$ può essere visto come il campo $K(\alpha)$, dove $\alpha$ è una delle radici di $g$ in $E$. Quindi il polinomio $f$ visto come elemento di $E[X]$ è fattorizzabile come
\[f = (X-\alpha_1) f_1 \text{ per qualche } f_1 \in E[X] .\]
Quest'ultimo ha grado $\deg f -1$ e quindi, per induzione esiste un campo di spezzamento $E \subseteq L := E\left(\alpha_2, \dots{}, \alpha_n\right)$ in cui $f_1 = (X-\alpha_2) \cdots{} (X-\alpha_n)$ con $\alpha_2, \dots{}, \alpha_n \in L$. Ecco la fattorizzazione in $L[X]$ di $f$:
\[f = (X-\alpha_1)\cdots{}(X-\alpha_n) .\]
Da costruzione, $L = E\left(\alpha_2, \dots{}, \alpha_n\right) = K\left(\alpha_1, \alpha_2, \dots{}, \alpha_n\right)$. Vediamo il secondo punto: per il passo induttivo possiamo scrivere
\[[L : K] = \underbrace{[L : E]}_{\le (n-1)!} \underbrace{[E : K]}_{= \deg g} \le n! . \qedhere\]
\end{proof}

\nota{Fare anche un esempio concreto per spiegare la dimostrazione sopra?}

%\section{Esistenza della chiusura algebrica}

%\begin{lemm}
%$K$ campo $\implies$ $\exi K\to K'$ estensione tale che ogni polinomio non costante a coefficienti in $K$ ha una radice in $K'$.
%\end{lemm}
%\begin{teor}
%Ogni campo $K$ ha una chiusura algebrica $K\subseteq\calg{K}$.
%\end{teor}
%\begin{proof}
%\begin{itemize}
%\item Posto $K_0:=K$, per il Lemma induttivamente $\all n\in\N$ $\exi K_n\subseteq K_{n+1}$ estensione tale che $f\in K_n[X]\setminus K$ $\implies$ $f$ ha una radice in $K_{n+1}$.
%\item $L:=\bigcup_{n\in\N}K_n$ campo ({\em esercizio}) tale che $K\subseteq L$ estensione con $L$ algebricamente chiuso: $f\in L[X]\setminus L$ $\implies$ $\exi n\in\N$ tale che $f\in K_n[X]$ $\implies$ $f$ ha una radice in $K_{n+1}\subseteq L$.
%\item $\calg{K}:=\calg[L]{K}$ tale che $K\subseteq\calg{K}$ chiusura algebrica di $K$ (gi\`a visto).
%\end{itemize}
%\end{proof}
%
%\begin{proof}
%$U:=\{f\in K[X]\st\text{$f$ irriducibile e monico}\}$, $A:=K[X_f\st f\in U]$.
%\smallskip
%
%$I:=(f(X_f)\st f\in U)\subsetneq A$ ideale: per assurdo $I=A$ $\implies$ $\exi f_1,\dots,f_n\in U$ distinti e $g_1,\dots,g_n\in A$ tali che
%\[
%h:=\sum_{l=1}^nf_i(X_{f_i})g_i=1.
%\]
%$K\subseteq L$ campo di spezzamento di $\prod_{i=1}^nf_i$ $\implies$ $\all i=1,\dots,n$ $\exi\alpha_i\in L$ tale che $f_i(\alpha_i)=0$. Valutando $h=1$ in
%\[
%X_f=
%\begin{cases}
%\alpha_i & \text{se $f=f_i$ per qualche $i=1,\dots,n$} \\
%0 & \text{altrimenti}  
%\end{cases}
%\]
%si ottiene $0=1$ in $L$, assurdo.
%\end{proof}
%
%$\exi J\subset A$ ideale massimale tale che $I\subseteq J$ $\implies$ $K':=A/J$ campo e $\pi\rest{K} : K\to K'$ (con $\pi : A\to K'$ proiezione) estensione di campi con la propriet\`a richiesta: dato $f\in K[X]\setminus K$, posso supporre $f\in U$ $\implies$ $f(\pi(X_f))=\pi(f(X_f))=0$ perché $f(X_f)\in I\subseteq J=\ker(\pi)$.


%\section{Omomorfismi e isomorfismi di estensioni}
%
%\begin{defi}
%$i : K\to L$ e $i' : K\to L'$ estensioni. Un {\em omomorfismo di estensioni di $K$} (o semplicemente un {\em $K$-omomorfismo}) da $i$ a $i'$ è un omomorfismo di $K$-algebre, cioè un omomorfismo di anelli $j : L\to L'$ tale che $i'=j\comp i$. \\
%Un tale omomorfismo è un {\em isomorfismo di estensioni di $K$} (o semplicemente un {\em $K$-isomorfismo}) se $j$ è un isomorfismo.
%\end{defi}
%\begin{osse}
%$i : K\to L$ estensione induce omomorfismo di anelli
%\[
%i : K[X]\to L[X] \qquad f=\sum_{n\ge0}a_nX^n\mapsto i(f)=\sum_{n\ge0}i(a_n)X^n.
%\]
%Spesso si scriver\`a ancora $f$ invece di $i(f)\in L[X]$. Se $\alpha\in L$, l'identificazione tra $f(\alpha)=\sum_{n\ge0}a_n\alpha^n$ e $i(f)(\alpha)=\sum_{n\ge0}i(a_n)\alpha^n$ è coerente con la struttura di $K$-spazio vettoriale su $L$.
%\end{osse}
%
%\begin{prop}
%$K\subseteq K'$ e $i : K\to L$ estensioni, $\alpha\in K'$ algebrico su $K$, $\beta\in L$. Allora esiste un $K$-omomorfismo $j : K(\alpha)\to L$ tale che $j(\alpha)=\beta$ $\iff$ $\polmin_{\alpha,K}(\beta)=0$; inoltre se esiste è unico.
%\end{prop}
%
%\begin{proof}
%$\alpha$ algebrico su $K$ $\implies$ $K(\alpha)=K[\alpha]\iso K[X]/(\polmin_{\alpha,K})$. Dunque va dimostrato che $\exi$ (unico) $K$-omomorfismo $K[X]/(\polmin_{\alpha,K})\to L$ tale che $\cl{X}\mapsto\beta$ $\iff$ $\polmin_{\alpha,K}(\beta)=0$.
%Per il teorema di omomorfismo per anelli dare un omomorfismo di anelli $\cl{\varphi} : K[X]/(\polmin_{\alpha,K})\to L$ equivale a dare un omomorfismo di anelli $\varphi : K[X]\to L$ tale che $(\polmin_{\alpha,K})\subseteq\ker(\varphi)$, e chiaramente $\cl{\varphi}$ è un $K$-omomorfismo $\iff$ $\varphi$  è un omomorfismo di $K$-algebre.
%Poiché $\all\beta\in L$ $\exiun\varphi : K[X]\to L$ omomorfismo di $K$-algebre tale che $\varphi(X)=\beta$, per concludere basta osservare che per un tale $\varphi$ $(\polmin_{\alpha,K})\subseteq\ker(\varphi)$ $\iff$ $0=\varphi(\polmin_{\alpha,K})=\polmin_{\alpha,K}(\beta)$.
%\end{proof}



%\section{Unicità del campo di spezzamento}

\begin{teor}[Unicità campo di spezzamento]\label{teor:UnicitaCampoDiSpezzamento}
Sia $i : K \to K_f$ campo di spezzamento di un $f \in K[X]$ non nullo e $j : K \to L$ estensione. Allora esiste almeno un morfismo di estensioni $h : K_f \to L$
\[\begin{tikzcd}[column sep=small]
K_f \ar["h", rr] & & L \\
& K \ar["i", ul] \ar["j", ur, swap]
\end{tikzcd}\]
se e solo se $f$ si spezza su $L$. In particolare, il campo di spezzamento di un polinomio non nullo è unico a meno di isomorfismi. 
\end{teor}

\begin{proof}
Allora esistono $c \in K^\ast$ e $\alpha_1, \dots{}, \alpha_n \in K_f$, con $n := \deg f$, tali che $i_\ast(f) = c\prod_{l=1}^n(X-\alpha_l)$ e $K_f = K(\alpha_1,\dots,\alpha_n)$.\newline
%\begin{itemize}
%\item[$\implies$] 
Se abbiamo un morfismo di estensioni $h : K_f \to L$, allora
\[j_\ast(f) = (h i)_\ast (f) = \underbrace{h_\ast i_\ast (f) = h_\ast \left( c \prod_{l=1}^n (X-\alpha_l) \right)}_{f \text{ si spezza su } K_f} = h(c) \prod_{l=1}^n \left(X-h\left(\alpha_l\right)\right) .\]
Vediamo il viceversa per induzione. \nota{Rileggere ed espandere alcune parti.} La base dell'induzione $n=0$ funziona perché $K_f \iso K$ e possiamo scegliere $h = j i^{-1}$. Passiamo al passo induttivo. Sia $n > 0$. Il polinomio minimo $m \in K[X]$ di $\alpha_1$ si spezza su $L$, cioè $j_\ast(m)$ si scrive come prodotto di fattori lineari. Se indichiamo con $\beta \in L$ una delle radici di $m$, allora esiste un morfismo di estensioni $k : K(\alpha_1) \to L$ che manda $\alpha_1$ in $\beta$.
\[\begin{tikzcd}[column sep=small]
K_f  & & L \\
K(\alpha_1) \ar["{i_2}", u] \ar["k", urr, swap] \\
& K \ar["{i_1}", ul] \ar["j", uur, swap]
\end{tikzcd}\]
%\item[$\impliedby$] Per induzione su $n$: $n=0$ $\implies$ $K'=K$ e $i'=i$. \\
%$n>0$ $\implies$ $\polmin_{\alpha_1,K}$ si spezza su $L$ (perché $\polmin_{\alpha_1,K}\dvd f$ e $f$ si spezza su $L$) $\implies$ $\exi\beta\in L$ tale che $\polmin_{\alpha_1,K}(\beta)=0$ $\implies$ per la Proposizione $\exi$ $K$-omomorfismo $j : K(\alpha_1)\to L$ (tale che $j(\alpha_1)=\beta$) $\implies$
Poiché $\alpha_1 \in K(\alpha_1)$ è una radice di $i_{1\ast}(f)$, allora
\[i_{1\ast}(f) = (X-\alpha_1) g \text{ per qualche } g \in K(\alpha_1)[X] .\]
%Consideriamo ora il polinomio di grado $n-1$ a coefficienti in $K(\alpha_1)$
%\[ g := \prod_{l=2}^n(X-\alpha_l) .\]
%tale che $\deg(g)=n-1$,
Il polinomio $g$ è di grado $n-1$. Osserviamo come $i_2 : K(\alpha_1) \to K_f$ campo di spezzamento di $g$ e $g$ si spezza su $L$, perché $g$ divide $f$ e $f$ si spezza su $L$. Per induzione esiste un morfismo di estensioni da $i_2$ a $k$ che chiamiamo $h : K_f \to L$. Segue subito che $h$ è un morfismo di estensioni come nell'enunciato.
%\end{itemize}
\end{proof}

%\begin{coro}
%$K$ campo, $f\in K[X]\setminus\{0\}$ $\implies$ un campo di spezzamento di $f$ esiste e è unico a meno di $K$-isomorfismo.
%\end{coro}
%
%\begin{proof}
%Esistenza gi\`a vista. \\
%Se $K\subseteq K'$ e $i : K\to L$ sono due campi di spezzamento di $f$, per il Teorema $\exi$ $K$-omomorfismo $i' : K'\to L$. Sempre per il Teorema $f$ si spezza su $i'(K')\subseteq L$ $\implies$ $i'(K')=L$. 
%\end{proof}
%
%\begin{osse}
%Segue dal Corollario che il grado $[K':K]$ di un campo di spezzamento $K\subseteq K'$ di $f\in K[X]\setminus\{0\}$ dipende solo da $f$.
%\end{osse}

%\begin{teor}
%$K\subseteq L$ estensione algebrica, $i : K\to\calg{K}$ chiusura algebrica di $K$ $\implies$ esiste un $K$-omomorfismo $j : L\to\calg{K}$.
%\end{teor}
%
%\begin{osse}
%$L\subseteq L'$ estensione algebrica, $L$ algebricamente chiuso $\implies$ $L=L'$: $\alpha\in L'$ $\implies$ $\polmin_{\alpha,L}\in L[X]$ irriducibile e monico con una radice in $L$ $\implies$ $\deg(\polmin_{\alpha,L})=1$ $\implies$ $\polmin_{\alpha,L}=X-\alpha$ $\implies$ $\alpha\in L$.
%\end{osse}
%
%\begin{coro}
%$K$ campo $\implies$ una chiusura algebrica di $K$ è unica a meno di $K$-isomorfismo.
%\end{coro}
%
%\begin{proof}
%$K\subseteq L$ e $i : K\to\calg{K}$ chiusure algebriche di $K$ $\implies$ per il Teorema $\exi$ $K$-omomorfismo $j : L\to\calg{K}$. \\
%$j$ estensione algebrica (perché $i$ lo è), $L$ algebricamente chiuso $\implies$ $j$ $K$-isomorfismo per l'Osservazione.  
%\end{proof}
%
%\begin{proof}
%Nell'insieme parzialmente ordinato e $\ne\emptyset$
%\[
%\{(L',j')\st\text{$K\subseteq L'\subseteq L$ estensioni, $j' : L'\to\calg{K}$ $K$-omomorifsmo}\}
%\]
%(in cui $(L',j')\le(L'',j'')$ $\iff$ $L'\subseteq L''$ e $j''\rest{L'}=j'$) ogni catena $\{(L_{\lambda},j_{\lambda})\st\lambda\in\Lambda\}$ ha un maggiorante $(\tilde{L},\tilde{j})$ con $\tilde{L}:=\bigcup_{\lambda\in\Lambda}L_{\lambda}$ e
%\[
%\tilde{j} : \tilde{L}\to\calg{K} \qquad \alpha\mapsto j_{\lambda}(\alpha)\qquad\text{se }\alpha\in L_{\lambda}
%\]
%({\em esercizio}). Per il lemma di Zorn esiste un elemento massimale $(L_0,j_0)$, e basta dimostrare $L_0=L$. \\
%$\alpha\in L$ $\implies$ $\alpha$ algebrico su $K$ $\implies$ $\alpha$ algebrico su $L_0$. \\
%$\calg{K}$ algebricamente chiuso $\implies$ $\exi\beta\in\calg{K}$ tale che $\polmin_{\alpha,L_0}(\beta)=0$ $\implies$ per la Proposizione $\exi$ $L_0$-omomorfismo $j'_0 : L_0(\alpha)\to\calg{K}$ (tale che $j'_0(\alpha)=\beta$) $\implies$ $(L_0,j_0)\le(L_0(\alpha),j'_0)$ $\implies$ $L_0=L_0(\alpha)$ per la massimalit\`a di $(L_0,j_0)$ $\implies$ $\alpha\in L_0$.
%\end{proof}


\section{Estensioni normali}

\begin{defi}
Un'estensione algebrica $K \hookrightarrow L$ è {\em normale} quando il polinomio minimo di ogni elemento di $L$ si spezza su $L$.
\end{defi}

%\nota{Definizioni equivalenti?}

\begin{prop}
Sia $i : K \to L$ un'estensione. Allora sono equivalenti:
\begin{enumerate}
\item $i : K \to L$ è normale.
\item Ogni $f \in K[X]$ irriducibile che ha una radice in $L$ si spezza in $L$. 
\end{enumerate}
\end{prop}

\begin{proof}
($1 \Rightarrow 2$) Sia $f \in K[X]$ irriducibile e indichiamo con $\alpha \in L$ una delle sue radici. Siano $g \in K[X]$ monico e $c \in K^\ast$ tali che $f = c g$. Ora $\alpha$ è radice di $g$ e $g$ è irriducibile: quindi, grazie alla Proposizione~\ref{prop:EquivalentiPolinomioMinimo}, $g$ è proprio il polinomio minimo di $\alpha$. Assumendo $(1)$, possiamo concludere che $f$ si spezza completamente in $L$.\newline
($2 \Rightarrow 1$) Banale.
\end{proof}

%\begin{esem}
%$K\subseteq\calg{K}$ chiusura algebrica di $K$ $\implies$ $K\subseteq\calg{K}$ è normale.
%\end{esem}

Molto presto vedremo altre definizioni di estensioni normali, forse più interessanti per la piega che prenderanno le cose.

\begin{prop}
Un'estensione finita è normale se e solo se è il campo di spezzamento di un polinomio non nullo.
\end{prop}

Un verso è banale, l'altro dovrebbe sorprenderti: un campo di spezzamento di {\em un} polinomio non nullo consente di spezzare completamente tutti i polinomi minimi. Non è banale, richiederà del lavoro non indifferente, e manca solo di introdurre la separabilità per poter parlare delle {\em estensioni di Galois}, il fine di queste pagine.

\begin{proof}%[Dimostrazione di $\implies$]
($\Rightarrow$) Esercizio.\newline
($\Leftarrow$) Sia $i : K \to L$ campo di spezzamento di un fissato $f \in K[X]$ non nullo e preso $\alpha \in L$ proviamo che il polinomio minimo $m \in K[X]$ si spezza completamente in $L$. Chiaramente $\alpha \in L$, quindi mostriamo che qualsiasi altra sua radice $\beta$ appartiene a $L$. Siamo più precisi: dove dovrebbero vivere le radici? Costruiamo a questo fine il campo di spezzamento $j : L \to L'$ di $i_\ast(m) \in L[X]$, cioè un campo che sicuramente contiene tutte le radici di $m$. Quindi, tecnicamente parlando, non giungeremo a provare che $\beta \in L$, ma che $\beta$ sta nella copia di $L$ da qualche parte. Se ancora il discorso è fumoso, ci arriveremo piano piano. Disegniamo per cominciare:
\[\begin{tikzcd}[row sep=small]
& L \ar["{i'}", dr] \\
K \ar["i", ur] \ar["{k_1}", dr, swap] &  & L' \\
& K(\alpha) \ar["{k_2}", uu]
\end{tikzcd}\]
%\[\begin{tikzcd}
%K \ar["i", r] & L \ar["{i'}", r] & L
%\end{tikzcd}\]
Qui introduciamo notazioni: $k_1$ manda $r \in K$ in $i(k)$ mentre $k_2$ è una mera inclusione insiemistica. Se $\beta \in L'$ è una delle radici di $m \in K[X]$, allora possiamo costruire un morfismo di estensioni $j : K(\alpha) \to L'$ da $k_2$ a $i'i$ che manda $\alpha$ in $\beta$:
\[\begin{tikzcd}[row sep=small]
& L \ar["{i'}", dr] \\
K \ar["i", ur] \ar["{k_1}", dr, swap] &  & L' \\
& K(\alpha) \ar["{k_2}", uu] \ar["j", ur, swap]
\end{tikzcd}\]
Adesso constatiamo che $f$ si spezza completamente pure su $L'$. Infatti se $f$ si spezza come $c \prod_{l = 1}^n (X-\alpha_l)$ in $L$, abbiamo
\[\left(i'i\right)_\ast (f) = i'_\ast i_\ast (f) = i'_\ast \left(c \prod_{l = 1}^n (X-\alpha_l)\right) = i'(x) \prod_{l = 1}^n \left(X-i'(\alpha_l)\right) .\]
Questo fatto apparentemente inutile non lo è se si ricorda $i'i = jk_1$: segue che $k_{1\ast} (f)$ si spezza completamente su $L'$. Per il Teorema~\ref{teor:UnicitaCampoDiSpezzamento} esiste un morfismo di estensioni $h : L \to L'$ da $k_2$ a $j$.
\[\begin{tikzcd}[row sep=small]
& L \ar["{i'}", dr, shift left] \ar["h", dr, shift right, swap] \\
K \ar["i", ur] \ar["{k_1}", dr, swap] &  & L' \\
& K(\alpha) \ar["{k_2}", uu] \ar["j", ur, swap]
\end{tikzcd}\]
Qui abbiamo che $h(\alpha) = h\left(k_2 (\alpha)\right) = \beta$. Avevamo detto che volevamo vedere che $\beta \in L$: a essere precisi abbiamo trovato $\beta$ appartiene alla copia di $L$ dentro $L'$. Va bene\dots{} no?
\end{proof}

\begin{eser}
Quindi nella dimostrazione sopra a cos'è servito $i'$?
\end{eser}

%Propriet\`a delle estensioni normali: $F\subseteq K\subseteq L$ estensioni.
%\begin{itemize}
%\item $F\subseteq L$ normale $\implies$ $K\subseteq L$ normale: $\alpha\in L$ $\implies$ $\polmin_{\alpha,K}$ si spezza su $L$ perché $\polmin_{\alpha,K}\dvd\polmin_{\alpha,F}$ e $\polmin_{\alpha,F}$ si spezza su $L$.
%\item $F\subseteq L$ finita e normale $\notimplies$ $F\subseteq K$ normale: per esempio, $F=\Q$, $K=\Q(\sqrt[3]{2})$ e $L=\Q(\sqrt[3]{2},\omega)$ con $\omega^3=1$ e $\omega\ne1$ ($F\subseteq K$ non normale perché $\polmin_{\sqrt[3]{2},F}=X^3-2$ non si spezza su $K\subseteq\R$, $F\subseteq L$ normale perché campo di spezzamento di $X^3-2$).
%\item $[L:K]=2$ $\implies$ $K\subseteq L$ normale: $\alpha\in L\setminus K$ $\implies$ $L=K(\alpha)$ $\implies$ $\deg(\polmin_{\alpha,K})=[K(\alpha):K]=2$ $\implies$ $K\subseteq L$ campo di spezzamento di $\polmin_{\alpha,K}$.
%\item $F\subseteq K$ e $K\subseteq L$ finite e normali $\notimplies$ $F\subseteq L$ normale: per esempio, $F=\Q$, $K=\Q(\sqrt{2})$ e $L=\Q(\sqrt[4]{2})$ ($F\subseteq K$ e $K\subseteq L$ normali perché di grado $2$, $F\subseteq L$ non normale perché $\polmin_{\sqrt[4]{2},F}=X^4-2$ non si spezza su $L\subseteq\R$).
%\end{itemize}


\section{Estensioni separabili}

\begin{defi}
Sia $K$ un campo. Un $f \in K[X]$ non nullo è detto {\em separabile} quando ha $\deg(f)$ radici distinte in un campo di spezzamento di $f$.
\end{defi}

\begin{rich}[Derivata di un polinomio]
Dato $R$ un anello e $f := \sum_{k \in \N} a_kX^k \in R[X]$, la {\em derivata} di $f$ è definito come il polinomio
\[f' := \sum_{k \ge 1} ka_kX^{k-1} .\] 
Ricordiamo anche che soddisfa le note proprietà della derivazione vista in Analisi: in particolare $(f+g)' = f'+g'$ e $(fg)' = f'g + fg'$ e la derivata dei polinomi costanti è $0$.\newline
Questa nozione è importante per stabilire la molteplicità delle radici. Se $K \subseteq L$ è un'estensione, $f \in K[X]$ e $\alpha \in L$ è radice di $f$, allora $\alpha$ è radice multipla di $f$ (vale a dire che cioè $(X-\alpha)^2$ divide $f$) se e solo se $\alpha$ è radice di $f'$.
\end{rich}

Torniamo al discorso della definizione di polinomio separabile. È molto semplice provare la separabilità di un polinomio, ma questo richiede un'osservazione preliminare.

\begin{osse}
Sia $i : K \to L$ un'estensione e $f, g \in K[X]$. Se $\gcd(f, g) = 1$, allora $\gcd\left(i_\ast(f), i_\ast(g)\right) = 1$, cioè le estensioni preservano la relazione di essere coprimi.
\end{osse}

\begin{lemm}
Sia $K$ un campo e $f \in K[X]$ non nullo. Allora sono equivalenti:
\begin{enumerate}
\item $f$ è separabile.
\item $\gcd(f, f') = 1$.
\end{enumerate}
\end{lemm}

\begin{proof}
%($1 \implies 2$) Per $\alpha$ radice di $f$, possiamo scrivere $f = (X-\alpha) g_\alpha$ per dei $g_\alpha$. Derivando
%\[f' = g_\alpha + (X-\alpha)g'_\alpha\]
%si ha che $g_\alpha$ non si annullerà mai in $\alpha$ e quindi i polinomio $X-\alpha$ non divideranno mai $f'$. Concludiamo quindi che $\gcd (f, f') = 1$.\newline
%($2 \implies 1$)
Sia $K \subseteq L$ un campo di spezzamento di $f$. Per $\alpha \in L$ radice di $f$, indichiamo con $m_\alpha$ la molteplicità algebrica di $\alpha$. Possiamo quindi scrivere 
\[f = (X-\alpha)^{m_\alpha} g_\alpha\]
dove in particolare $g_\alpha$ non si annulla in $0$. Deriviamo:
\[f' = m_\alpha (X-\alpha)^{m_\alpha -1}g_\alpha + (X-\alpha)^{m_\alpha} g_\alpha' = (X-\alpha)^{m_\alpha -1} \left(m_\alpha g_\alpha + (X-\alpha)g_\alpha'\right) .\]
Se $m_\alpha = 1$ per ogni radice $\alpha$, allora $f$ e $f'$ non hanno alcun divisore comune $X-\alpha$. Quindi $f$ e $f'$ devono essere coprimi. Viceversa, se $\gcd(f, f') = 1$, allora tutti gli $m_\alpha$ devono essere $1$.
\end{proof}

\begin{prop}
Sia $K$ un campo e $f \in K[X]$ irriducibile . Allora un $f$ è separabile se e solo se $f' \ne 0$.
\end{prop}

È straordinariamente semplice verificare se un polinomio irriducibile è separabile o meno.

\begin{proof}
Se $f \ne 0$, allora $\gcd(f, f') = 1$ perché $f$ è irriducibile. Quindi $f$ è separabile. Viceversa, se $f$ è separabile, allora $\gcd(f, f') = 1$, cioè $fg + f'h = 1$ per qualche $g, h \in K[X]$. Valutando in una delle radici $\alpha \in L$ di $f$, si ha $f'(\alpha) h(\alpha) = 1$, e quindi $f'(\alpha) \ne 0$. Possiamo tranquillamente concludere che $f' \ne 0$.
%Sia $K\subseteq L$ campo di spezzamento di $f$.\newline
%($\implies$) Se $\alpha \in L$ è una radice semplice di $f$, allora $f = (X-\alpha) g$ per qualche $g \in L[X]$. Deriviamo: $f' = g + (X-\alpha)g'$. Qui $g$ non si annulla se valutato in $\alpha$, mentre il secondo addendo sì. Possiamo concludere che sicuramente $f' \ne 0$.\newline
%($\impliedby$) $\deg f' < \deg f$ $\implies$ $f\ndvd f'$, $\implies$ $\mcd(f,f') = 1$ in $K[X]$ (perché $f$ irriducibile in $K[X]$) $\implies$ $\exi g,h\in K[X]$ tali che $1=gf+hf'$ $\implies$ $\mcd(f,f')=1$ in $L[X]$ $\implies$ $f$ non ha radici multiple in $L$, cioè $f$ è separabile. \qedhere
\end{proof}

\begin{defi}
Un'estensione algebrica $K \subseteq L$ è detta {\em separabile} qualora il polinomio minimo di ogni elemento è separabile.
\end{defi}

Per fortuna, per certe estensioni $K \subseteq L$ non è necessario dimostrare che tutti gli elementi di $L$ abbiano polinomio minimo separabile.

\begin{prop}
Sia $K \subseteq L$ un'estensione generata da $\alpha_1, \dots{}, \alpha_n \in L$ algebrici. Se i polinomi minimi degli $\alpha_i$ sono separabili, allora l'estensione $K \subseteq L$ è separabile.
\end{prop}

\begin{proof}
Per induzione su $n$. Consideriamo l'estensione $K \subseteq L = K(\alpha_1)$ e $\beta \in L$. \nota{Da scrivere.}
\end{proof}

\begin{defi}[Campi perfetti]
\nota{Scrivere.}
\end{defi}

\begin{prop}
Sia $K$ un campo di caratteristica $0$. Tutti gli $f \in K[X]$ irriducibile sono separabili. Pertanto tutte le estensioni algebriche di campi di caratteristica $0$ sono separabili. 
\end{prop}

\begin{proof}
Se $f$ è irriducibile, in particolare $n:=\deg f >0$. Scriviamo $f := \sum_{k = 0}^n a_kX^k$ con $a_n \ne 0$. Derivando, $f' = \sum_{k=1}^n ka_kX^{k-1}$. Il polinomio sicuramente non nullo: $n a_n \ne 0$ perché $K$ ha caratteristica $0$. Concludiamo quindi che $f$ è separabile.
\end{proof}

\begin{prop}
Sia $K$ un campo di caratteristica $p$ primo. Tutti gli $f \in K[X]$ irriducibile sono separabili. Quindi le estensioni algebriche di siffatti campi sono separabili. 
\end{prop}

\begin{proof}
\nota{Da scrivere.}
\end{proof}

%\begin{defi}
%$A$ dominio, $\car(A)=p$ primo. L'{\em omomorfismo di Frobenius} (di $A$) è l'omomorfismo di anelli $\Fro : A\to A$, $a\mapsto a^p$.
%\end{defi}
%
%\begin{proof}
%$\Fro(1)=1$; $\all a,b\in A$ $\Fro(ab)=(ab)^p=a^pb^p=\Fro(a)\Fro(b)$ e $\Fro(a+b)=(a+b)^p=\sum_{i=0}^p\binom{p}{i}a^{p-i}b^i=a^p+b^p=\Fro(a)+\Fro(b)$ perché $p\dvd\binom{p}{i}=\frac{p!}{i!(p-i)!}$ per $0<i<p$.
%\end{proof}
%
%\begin{defi}
%Un campo $K$ è {\em perfetto} se $\car(K)=0$ o $\car(K)=p$ primo e $\Fro : K\to K$ è suriettivo (nel qual caso $\Fro\in\Gal{K}$).
%\end{defi}
%
%\begin{prop}
%$K$ campo è perfetto $\iff$ $f$ è separabile $\all f\in K[X]$ irriducibile.
%\end{prop}
%
%\begin{proof}
%Posso supporre $\car(K)=p$ primo.
%\begin{itemize}
%\item[$\implies$] $f\in K[X]$ irriducibile $\implies$ per il Lemma basta dimostrare $f'\ne0$. Per assurdo $f'=0$ $\implies$ $f=\sum_{i=0}^na_iX^{pi}$; $K$ perfetto $\implies$ $\exi b_i\in K$ tale che $a_i=\Fro(b_i)=b_i^p$ $\all i=0,\dots,n$ $\implies$ $f=\Fro(\sum_{i=0}^nb_iX^i)=(\sum_{i=0}^nb_iX^i)^p$, assurdo.
%\item[$\impliedby$] $a\in K$ $\implies$ $\exi K\subseteq L$ estensione tale che $X^p-a$ ha una radice $\alpha$ in $L$ $\implies$ $\polmin_{\alpha,K}\dvd(X^p-a)=X^p-\alpha^p=(X-\alpha)^p$ $\implies$ $\polmin_{\alpha,K}=X-\alpha$ (perché $\polmin_{\alpha,K}$ monico e irriducibile, quindi separabile) $\implies$ $\alpha\in K$ e $a=\alpha^p=\Fro(\alpha)$.\qedhere
%\end{itemize}
%\end{proof}
%
%\begin{defi}
%$K\subseteq L$ estensione.
%\begin{itemize}
%\item $\alpha\in L$ è {\em separabile} su $K$ se $\alpha$ è algebrico su $K$ e $\polmin_{\alpha,K}$ è separabile.
%\item $K\subseteq L$ è {\em separabile} se $\alpha$ è separabile su $K$ $\all\alpha\in L$.
%\end{itemize}
%\end{defi}
%
%\begin{osse}
%$F\subseteq K\subseteq L$ estensioni con $F\subseteq L$ separabile $\implies$ $F\subseteq K$ separabile (ovvio) e $K\subseteq L$ separabile (perché $\polmin_{\alpha,K}\dvd\polmin_{\alpha,F}$ $\all\alpha\in L$). \\
%%Si pu\`o dimostrare che $F\subseteq K$ e $K\subseteq L$ separabili $\implies$ $F\subseteq L$ separabile.  
%\end{osse}
%
%\begin{coro}
%$K$ è perfetto $\iff$ ogni estensione algebrica di $K$ è separabile.
%\end{coro}
%
%\begin{proof}
%Segue dalla Proposizione precedente, tenendo conto che $f\in K[X]$ irriducibile e monico $\implies$ $\exi K\subseteq L$ estensione algebrica e $\exi\alpha\in L$ tale che  $f=\polmin_{\alpha,K}$.
%\end{proof}
%
%Esempi di campi perfetti.
%
%\begin{itemize}
%\item $K$ finito $\implies$ $K$ perfetto: \\
%$\car(K)=p$ primo e $\Fro : K\to K$ è suriettivo perché iniettivo.
%\item $K$ algebricamente chiuso $\implies$ $K$ perfetto: \\
%posso supporre $\car(K)=p$ primo $\implies$ $\Fro : K\to K$ è suriettivo perché $\all a\in K$ $X^p-a$ ha una radice $b\in K$, cioè $a=b^p=\Fro(b)$.
%\item $K\subseteq L$ estensione algebrica, $K$ perfetto $\implies$ $L$ perfetto: \\
%per il Corollario basta dimostrare ogni estensione algebrica $L\subseteq L'$ è separabile. \\
%$K\subseteq L$ e $L\subseteq L'$ algebriche $\implies$ $K\subseteq L'$ algebrica $\implies$ $K\subseteq L'$ separabile (sempre per il Corollario) $\implies$ $L\subseteq L'$ separabile.
%\item $\car(K)=p$ primo $\implies$ $K(X)$ non perfetto: \\
%per assurdo $\exi f/g\in K(X)$ (con $f,g\in K[X]$ e $g\ne0$) tale che $X=\Fro(f/g)=f^p/g^p$ $\implies$ $f^p=Xg^p$ in $K[X]$, assurdo.
%\end{itemize}


\section{Gruppo di Galois}

\begin{defi}
Il {\em gruppo di Galois} di un'estensione $i : K \to L$ è
\[\Gal{K \mor i L} := \{\sigma : L \to L \text{ automorfismo} \mid \sigma \circ i = i \} .\]
Qualche volta l'omomorfismo è chiaro dal contesto o è una semplice inclusione, quindi non ci si scomoda a dargli un nome: alcune notazioni alternative sono $\Gal{K \hookrightarrow L}$, $\Gal{K \subseteq L}$, $\Gal{L/K}$ oppure $\Gal[K]{L}$.
\end{defi}

Cioè il gruppo di Galois di $i : K \to L$ è il gruppo degli automorfismi $L \to L$ che fissano gli elementi dell'immagine di $K$ in $L$. Se usiamo il solito abuso, possiamo dire che è il gruppo degli automorfismi di $L$ che fissano gli elementi di $K$, il che non scatena problemi in molti casi concreti.

\begin{defi}
Sia $K$ un campo, e $f \in K[X]$ non nullo. Il {\em gruppo di Galois} di $f$ su $K$ è il gruppo di Galois di un campo di spezzamento per $f$. Viene indicato molto semplicemente come $\Gal f$.
%(ben definito a meno di isomorfismo) è $\Gal[K]{f}:=\Gal[K]{L}$ con $K\subseteq L$ campo di spezzamento di $f$.
\end{defi}

Abbiamo visto che il campo di spezzamento è unico a meno di isomorfismo, e anche il corrispondente gruppo di Galois è definito a meno di isomorfismo. È importante capire che è importante avere una certa manualità nel calcolo dei campi di spezzamento di polinomi.

%\begin{prop}
%$F\subseteq K\subseteq L$ estensioni con $F\subseteq L$ finita e normale. \\
%Allora $F\subseteq K$ è normale $\iff$ $K$ è $\Gal[F]{L}$-stabile (cioè $\sigma(K)=K$ $\all\sigma\in\Gal[F]{L}$, o, equivalentemente, $\sigma(K)\subseteq K$ $\all\sigma\in\Gal[F]{L}$).
%\end{prop}
%
%\begin{proof}
%\begin{itemize}
%\item[$\implies$] $\sigma\in\Gal[F]{L}$ $\implies$ $\sigma(K)\subseteq K$:
%\smallskip
% 
%$\alpha\in K$ $\implies$ $\polmin_{\alpha,F}(\sigma(\alpha))=\sigma(\polmin_{\alpha,F}(\alpha))=\sigma(0)=0$ $\implies$ $\sigma(\alpha)\in K$ perché $\polmin_{\alpha,F}$ si spezza su $K$.
%\item[$\impliedby$] $F\subseteq L$ normale e finita $\implies$ $F\subseteq L$ campo di spezzamento di $f\in F[X]\setminus\{0\}$.
%\smallskip
%
%$\alpha\in K$ $\implies$ $\polmin_{\alpha,F}$ si spezza su $L$, e va dimostrato che si spezza su $K$, cioè che $\beta\in L$ tale che $\polmin_{\alpha,F}(\beta)=0$ $\implies$ $\beta\in K$.
%\smallskip
%
%$\exiun$ $F$-omomorfismo $i : F(\alpha)\to L$ tale che $i(\alpha)=\beta$. \\
%$F(\alpha)\subseteq L$ campo di spezzamento di $f$, $f=i(f)$ si spezza su $L$ $\implies$ $\exi$ $F(\alpha)$-omomorfismo (e quindi $F$-omomorfismo) $\sigma : L\to L$ (cioè tale che $\sigma\rest{F(\alpha)}=i$).
%\smallskip
%
%$\sigma(L)\subseteq L$, $\dim_F(\sigma(L))=\dim_F(L)<\infty$ $\implies$ $\sigma(L)=L$ $\implies$ $\sigma\in\Gal[F]{L}$ $\implies$ $\beta=i(\alpha)=\sigma(\alpha)\in\sigma(K)=K$.\qedhere
%\end{itemize}
%\end{proof}

Il gruppo di Galois può essere un oggetto piuttosto difficile da calcolare in generale, mentre noi ci limiteremo ad una classe di estensioni per cui si possono avere degli strumenti e delle indicazioni. Il lemma che segue riguarda il numero di morfismi di estensioni da un'estensione finita.

\begin{lemm}
Sia $i : K \to L$ un'estensione finita e $j : K \to L'$ un'altra estensione. Allora il numero dei morfismi di estensioni $L \to L'$ da $i$ a $j$ è $\le [L:K]$. Vale l'uguaglianza se e solo se per ogni $\alpha \in L$ il suo polinomio minimo come elemento di $L'[X]$ si spezza ed è separabile.
%$\polmin_{\alpha,K}$ ha $\deg(\polmin_{\alpha,K})$ radici distinte in $L'$ (cioè $\polmin_{\alpha,K}$ è separabile e si spezza su $L'$) $\all\alpha\in L$.
\end{lemm}

\begin{proof}
Andiamo per induzione su $n := [L:K]$. Il caso base, $n = 1$, significa che $L \iso K$ come campi e quindi il morfismo di estensioni da $i$ a $j$ è l'unico che può esserci, cioè $ji^{-1}$. Possiamo quindi supporre che $n > 0$ per il passo induttivo. In questo caso, a causa della Proposizione~\ref{prop:EstensioneFinitaEquivalenti}, esistono $\alpha_1, \dots{}, \alpha_n \in L$ algebrici tali che $L = K\left(\alpha_1, \dots{}, \alpha_n\right)$. Introduciamo il campo $E := K\left(\alpha_1, \dots{}, \alpha_{n-1}\right)$, quindi in particolare $L = E(\alpha_n)$. Decomponiamo l'estensione $i : K \to L$ come segue
\[\begin{tikzcd}
K \ar["i", rr] \ar["{i_1}", dr, swap] & & L \\
& E \ar["{i_2}", ur, swap]
\end{tikzcd}\]
dove $i_1(r) := i(r)$ e $i_2(s) := s$. Disegniamo allora
\[\begin{tikzcd}
E\left(\alpha_n\right) & & L' \\
E \ar["{i_2}", u] \\
& K \ar["{i_1}", ul] \ar["j", uur, swap]
\end{tikzcd}\]
Induttivamente, ci sono al massimo $[E:K]$ morfismi di estensioni da $i_1$ a $j$ e per ciascuno dei siffatti $h : E \to L'$, grazie al Corollario~\ref{coro:NumeroMorfismiEstensioniDaKAlgebrico}, sappiamo che il numero di morfismi di estensioni da $i_2$ a $h$
\[\begin{tikzcd}
E(\alpha_n) \ar[r] & L' \\
E \ar["{i_2}", u] \ar["h", ur, swap]
\end{tikzcd}\]
è minore o uguale a $\left[E\left(\alpha_n\right):E\right]$. Per costruzione questi morfismi di estensioni sono morfismi da $i$ a $j$. Quindi al massimo ci sono
\[[E(\alpha_n):E][E:K] = [L:E][E:K] = [L:K]\]
estensioni da $i$ a $j$. \nota{Riscrivere meglio e provare il se e solo se.}
\end{proof}

\begin{prop}[Cardinalità gruppi di Galois]\label{prop:CardGruppiGalois}
Se $K \subseteq L$ è un'estensione finita, allora $\card{\Gal[K]{L}} \le [L:K]$. Vale l'uguaglianza se e solo se $K \subseteq L$ è anche normale e separabile.
\end{prop}

\begin{proof}
$\Gal[K]{L}$ è il gruppo dei morfismi di estensione dall'estensione $K \subseteq L$ in sé. Quindi si applica il lemma di sopra. \nota{Più dettagli.}
\end{proof}

Quindi il gruppo di Galois di un'estensione finita è finito: l'interesse per i gruppi di ordine finito risiede anche in questo. Esistono estensioni il cui gruppo di Galois è infinito. 

\begin{esem}[$\Q \subseteq \bar\Q$]
\nota{Scrivere.}
\end{esem}

%La Proposizione~\ref{prop:CardGruppiGalois} indica la strada: le estensioni finite, normali e separabili hanno una caratterizzazione molto forte e meritano un nome.

%\section{Estensioni di Galois}

\begin{defi}
Un'estensione è detta {\em di Galois} qualora è finita, normale e separabile. Equivalentemente, un'estensione è di Galois quando è separabile e campo di spezzamento di un qualche polinomio non nullo.
\end{defi}

La Proposizione~\ref{prop:CardGruppiGalois} fornisce un criterio che può essere comodo a volte per capire se un'estensione $K \subseteq L$ è di Galois, a patto di avere l'informazione della cardinalità di $\Gal[K]{L}$.

Prima di fare i primi esempi, indugiamo sui gruppi di Galois di estensioni finite par farci un'idea su come siano fatti gli automorfismi di questo tipo di estensioni.

\begin{lemm}
Sia $i : K \to L$ un'estensione, $f \in K[X]$ e $\phi \in \Gal[K]{L}$. Allora per ogni $\alpha \in L$ si ha $i_\ast(f)(\phi(\alpha)) = \phi\left(i_\ast(f)(\alpha)\right)$. Quindi in particolare, gli zeri di $f$ vengono mandati negli zeri di $f$.
\end{lemm}

\begin{proof}
Scriviamo $f := \sum_{k \in \N} a_k X^k$ e ricordiamo che $\phi \circ i = i$.
\begin{align*}
i_\ast (f) (\phi(\alpha)) &= \sum_{k \in \N} i\left(a_k\right) (\phi(\alpha))^k = \\
                          &= \sum_{k \in \N} \phi\left(i\left(a_k\right)\right) \phi\left(\alpha^k\right) = \\
                          &= \phi\left( \sum_{k \in \N} i\left(a_k\right) \alpha^k \right) = \\
                          &= \phi\left(i_\ast(f)(\alpha)\right) .\qedhere
\end{align*}
\end{proof}

\begin{prop}
Sia $i : K \to L$ un'estensione finitamente generata, cioè $L = K\left(\alpha_1, \dots{}, \alpha_n\right)$ per degli $\alpha_1, \dots{}, \alpha_n \in L$. Allora ogni $\phi \in \Gal[K]{L}$ è univocamente determinato da $\phi\left(\alpha_1\right), \dots{}, \phi\left(\alpha_n\right)$.
\end{prop}

\begin{proof}
Da definizione, gli elementi di $\Gal[K]{L}$ fissano gli elementi di $K$ e se due elementi di $\Gal[K]{L}$ sono uguali su $\left\{\alpha_1, \dots{}, \alpha_n\right\}$, allora sono uguali ovunque.
\end{proof}

\begin{coro}
Sia $i : K \to L$ un'estensione finitamente generata, cioè $L = K\left(\alpha_1, \dots{}, \alpha_n\right)$ per degli $\alpha_1, \dots{}, \alpha_n \in L$. Allora $\Gal[K]{L}$ è isomorfo ad un sottogruppo di $S_n$.
\end{coro}

E qui è chiaro anche come mai ci sia un interesse verso i gruppi simmetrici $S_n$ e i suoi sottogruppi: sostanzialmente gli elementi di $\Gal[K]{L}$ sono permutazioni degli $\alpha_i$. Ne riparleremo sicuramente in seguito, per ora facciamo degli esempi.

\begin{esem}[Gruppo di Galois di $\Q \subseteq \Q(i)$]
\nota{Scrivere.}
\end{esem}

%\begin{osse}
%$F\subseteq K\subseteq L$ estensioni con $F\subseteq L$ di Galois $\implies$ $K\subseteq L$ di Galois.
%\end{osse}

%\begin{coro}
%$K\subseteq L$ estensione finita $\implies$ $\card{\Gal[K]{L}}\le[L:K]$ e vale l'uguaglianza $\iff$ $K\subseteq L$ è di Galois.
%\end{coro}

%\begin{proof}
%$j : L\to L$ $K$-omomorfismo $\implies$ $j$ suriettivo (perché $j$ omomorfismo iniettivo di $K$-spazi vettoriali e $\dim_K(L)<\infty$) $\implies$ per il Teorema
%\[
%\card{\Gal[K]{L}}=\card{\{j : L\to L\st j\text{ $K$-omomorfismo}\}}\le[L:K]
%\]
%e vale l'uguaglianza $\iff$ $\polmin_{\alpha,K}$ separabile e si spezza su $L$ $\all\alpha\in L$ $\iff$ $K\subseteq L$ separabile e normale $\iff$ $K\subseteq L$ di Galois.
%\end{proof}

%\nota{Anticipare la sezione sul gruppo di Galois di un polinomio?}

%\section{Il gruppo di Galois di un polinomio}


%\begin{osse}
%$K\subseteq L$ campo di spezzamento di $f\in K[X]\setminus\{0\}$, $G:=\Gal[K]{f}$.
%\begin{itemize}
%\item $K$ perfetto $\implies$ $K\subseteq L$ di Galois $\implies$ $\card{G}=[L:K]$.
%\item $R:=\{\alpha\in L\st f(\alpha)=0\}$ $\implies$ $n:=\card{R}\le\deg(f)$. \\
%$\sigma\in G$, $\alpha\in R$ $\implies$ $\sigma(\alpha)\in R$, quindi si ottiene una funzione $G\to S(R)\iso S_n$, $\sigma\mapsto\sigma\rest{R}$, che è un omomorfismo iniettivo (perché $L=K(R)$) $\implies$ $G\iso G'< S_n$ ($\implies$ $\card{G}\dvd n!$).
%\item $K$ perfetto, $f$ irriducibile $\implies$ $\deg(f)=n$ e $n\dvd\card{G}\dvd n!$.
%\end{itemize}
%\end{osse}
%
%Esempi: $K$ perfetto, $f\in K[X]$ irriducibile, $n:=\deg(f)$, $G:=\Gal[K]{f}$.
%\begin{itemize}
%\item $n=2$ $\implies$ $2\dvd\card{G}\dvd2!$ $\implies$ $\card{G}=2$ $\implies$ $G\iso C_2$.
%\item $n=3$ $\implies$ $3\dvd\card{G}\dvd3!$ $\implies$ $\card{G}=3$ o $6$ $\implies$ $G\iso C_3$ o $S_3$ (perché $G\iso G'<S_3$).
%\smallskip
%
%$f=X^3-2$ $\implies$ $G\iso S_3$ se $K=\Q$, $G\iso C_3$ se $K=\Q(\omega)$.
%\item $n=4$ $\implies$ $4\dvd\card{G}\dvd4!$ $\implies$ $\card{G}=4$, $8$, $12$ o $24$ $\implies$ $G\iso C_4$, $C_2^2$, $D_4$, $A_4$ o $S_4$ (perché $G\iso G'<S_4$).
%\smallskip
%
%$f=X^4-10X^2+1=\polmin_{\alpha,\Q}$ con $\alpha=\sqrt{2}+\sqrt{3}$ $\implies$ $G\iso C_2^2$: $\Q\subset\Q(\alpha)=\Q(\sqrt{2},\sqrt{3})$ normale (perché campo di spezzamento di $(X^2-2)(X^2-3)$) $\implies$ $f$ si spezza su $\Q(\alpha)$ $\implies$ $\Q\subset\Q(\alpha)$ campo di spezzamento di $f$ $\implies$ $G=\Gal[\Q]{\Q(\alpha)}$ $\implies$ $\card{G}=[\Q(\alpha):\Q]=\deg{f}=4$ e $G\niso C_4$ perché $\sigma\in G=\Gal[\Q]{\Q(\sqrt{2},\sqrt{3})}$ $\implies$ $\sigma(\sqrt{2})=\pm\sqrt{2}$ e $\sigma(\sqrt{3})=\pm\sqrt{3}$ $\implies$ $\sigma^2(\sqrt{2})=\sqrt{2}$ e $\sigma^2(\sqrt{3})=\sqrt{3}$ $\implies$ $\sigma^2=\id_{\Q(\sqrt{2},\sqrt{3})}$.
%\end{itemize}

%\section{Campo fisso di un gruppo di automorfismi}
%\section{Corrispondenza di Galois}

\section{Il teorema fondamentale}

\nota{Riscrivere la parte su come $\Gal[K]{L}$ agisce sull'insieme delle radici.}

$G$ gruppo, $X$ $G$-insieme $\implies$
\[
X^G:=\{x\in X\st gx=x\ \all g\in G\}\subseteq X.
\]
$L$ campo, $G<\Gal{L}$ $\implies$
\[
L^G=\{\alpha\in L\st\sigma(\alpha)=\alpha\ \all\sigma\in G\}\subseteq L
\]
sottocampo (detto {\em campo fisso} di $G$).
\begin{osse}
$F\subseteq L$ sottocampo primo $\implies$ $F\subseteq L^{\Gal{L}}$ (perché $L^{\Gal{L}}\subseteq L$ sottocampo) $\implies$ $\Gal[F]{L}=\Gal{L}$.
\end{osse}
\begin{teor}[Artin]
$L$ campo, $G<\Gal{L}$ finito $\implies$ $[L:L^G]\le\card{G}$.
\end{teor}
\begin{proof}
%\renewcommand{\qedsymbol}{}
$\card{G}=m$, $G=\{\sigma_1=\id_L,\dots,\sigma_m\}$.

Dati $\alpha_1,\dots,\alpha_n\in L$ distinti con $n>m$, basta dimostrare che $\{\alpha_1,\dots,\alpha_n\}$ è linearmente dipendente su $L^G$.
$v_j:=(\sigma_1(\alpha_j),\dots,\sigma_m(\alpha_j))\in L^m$ (per $j=1,\dots,n$) distinti.
\smallskip

$\{v_1,\dots,v_n\}$ linearmente dipendente su $L$ (perché $n>m$) $\implies$
\[
W:=\{(\beta_1,\dots,\beta_n)\in L^n\st\sum_{j=1}^n\beta_jv_j=0\}
\]
$L$-sottospazio vettoriale non nullo di $L^n$.
\smallskip

$\sigma\in G$, $(\beta_1,\dots,\beta_n)\in W$ $\implies$ $(\sigma(\beta_1),\dots,\sigma(\beta_n))\in W$: $(\beta_1,\dots,\beta_n)\in W$ $\iff$ $\sum_{j=1}^n\beta_j\sigma_i(\alpha_j)=0$ $\all i=1,\dots,m$ $\implies$ $\sum_{j=1}^n\sigma(\beta_j)(\sigma\comp\sigma_i)(\alpha_j)=0$ $\all i=1,\dots,m$ $\iff$ $(\sigma(\beta_1),\dots,\sigma(\beta_n))\in W$ perché $\{\sigma\comp\sigma_1,\dots,\sigma\comp\sigma_m\}=G$.
\smallskip

$(\gamma_1,\dots,\gamma_n)\in W\setminus\{0\}$ ($\implies$ $\exi j_0\in\{1,\dots,n\}$ tale che $\gamma_{j_0}\ne0$ e posso supporre $\gamma_{j_0}=1$) con il minimo numero di componenti $\ne0$.
\smallskip

$\sigma\in G$ $\implies$ $\delta_j:=\gamma_j-\sigma(\gamma_j)$ tali che $(\delta_1,\dots,\delta_n)\in W$ e $\delta_j=0$ se $\gamma_j=0$ o $j=j_0$ $\implies$ $(\delta_1,\dots,\delta_n)=0$ per l'ipotesi su $(\gamma_1,\dots,\gamma_n)$ $\implies$ $\gamma_j=\sigma(\gamma_j)\in L^G$ (per $j=1,\dots,n$) tali che $\sum_{j=1}^n\gamma_j\alpha_j=0$ perché $\sum_{j=1}^n\gamma_j\sigma_1(\alpha_j)=0$ e $\sigma_1=\id_L$.
\end{proof}


%\section{Corrispondenza tra sottocampi e sottogruppi}

$L$ campo $\implies$ le funzioni
\begin{gather*}
\phi : \{K\st K\subseteq L\text{ sottocampo}\}\to\{G\st G<\Gal{L}\}\quad K\mapsto\Gal[K]{L} \\
\psi : \{G\st G<\Gal{L}\}\to\{K\st K\subseteq L\text{ sottocampo}\}\quad G\mapsto L^G
\end{gather*}
soddifano le seguenti propriet\`a:
\begin{enumerate}%[i]
\item $K'\subseteq K\subseteq L$ sottocampi $\implies$ $\phi(K)\subseteq\phi(K')$ (cioè $\Gal[K]{L}<\Gal[K']{L}$);
\item $G'<G<\Gal{L}$ $\implies$ $\psi(G)\subseteq\psi(G')$ (cioè $L^G\subseteq L^{G'}$);
\item $K\subseteq L$ sottocampo $\implies$ $K\subseteq\psi(\phi(K))$ (cioè $K\subseteq L^{\Gal[K]{L}}$);
\item $G<\Gal{L}$ $\implies$ $G\subseteq\phi(\psi(G))$ (cioè $G<\Gal[L^G]{L}$).
\end{enumerate}
Segue formalmente che valgono queste ulteriori propriet\`a:
\begin{enumerate}%[i]
\setcounter{enumi}{4}
\item $K\subseteq L$ sottocampo $\implies$ $\phi(K)=\phi(\psi(\phi(K)))$ (cioè $\Gal[K]{L}=\Gal[L^{\Gal[K]{L}}]{L}$) perché $\phi(K)\subseteq\phi(\psi(\phi(K)))$ per iv e $K\subseteq\psi(\phi(K))$ per iii, quindi $\phi(\psi(\phi(K)))\subseteq\phi(K)$ per i;
\item $G<\Gal{L}$ $\implies$ $\psi(G)=\psi(\phi(\psi(G)))$ (cioè $L^G=L^{\Gal[L^G]{L}}$).
\end{enumerate}

Da v e vi segue anche che $\phi\rest{\im(\psi)} : \im(\psi)\to\im(\phi)$ è biunivoca con inversa $\psi\rest{\im(\phi)} : \im(\phi)\to\im(\psi)$, dove $\im(\psi)=\{L^G\st G<\Gal{L}\}$ e $\im(\phi)=\{\Gal[K]{L}\st K\subseteq L\text{ sottocampo}\}$.

\begin{teor}
\begin{enumerate}
\item $G<\Gal{L}$ finito $\implies$ $[L:L^G]=\card{G}$, $L^G\subseteq L$ di Galois e $G=\Gal[L^G]{L}$.
\item $K\subseteq L$ estensione finita $\implies$ $\card{\Gal[K]{L}}\le[L:K]$ e vale l'uguaglianza $\iff$ $K\subseteq L$ di Galois $\iff$ $K=L^{\Gal[K]{L}}$.
\end{enumerate}
\end{teor}

\begin{coro}
\begin{gather*}
\{K\st K\subseteq L\text{ di Galois}\}\to\{G\st G<\Gal{L}\text{ finito}\}\quad K\mapsto\Gal[K]{L} \\
\{G\st G<\Gal{L}\text{ finito}\}\to\{K\st K\subseteq L\text{ di Galois}\}\quad G\mapsto L^G
\end{gather*}
sono funzioni biunivoche una l'inversa dell'altra e che invertono le inclusioni. Inoltre $K\subseteq L$ di Galois $\implies$ $\card{\Gal[K]{L}}=[L:K]$.
\end{coro}

\begin{proof}
Sappiamo gi\`a che:
\begin{enumerate}
\item[1'] $G<\Gal{L}$ finito $\implies$ $[L:L^G]\le\card{G}$;
\item[2'] $K\subseteq L$ estensione finita $\implies$ $\card{\Gal[K]{L}}\le[L:K]$ e vale l'uguaglianza $\iff$ $K\subseteq L$ di Galois.
\end{enumerate}
\begin{enumerate}
\item Per 1' $[L:L^G]\le\card{G}<\infty$. \\
Per iv $G<\Gal[L^G]{L}$, e quindi $\card{G}\le\card{\Gal[L^G]{L}}$. \\
Per 2' $\card{\Gal[L^G]{L}}\le[L:L^G]$. \\
Dunque $[L:L^G]=\card{G}=\card{\Gal[L^G]{L}}$ (per cui $G=\Gal[L^G]{L}$) e, ancora per 2', $L^G\subseteq L$ di Galois.
\item Per iii $K\subseteq L^{\Gal[K]{L}}$. \\
Per 2' $\card{\Gal[K]{L}}\le[L:K]<\infty$. \\
Per 1 $L^{\Gal[K]{L}}\subseteq L$ di Galois e $[L:L^{\Gal[K]{L}}]=\card{\Gal[K]{L}}$. \\
Dunque, se $K=L^{\Gal[K]{L}}$, allora $K\subseteq L$ è di Galois. \\
Viceversa, se $K\subseteq L$ è di Galois, allora, sempre per 2', $[L:K]=\card{\Gal[K]{L}}=[L:L^{\Gal[K]{L}}]$, da cui segue $K=L^{\Gal[K]{L}}$. \qedhere
\end{enumerate}
\end{proof}


%\section{Il teorema fondamentale}

Ricordiamo che, fissato un campo $L$,
\begin{gather*}
\{K\st K\subseteq L\text{ di Galois}\}\to\{G\st G<\Gal{L}\text{ finito}\}\quad K\mapsto\Gal[K]{L} \\
\{G\st G<\Gal{L}\text{ finito}\}\to\{K\st K\subseteq L\text{ di Galois}\}\quad G\mapsto L^G
\end{gather*}
sono funzioni biunivoche una l'inversa dell'altra e che invertono le inclusioni. Inoltre $K\subseteq L$ di Galois $\implies$ $\card{\Gal[K]{L}}=[L:K]$.

\begin{teor}[Il teorema fondamentale]
$K\subseteq L$ estensione di Galois, $G:=\Gal[K]{L}$. Allora
\begin{gather*}
\{F\st K\subseteq F\subseteq L\text{ sottocampo}\}\to\{H\st H<G\}\quad F\mapsto\Gal[F]{L} \\
\{H\st H<G\}\to\{F\st K\subseteq F\subseteq L\text{ sottocampo}\}\quad H\mapsto L^H
\end{gather*}
sono funzioni biunivoche una l'inversa dell'altra e che invertono le inclusioni. Inoltre, se $K\subseteq F\subseteq L$ è un sottocampo, allora
\begin{enumerate}
\item $F\subseteq L$ di Galois e $\card{\Gal[F]{L}}=[L:F]$;
\item $K\subseteq F$ normale $\iff$ $H:=\Gal[F]{L}\normal G$ $\implies$ $\Gal[K]{F}\iso G/H$.
\end{enumerate}
\end{teor}

\begin{proof}
\begin{itemize}
\item La prima parte e il punto 1 seguono da quanto gi\`a visto, tenendo conto che $K\subseteq F\subseteq L$ sottocampo $\implies$ $F\subseteq L$ di Galois (perché $K\subseteq L$ di Galois).
\item Per dimostrare il punto 2, ricordiamo che $K\subseteq F$ è normale (e quindi di Galois) $\iff$ $F$ è $G$-stabile.
\item $K\subseteq F$ normale $\implies$ $f : G\to\Gal[K]{F}$, $\sigma\mapsto\sigma\rest{F}$ ben definita. Chiaramente $f$ omomorfismo e $\ker(f)=H$, per cui $H\normal G$ e $G/H\iso\im(f)$ per il primo teorema di isomorfismo. Inoltre
\[
\card{\im(f)}=\card{(G/H)}=\frac{\card{G}}{\card{H}}=\frac{\card{[L:K]}}{\card{[L:F]}}=[F:K]=\card{\Gal[K]{F}},
\]
$\implies$ $f$ suriettiva e $\Gal[K]{F}\iso G/H$.
\item $H\normal G$ $\implies$ $\sigma(\alpha)\in F$ $\all\sigma\in G$ e $\all\alpha\in F$ (quindi $K\subseteq F$ normale): $\sigma(\alpha)\in F=L^H$ $\iff$ $\tau(\sigma(\alpha))=\sigma(\alpha)$ $\all \tau\in H$ $\iff$ $(\sigma^{-1}\tau\sigma)(\alpha)=\alpha$ $\all\tau\in H$, vero perché $\sigma^{-1}\tau\sigma\in H$ e $\alpha\in F=L^H$.
\end{itemize}
\end{proof}


\begin{esem}
$K:=\Q$ e $L:=\Q(\sqrt[3]{2},\omega)$ con $1\ne\omega\in\C$ tale che $\omega^3=1$.
\begin{itemize}
\item $\Q\subset L$ di Galois (è campo di spezzamento di $X^3-2$) e $G:=\Gal[\Q]{L}=\Gal{L}$ tale che $\card{G}=[L:\Q]=6$.
\item $\Q\subset\Q(\sqrt[3]{2})$ non normale $\implies$ $\Gal[{\Q(\sqrt[3]{2})}]{L}<G$ non normale $\implies$ $G\iso S_3$.
\item $\exiun H\normal G$ non banale (di ordine $3$) $\implies$ $[L:L^H]=3$, $\Q\subset L^H$ normale e $\Gal[\Q]{L^H}\iso G/H\iso C_2$ $\implies$ $L^H=\Q(\omega)$.
\item $G$ ha anche $3$ sottogruppi non normali non banali (di ordine $2$), che corrispondono a $\Q(\omega^i\sqrt[3]{2})$ per $i=0,1,2$.
\end{itemize}
\end{esem}

\begin{osse}
\begin{itemize}
\item $K\subseteq L$ di Galois $\implies$ $\card{\{F\st K\subseteq F\subseteq L\text{ sottocampo}\}}<\infty$ perché coincide con $\card{\{H\st H<\Gal[K]{L}\}}$.
\item $K\subseteq L$ finita $\implies$ $\card{\Gal[K]{L}}\le[L:K]<\infty$ $\implies$ $L^{\Gal[K]{L}}\subseteq L$ di Galois e $\card{\Gal[K]{L}}=[L:L^{\Gal[K]{L}}]\dvd [L:K]$ (perché $K\subseteq L^{\Gal[K]{L}}\subseteq L$, quindi $[L:K]=[L:L^{\Gal[K]{L}}][L^{\Gal[K]{L}}:K]$).
\end{itemize}
\end{osse}



%\section{Il gruppo di Galois di un polinomio}
%\begin{defi}
%$K$ campo, $0\ne f\in K[X]$. Il {\em gruppo di Galois} di $f$ su $K$ (ben definito a meno di isomorfismo) è $\Gal[K]{f}:=\Gal[K]{L}$ con $K\subseteq L$ campo di spezzamento di $f$.
%\end{defi}
%\begin{osse}
%$K\subseteq L$ campo di spezzamento di $f\in K[X]\setminus\{0\}$, $G:=\Gal[K]{f}$.
%\begin{itemize}
%\item $K$ perfetto $\implies$ $K\subseteq L$ di Galois $\implies$ $\card{G}=[L:K]$.
%\item $R:=\{\alpha\in L\st f(\alpha)=0\}$ $\implies$ $n:=\card{R}\le\deg(f)$. \\
%$\sigma\in G$, $\alpha\in R$ $\implies$ $\sigma(\alpha)\in R$, quindi si ottiene una funzione $G\to S(R)\iso S_n$, $\sigma\mapsto\sigma\rest{R}$, che è un omomorfismo iniettivo (perché $L=K(R)$) $\implies$ $G\iso G'< S_n$ ($\implies$ $\card{G}\dvd n!$).
%\item $K$ perfetto, $f$ irriducibile $\implies$ $\deg(f)=n$ e $n\dvd\card{G}\dvd n!$.
%\end{itemize}
%\end{osse}
%
%Esempi: $K$ perfetto, $f\in K[X]$ irriducibile, $n:=\deg(f)$, $G:=\Gal[K]{f}$.
%\begin{itemize}
%\item $n=2$ $\implies$ $2\dvd\card{G}\dvd2!$ $\implies$ $\card{G}=2$ $\implies$ $G\iso C_2$.
%\item $n=3$ $\implies$ $3\dvd\card{G}\dvd3!$ $\implies$ $\card{G}=3$ o $6$ $\implies$ $G\iso C_3$ o $S_3$ (perché $G\iso G'<S_3$).
%\smallskip
%
%$f=X^3-2$ $\implies$ $G\iso S_3$ se $K=\Q$, $G\iso C_3$ se $K=\Q(\omega)$.
%\item $n=4$ $\implies$ $4\dvd\card{G}\dvd4!$ $\implies$ $\card{G}=4$, $8$, $12$ o $24$ $\implies$ $G\iso C_4$, $C_2^2$, $D_4$, $A_4$ o $S_4$ (perché $G\iso G'<S_4$).
%\smallskip
%
%$f=X^4-10X^2+1=\polmin_{\alpha,\Q}$ con $\alpha=\sqrt{2}+\sqrt{3}$ $\implies$ $G\iso C_2^2$: $\Q\subset\Q(\alpha)=\Q(\sqrt{2},\sqrt{3})$ normale (perché campo di spezzamento di $(X^2-2)(X^2-3)$) $\implies$ $f$ si spezza su $\Q(\alpha)$ $\implies$ $\Q\subset\Q(\alpha)$ campo di spezzamento di $f$ $\implies$ $G=\Gal[\Q]{\Q(\alpha)}$ $\implies$ $\card{G}=[\Q(\alpha):\Q]=\deg{f}=4$ e $G\niso C_4$ perché $\sigma\in G=\Gal[\Q]{\Q(\sqrt{2},\sqrt{3})}$ $\implies$ $\sigma(\sqrt{2})=\pm\sqrt{2}$ e $\sigma(\sqrt{3})=\pm\sqrt{3}$ $\implies$ $\sigma^2(\sqrt{2})=\sqrt{2}$ e $\sigma^2(\sqrt{3})=\sqrt{3}$ $\implies$ $\sigma^2=\id_{\Q(\sqrt{2},\sqrt{3})}$.
%\end{itemize}



\section{Campi finiti}

$K$ campo finito $\implies$ $\car(K)=p$ primo.

$0<n:=[K:\F_p]<\infty$ $\implies$ $K\iso\F_p^n$ come $\F_p$-spazio vettoriale (quindi $K\iso C_p^n$ come gruppo abeliano) $\implies$ $\card{K}=p^n$.
\begin{teor}
$\all p$ primo e $\all n>0$ $\exiun$ a meno di isomorfismo un campo $\F_{p^n}$ di ordine $p^n$; inoltre $\F_{p^n}$ è campo di spezzamento di $X^{p^n}-X$ su $\F_p$.
\end{teor}
\begin{proof}
$\F_p\subseteq\F_{p^n}$ campo di spezzamento di $X^{p^n}-X$ $\implies$

$R:=\{\alpha\in\F_{p^n}\st\alpha\text{ radice di }X^{p^n}-X\}=\{\alpha\in\F_{p^n}\st\Fro^n(\alpha)=\alpha\}$

sottocampo di $\F_{p^n}$ $\implies$ $\F_{p^n}=\F_p(R)=R$.

$(X^{p^n}-X)'=-1$ non ha radici $\implies$ $X^{p^n}-X$ non ha radici multiple $\implies$ $\card{\F_{p^n}}=\card{R}=\deg(X^{p^n}-X)=p^n$.

$K$ altro campo di ordine $p^n$ $\implies$ $\alpha^{p^n-1}=1$ $\all\alpha\in K^*$ (per il teorema di Lagrange) $\implies$ ogni elemento di $K$ è radice di $X^{p^n}-X$ $\implies$ $\prod_{\alpha\in K}(X-\alpha)\dvd(X^{p^n}-X)$ $\implies$ $X^{p^n}-X=\prod_{\alpha\in K}(X-\alpha)$ $\implies$ $\F_p\subseteq K$ campo di spezzamento di $X^{p^n}-X$.
\end{proof}

Se $n,m>0$, esiste un'estensione $\F_{p^n}\subseteq\F_{p^m}$ $\iff$ $n\dvd m$:
\begin{itemize}
\item[$\implies$] $d:=[\F_{p^m}:\F_{p^n}]$ $\implies$ $\F_{p^m}\iso\F_{p^n}^d$ (come $\F_{p^n}$-spazi vettoriali) $\implies$ $p^m=\card{\F_{p^m}}=\card{\F_{p^n}^d}=(p^n)^d=p^{nd}$ $\implies$ $m=nd$;
\item[$\impliedby$] $\F_{p^n}=\{\alpha\in\calg{\F}_p\st\alpha^{p^n}=\Fro^n(\alpha)=\alpha\}\subseteq\F_{p^m}$ perché, se $\Fro^n(\alpha)=\alpha$, allora $\Fro^m(\alpha)=(\Fro^n)^{m/n}(\alpha)=\alpha$.
\end{itemize}

\begin{coro}
$n\dvd m$ $\implies$ $\F_{p^n}\subseteq\F_{p^m}$ di Galois e $\Gal[\F_{p^n}]{\F_{p^m}}=\gen{\Fro^n}\iso C_{m/n}$.
\end{coro}
\begin{proof}
$\F_{p^n}\subseteq\F_{p^m}$ è di Galois perché campo di spezzamento di $X^{p^m}-X$ (e $\F_{p^n}$ è perfetto) $\implies$ $\card{\Gal[\F_{p^n}]{\F_{p^m}}}=[\F_{p^m}:\F_{p^n}]=m/n$.

$\F_{p^n}=\{\alpha\in\F_{p^m}\st\Fro^n(\alpha)=\alpha\}$ $\implies$ $\Fro^n\in\Gal[\F_{p^n}]{\F_{p^m}}$, e basta dimostrare $\ord(\Fro^n)\ge m/n$, cioè $\ord(\Fro)\ge m$ in $\Gal{\F_{p^m}}$, vero perché $0<i<m$ $\implies$ $\card{\{\alpha\in\F_{p^m}\st\Fro^i(\alpha)=\alpha^{p^i}=\alpha\}}\le p^i<p^m$ $\implies$ $\Fro^i\ne\id_{\F_{p^m}}$.
\end{proof}


$p$ primo, $n>0$, $q:=p^n$, $0\ne f\in\F_q[X]$, $G:=\Gal[\F_q]{f}$.
\begin{itemize}
\item $f$ irriducibile, $d:=\deg(f)$ $\implies$ $\F_q\subseteq\F_{q^d}$ campo di spezzamento di $f$ ($\implies$ $G\iso C_d$):

$\alpha\in\calg{\F}_p$ radice di $f$ $\implies$ $[\F_q(\alpha):\F_q]=d$ $\implies$ $\F_q(\alpha)=\F_{q^d}$.
\item in generale $f=\prod_{i=1}^kf_i$ con $f_i$ irriducibile, $d_i:=\deg(f_i)$ $\all i=1,\dots,k$ $\implies$ $d:=\mcm(d_1,\dots,d_k)$ tale che $\F_q\subseteq\F_{q^d}$ campo di spezzamento di $f$ ($\implies$ $G\iso C_d$):

per il punto precedente $\F_{q^{d_i}}$ è campo di spezzamento di $f_i$ su $\F_q$, quindi $f$ si spezza su $\F_{q^{d'}}$ $\iff$ $\F_{q^{d_i}}\subseteq\F_{q^{d'}}$ $\all i=1,\dots,k$ $\iff$ $d_i\dvd d'$ $\all i=1,\dots,k$ $\iff$ $d\dvd d'$.
\end{itemize}



%\section{Il gruppo di Galois di $X^n-1$}

$n>0$, $\car(K)\ndvd n$, $K\subseteq L$ campo di spezzamento di $X^n-1$.
\begin{itemize}
\item $(X^n-1)'=nX^{n-1}\ne0$ ha solo la radice $0$ (che non è radice di $X^n-1$) $\implies$ $X^n-1$ non ha radici multiple in $L$ $\implies$ $R:=\{\alpha\in L\st\alpha^n=1\}$ tale che $\card{R}=n$.
\item $R<L^*$ $\implies$ $R$ ciclico $\implies$ $\exi\omega\in R$ tale che $R=\gen{\omega}$ (quindi $\ord(\omega)=n$ in $L^*$, e si dice che $\omega$ è una radice $n$-esima {\em primitiva} dell'unit\`a; per esempio $\omega=e^{(2\pi i)/n}$ se $K\subseteq\C$).
\item $L=K(R)=K(\omega)$ $\implies$ $\card{\Gal[K]{L}}=\card{R'}$ con $R':=\{\alpha\in L\st\polmin_{\omega,K}(\alpha)=0\}\subseteq R$ (perché $\polmin_{\omega,K}\dvd(X^n-1)$).
\item $\polmin_{\omega,K}$ si spezza su $L$ e non ha radici multiple $\implies$ $\card{\Gal[K]{L}}=\deg(\polmin_{\omega,K})=[L:K]$ $\implies$ $K\subseteq L$ di Galois.
\item La funzione $\Gal[K]{L}\to\Aut(R)<S(R)$, $\sigma\mapsto\sigma\rest{R}$ è ben definita e è un omomorfismo iniettivo di gruppi.
\item $\Gal[K]{X^n-1}=\Gal[K]{L}\iso G<\Z/n\Z^*\iso\Aut(R)$ $\implies$ $G$ abeliano e $\card{G}\dvd\varphi(n)$.
\end{itemize}



%\section{Polinomi ciclotomici}
$\alpha\in R'$ $\implies$ $\exi\sigma\in\Gal[K]{L}$ tale che $\alpha=\sigma(\omega)$ $\implies$ $\ord(\alpha)=\ord(\omega)=n$ $\implies$ $\exi\cl{j}\in\Z/n\Z^*$ tale che $\alpha=\omega^j$ $\implies$
\[
\polmin_{\omega,K}=\prod_{\alpha\in R'}(X-\alpha)\dvd\Phi_n:=\prod_{\cl{j}\in\Z/n\Z^*}(X-\omega^j)\in L[X],
\]
dove $\Phi_n$ è detto $n$-esimo {\em polinomio ciclotomico}. Chiaramente
\[
\Phi_n\dvd(X^n-1)=\prod_{\cl{j}\in\Z/n\Z}(X-\omega^j)
\]
e $\deg(\Phi_n)=\varphi(n)$. 
\begin{teor}
\begin{enumerate}
\item $\Phi_n\in K[X]$.
\item $K=\Q$ $\implies$ $\Phi_n\in\Q[X]$ irriducibile.
\end{enumerate}
\end{teor}
\begin{coro}
$\polmin_{\omega,\Q}=\Phi_n$ e $\Gal[\Q]{X^n-1}\iso\Z/n\Z^*$.
\end{coro}



\section{Discriminante}

\begin{itemize}
\item $K$ campo, $0\ne f\in K[X]$, $K\subseteq L$ campo di spezzamento di $f$.
\item $n:=\deg(f)$, $\alpha_1,\dots,\alpha_n\in L$ radici di $f$ $\implies$
\[
\delta:=\prod_{1\le i<j\le n}(\alpha_i-\alpha_j)\in L
\]
è ben definito a meno del segno (dipende dall'ordine delle radici), e chiaramente $\delta\ne0$ $\iff$ $f$ non ha radici multiple.
\item Il {\em discriminante} di $f$ è $\Delta=\Delta(f):=\delta^2\in L$ (ben definito e tale che $\Delta\ne0$ $\iff$ $f$ non ha radici multiple).
\end{itemize}
\begin{osse}
$\sigma(\delta)=\varepsilon(\sigma\rest{R})\delta$ (con $R:=\{\alpha_1,\dots,\alpha_n\}$) $\all\sigma\in\Gal[K]{f}=\Gal[K]{L}$:

posso supporre $\delta\ne0$ (quindi $\card{R}=n$), e allora per definizione di segno di una permutazione in $S(R)\iso S_n$
\[
\sigma(\delta)=\prod_{1\le i<j\le n}(\sigma(\alpha_i)-\sigma(\alpha_j))=\prod_{1\le i<j\le n}(\sigma\rest{R}(\alpha_i)-\sigma\rest{R}(\alpha_j))=\varepsilon(\sigma\rest{R})\delta.
\]
\end{osse}



%\section{Propriet\`a del discriminante}
\begin{prop}
$K$ perfetto $\implies$ $\Delta=\Delta(f)\in K$. Se inoltre $\car(K)\ne2$, $f$ non ha radici multiple e identifico $\Gal[K]{f}$ a un sottogruppo di $S_n\iso S(R)$, allora $\Gal[K]{f}\subseteq A_n$ $\iff$ $\delta\in K$ $\iff$ $\Delta$ è un quadrato in $K$.
\end{prop}
\begin{proof}
\begin{itemize}
\item $K\subseteq L$ di Galois $\implies$ $K=L^{\Gal[K]{L}}$. Dunque, dato $\alpha\in L$, $\alpha\in K$ $\iff$ $\sigma(\alpha)=\alpha$ $\all\sigma\in\Gal[K]{L}=\Gal[K]{f}$.
\item $\sigma(\Delta)=\sigma(\delta^2)=\sigma(\delta)^2=(\varepsilon(\sigma\rest{R})\delta)^2=\varepsilon(\sigma\rest{R})^2\delta^2=\delta^2=\Delta$ $\all\sigma\in\Gal[K]{f}$ $\implies$ $\Delta\in K$.
\item $\Gal[K]{f}\subseteq A_n$ $\implies$ $\sigma(\delta)=\delta$ $\all\sigma\in\Gal[K]{f}$ $\implies$ $\delta\in K$.
\item $\delta\in K$, $f$ senza radici multiple $\implies$ $\delta\in K^*$ e $\all\sigma\in\Gal[K]{f}$ $\delta=\sigma(\delta)=\varepsilon(\sigma\rest{R})\delta$ $\implies$ $\varepsilon(\sigma\rest{R})_K=1_K$ $\implies$ $\varepsilon(\sigma\rest{R})=1$ (cioè $\Gal[K]{f}\subseteq A_n$) se $\car(K)\ne2$.
\item Chiaramente $\delta\in K$ $\iff$ $\Delta=\delta^2$ è un quadrato in $K$.
\end{itemize}
\end{proof}



%\section{Discriminante dei polinomi di grado $2$ e $3$}
\begin{itemize}
\item Si pu\`o dimostrare che $\Delta(f)$ è esprimibile come polinomio valutato nei coefficienti di $f$.
\item $\Delta(X^2+aX+b)=a^2-4b$:
\smallskip

$X^2+aX+b=(X-\alpha)(X-\beta)=X^2-(\alpha+\beta)X+\alpha\beta$ $\implies$ $a=-\alpha-\beta$, $b=\alpha\beta$;

$\delta=\alpha-\beta$ $\implies$ $\Delta=(\alpha-\beta)^2=(\alpha+\beta)^2-4\alpha\beta=a^2-4b$.
\item $\Delta(X^3+aX+b)=-4a^3-27b^2$:
\smallskip

$X^3+aX+b=(X-\alpha)(X-\beta)(X-\gamma)=X^3-(\alpha+\beta+\gamma)X^2+(\alpha\beta+\alpha\gamma+\beta\gamma)X-\alpha\beta\gamma$ $\implies$ $\alpha+\beta+\gamma=0$, $a=\alpha\beta+\alpha\gamma+\beta\gamma$, $b=-\alpha\beta\gamma$ $\implies$ $a=-(\alpha^2+\alpha\beta+\beta^2)$, $b=\alpha\beta(\alpha+\beta)$;

$\delta=(\alpha-\beta)(\alpha-\gamma)(\beta-\gamma)=(\alpha-\beta)(2\alpha+\beta)(\alpha+2\beta)=2\alpha^3+3\alpha^2\beta-3\alpha\beta^2-2\beta^3$ $\implies$ $\Delta=(2\alpha^3+3\alpha^2\beta-3\alpha\beta^2-2\beta^3)^2=4\alpha^6+12\alpha^5\beta-3\alpha^4\beta^2-26\alpha^3\beta^3-3\alpha^2\beta^4+12\alpha\beta^5+4\beta^6=-4a^3-27b^2$.
\end{itemize}



%\section{Gruppo di Galois di un polinomio di grado $3$}

$K$ perfetto, $\car(K)\ne2$, $\deg(f)=3$, $f$ irriducibile in $K[X]$ $\implies$
$\Gal[K]{f}\iso
\begin{cases}
C_3 & \text{se $\Delta(f)$ è un quadrato in $K$}\\
S_3 & \text{altrimenti.}
\end{cases}
$
\begin{esem}
\begin{itemize}
\item $f=X^3-3X+1$ irriducibile in $\Q[X]$ (non ha radici in $\Q$) $\implies$ $\Delta=-4(-3)^3-27\cdot1^2=81=9^2$ $\implies$ $\Gal[\Q]{f}\iso C_3$.
\item $f=X^3+3X+1$ irriducibile in $\Q[X]$ (non ha radici in $\Q$) $\implies$ $\Delta=-4\cdot3^3-27\cdot1^2=-135<0$ $\implies$ $\Gal[\Q]{f}\iso S_3$.
\end{itemize}
\end{esem}
\begin{osse}
$\car(K)\ne3$, $f(X)=X^3+aX^2+bX+c\in K[X]$ $\implies$ con la sostituzione $X=Y-a/3$ si ottiene $f(X)=f(Y-a/3)=:g(Y)$ con $g(Y)=Y^3+a'Y+b'\in K[Y]$. Chiaramente $\alpha\in L$ (campo di spezzamento di $f$ su $K$) è radice di $g$ $\iff$ $\alpha-a/3$ è radice di $f$ $\implies$ $L$ è campo di spezzamento di $g$ su $K$ $\implies$ $\Gal[K]{f}\iso\Gal[K]{g}$.
\end{osse}



\section{Esercizi di riepilogo}

\begin{eser}
Determinare il gruppo di Galois $G$ di $f:=X^5-X+3$ su $\F_q$ per $q=2,3,4,5$.
\end{eser}

\begin{proof}[Svolgimento]
In ogni caso $G\iso C_d$ se $\F_q\subseteq\F_{q^d}$ campo di spezzamento di $f$, e $d=\mcm(d_1,\dots,d_r)$ se $f=\prod_{i=1}^rf_i$ con $f_1,\dots,f_r$ irriducibili.
\begin{itemize}
\item[$q=2$] $f$ non ha radici (perché $f(\cl{0})=f(\cl{1})=\cl{1}$), ma è divisibile per $X^2+X+1$ (l'unico irriducibile di grado $2$ in $\F_2[X]$) e risulta $f=(X^2+X+1)(X^3+X^2+1)$ $\implies$ $d=\mcm(2,3)=6$.
\item[$q=3$] $f=X^5-X=X(X-1)(X+1)(X^2+1)$ $\implies$ $d=2$.
\item[$q=4$] $\F_{2^6=4^3}$ campo di spezzamento di $f$ su $\F_2$ $\implies$ anche su $\F_4$ $\implies$ $d=3$.
\item[$q=5$] $\alpha\in\F_{5^d}$ $\implies$ $f(\alpha)=\Fro(\alpha)-\alpha+\cl{3}$ $\implies$ $f(a)=\cl{3}\ne\cl{0}$ se $a\in\F_5$ e $f(\alpha+a)=f(\alpha)$ $\all\alpha\in\F_{5^d}$ e $\all a\in\F_5$ $\implies$ $\F_{5^d}=\F_5(\alpha)$ se $\alpha$ radice di $f$ $\implies$ $d=\deg(\polmin_{\alpha,\F_5})$ non dipende dalla radice $\alpha$ di $f$ $\implies$ $f$ irriducibile in $\F_5[X]$ (non ha radici in $\F_5$ e non pu\`o essere $f=gh$ in $\F_5[X]$ con $\deg(g)=2$ e $\deg(h)=3$) $\implies$ $d=5$.\qedhere
\end{itemize}
\end{proof}


\begin{eser}[Esercizio sul gruppo di Galois di $X^n-2$]
$n>1$, $\alpha:=\sqrt[n]{2}\in\R_{>0}$, $\omega:=e^{(2\pi i)/n}\in\C$, $G:=\Gal[\Q]{X^n-2}$.
\begin{enumerate}
\item $\Q\subseteq\Q(\alpha,\omega)$ campo di spezzamento di $X^n-2$.
\item $n$ primo $\implies$ $\card{G}=n\varphi(n)=n(n-1)$.
\item $n=4$ o $6$ $\implies$ $\card{G}=n\varphi(n)$.
\item $n=8$ $\implies$ $\card{G}<n\varphi(n)$.
\item $\card{G}=n\varphi(n)$ $\implies$ $G\iso C_n\rtimes_{\theta}\Z/n\Z^*$ con $\theta : \Z/n\Z^*\to\Aut(C_n)$ isomorfismo.
\end{enumerate}
\end{eser}

\begin{proof}[Svolgimento]
\begin{enumerate}
\item Le radici in $\C$ di $X^n-2$ sono $\alpha\omega^j$ per $j=0,\dots,n-1$ $\implies$ il campo di spezzamento in $\C$ di $X^n-2$ su $\Q$ è $L:=\Q(\alpha\omega^j\st j=0,\dots,n-1)=\Q(\alpha,\omega)$.
\item $\Q\subseteq L$ di Galois, $G=\Gal[\Q]{L}$ $\implies$ $\card{G}=[L:\Q]$. \\
$X^n-2$ irriducibile per Eisenstein $\implies$ $\polmin_{\alpha,\Q}=X^n-2$ $\implies$ $[\Q(\alpha):\Q]=\deg(X^n-2)=n$. \\
$\polmin_{\omega,\Q}=\Phi_n$ $\implies$ $[\Q(\omega):\Q]=\deg(\Phi_n)=\varphi(n)=n-1$. \\
$\mcd(n,n-1)=1$ $\implies$ $[L:\Q]=n(n-1)$.
\item $\varphi(4)=\varphi(6)=2$ $\implies$ $[\Q(\omega):\Q]=\varphi(n)=2$ $\implies$ $n=\mcm(n,2)\dvd[L:\Q]\le2n$ $\implies$ $[L:\Q]=2n=n\varphi(n)$ (altrimenti $[L:\Q]=n$ $\implies$ $\omega\in L=\Q(\alpha)\subset\R$, assurdo).
\item $\omega=\sqrt{2}(1+i)/2$, $\omega^2=i$ $\implies$ $\sqrt{2}\omega=1+i=1+\omega^2$ $\implies$ $\omega$ radice di $X^2-\sqrt{2}X+1\in\Q(\alpha)[X]$ (perché $\sqrt{2}=\alpha^4$) $\implies$ $[L:\Q]=[L:\Q(\alpha)][\Q(\alpha):\Q]\le2\cdot8=16<32=8\varphi(8)$.
\item $H:=\Gal[\Q(\alpha)]{L}<G$ tale che $\card{H}=\card{G}/[\Q(\alpha):\Q]=\varphi(n)$, $H':=\Gal[\Q(\omega)]{L}\normal G$ (perché $\Q\subseteq\Q(\omega)$ normale) tale che $\card{H'}=\card{G}/[\Q(\omega):\Q]=n$ e $H\cap H'=\{1\}$ $\implies$ $\card{(HH')}=n\varphi(n)=\card{G}$ $\implies$ $G=HH'$ $\implies$ $G=H'\rtimes H$. \\
$H'=\{\sigma_{\cl{j}}\st\cl{j}\in\Z/n\Z\}$ con $\sigma_{\cl{j}}(\alpha)=\alpha\omega^j$ (e $\sigma_{\cl{j}}(\omega)=\omega$) $\implies$ $\sigma_{\cl{j}}=\sigma_{\cl{1}}^j$ $\all \cl{j}\in\Z/n\Z$ $\implies$ $H'=\gen{\sigma_{\cl{1}}}\iso C_n$. \\
$H\iso G/H'\iso\Gal[\Q]{\Q(\omega)}\iso\Z/n\Z^*$ $\implies$ $G\iso C_n\rtimes_{\theta}\Z/n\Z^*$ con $\theta : \Z/n\Z^*\to\Aut(C_n)\iso\Z/n\Z^*$ omomorfismo iniettivo (quindi isomorfismo) perché $\tau\in H$ $\implies$ $\exi\cl{l}\in\Z/n\Z^*$ tale che $\tau(\omega)=\omega^l$ (e $\tau(\alpha)=\alpha$) $\implies$ $\tau\sigma_{\cl{1}}\tau^{-1}=\sigma_{\cl{l}}=\sigma_{\cl{1}}$ $\iff$ $\tau=1$ \qedhere
\end{enumerate}
\end{proof}


\begin{eser}[Polinomi con gruppo di Galois $S_n$]
$K$ campo, $0\ne f\in K[X]$ tale che $\deg(f)=n>0$ e $\Gal[K]{f}\iso S_n$; \\
$\alpha$ radice di $f$ (in un campo di spezzamento $L$ di $f$ su $K$).
\begin{enumerate}
\item $f$ è irriducibile in $K[X]$.
\item $n>2$ $\implies$ $\Gal[K]{K(\alpha)}=\{1\}$.
\item $n>3$ $\implies$ $\alpha^n\not\in K$.
\end{enumerate}
\end{eser}

\begin{proof}[Svolgimento]
\begin{enumerate}
\item $\Gal[K]{f}\iso S_n$ $\implies$ le radici $\alpha=\alpha_1\dots,\alpha_n\in L$ di $f$ sono distinte e $\all i=1,\dots,n$ $\exi\sigma\in\Gal[K]{f}$ tale che $\sigma(\alpha)=\alpha_i$ $\implies$ $\alpha_i$ radice di $\polmin_{\alpha,K}$ $\implies$ $f\dvd\polmin_{\alpha,K}$ $\implies$ $f$ irriducibile.
\item $\sigma\in\Gal[K]{K(\alpha)}$ $\implies$ $\exi i=1,\dots,n$ tale che $\sigma(\alpha)=\alpha_i$, e basta dimostrare $i=1$. Per assurdo $i=2$ $\implies$ $K(\alpha)\subseteq L$ campo di spezzamento di $\prod_{i=3}^n(X-\alpha_i)$ $\implies$ $[L:K]=[L:K(\alpha)][K(\alpha):K]\le(n-2)!n<n!$, assurdo perché $[L:K]\ge\card{\Gal[K]{L}}=n!$.
\item Per assurdo $\alpha^n=a\in K$ $\implies$ posso supporre $f=X^n-a$ $\implies$ $L=K(\alpha,\omega)$ con $\gen{\omega}=\{\beta\in L\st\beta^n=1\}<L^*$ $\implies$ $[L:K]\le[K(\alpha):K][K(\omega):K]\le n(n-1)<n!$, assurdo. \qedhere
\end{enumerate}
\end{proof}


\begin{eser}[Gruppi di Galois di polinomi biquadratici]
Determinare campo di spezzamento $\Q\subseteq L$ e gruppo di Galois $G$ di $f$ su $\Q$ nei seguenti casi:
\begin{enumerate}
\item $f=X^4-4X^2+2$;
\item $f=X^4-4X^2-2$.
\end{enumerate}
\end{eser}

\begin{proof}[Svolgimento]
\begin{enumerate}
\item $f$ irriducibile per Eisenstein. \\
$f(X)=g(X^2)$ con $g(Y):=Y^2-4Y+2$ che ha radici $2\pm\sqrt{2}\in\R_{>0}$ $\implies$ le radici di $f$ sono $\pm\alpha,\pm\beta$ con $\alpha:=\sqrt{2+\sqrt{2}},\beta:=\sqrt{2-\sqrt{2}}\in\R_{>0}$ $\implies$ $L=\Q(\alpha,\beta)$. \\
$\alpha^2-2=\sqrt{2}=\alpha\beta$ $\implies$ $\beta=(\alpha^2-2)/\alpha\in\Q(\alpha)$ $\implies$ $L=\Q(\alpha)$ $\implies$ $\card{G}=[L:\Q]=\deg(\polmin_{\alpha,\Q})=\deg(f)=4$. \\
$\exi\sigma\in G=\Gal[\Q]{\Q(\alpha)}$ tale che $\sigma(\alpha)=\beta$ (perché $\beta$ radice di $\polmin_{\alpha,\Q}=f$) $\implies$
\[
\sigma^2(\alpha)=\sigma(\beta)=\sigma\biggl(\frac{\alpha^2-2}{\alpha}\biggr)=\frac{\beta^2-2}{\beta}=\frac{-\sqrt{2}}{\beta}=-\alpha
\]
$\implies$ $\sigma^2\ne\id_L$ $\implies$ $G\iso C_4$.
\item $f$ irriducibile per Eisenstein. \\
$f(X)=g(X^2)$ con $g(Y):=Y^2-4Y-2$ che ha radici $2\pm\sqrt{6}\in\R$ $\implies$ le radici di $f$ sono $\pm\alpha,\pm\beta i$ con $\alpha:=\sqrt{\sqrt{6}+2},\beta:=\sqrt{\sqrt{6}-2}\in\R_{>0}$ $\implies$ $L=\Q(\alpha,\beta i)$. \\
$\alpha\beta=\sqrt{2}$ $\implies$ $\alpha\beta i=\sqrt{2}i$ $\implies$ $L=\Q(\alpha,\sqrt{2}i)$. \\
$[\Q(\alpha):\Q]=\deg(\polmin_{\alpha,\Q})=\deg(f)=4$, $[\Q(\sqrt{2}i):\Q]=\deg(\polmin_{\sqrt{2}i,\Q})=2$ (perché $\polmin_{\sqrt{2}i,\Q}=X^2+2$) $\implies$ 
\[
\mcm(4,2)=4\dvd\card{G}=[\Q(\alpha,\sqrt{2}i):\Q]\le4\cdot2=8
\]
e non pu\`o essere $[\Q(\alpha,\sqrt{2}i):\Q]=4$ (perché $\sqrt{2}i\not\in\Q(\alpha)\subset\R$) $\implies$ $\card{G}=8$. \\
$G\iso G'<S_4$ $\implies$ $G\iso D_4$. \qedhere
\end{enumerate}
\end{proof}

\end{document}






