% !TEX program = lualatex
% !TEX spellcheck = en_GB
% !TEX root = ../groups.tex

\section{Cyclic groups}

For \(G\) group and \(x \in G\), we denote \(\gen x\) the smallest subgroup of \(G\) that owns \(x\).

\begin{definition}[Cyclic groups]
We say a group \(G\) is {\em cyclic} whenever there exists a \(x \in G\) such that \(G = \gen{x}\).
\end{definition}

\begin{definition}
Provided a group \(G\) and \(x \in G\), we provide the exponentiation function
\[x^\bullet : \zz \to G \,, \ n \to x^n\]
defined by recursion as follows:
\[x^n \coloneq \begin{cases}
1        & \text{if } n = 0 \\
x^{n-1}x & \text{if } n \ge 1 \\
\inv{(x^{-n})}   & \text{if } n \le -1.
\end{cases}\]
\end{definition}

When the symbol \(+\) is selected to indicate the operation of group and the identity is written \(0\), it is usually preferred \(nx\) instead of \(x^n\). In that case \(0x = 0\).

\begin{proposition}
Let \(G\) be a group and \(x \in G\). Then \(\gen x = \set{x^j \mid j \in \zz}\).
\end{proposition}

\begin{proof}
For sure \(x^i \in \gen x\) for every \(i \in \zz\), hence \(\set{x^i \mid i \in \zz} \subseteq \gen x\). Besides, \(\set{x^j \mid j \in \zz}\) is a group which owns \(x\), because \(x^1 = x\): thus \(\gen x \subseteq \set{x^j \mid j \in \zz}\) as well.
\end{proof}

Thus, to prove a certain group \(G\) is cyclic you can show there is some \(x \in G\) with the property: for every \(a \in G\) there is a \(n \in \zz\) such that \(a = x^n\).

\begin{example}
The subgroup of \(\zz\) generated by one of its elements \(a\) is \(a \zz\).
\end{example}

%\begin{corollary}
%Let \(G\) be a group and \(x \in G\). Then these facts are equivalent:
%\begin{tcbenum}
%\item \(G = \gen x\)
%\item for every \(a \in G\) there is a \(n \in \zz\) such that \(a = x^n\).
%\end{tcbenum}
%\end{corollary}

So, a group has cyclic subgroups \(\gen x\) for \(x \in G\). In general, a group might have non cyclic subgroups, but this is not the case if \(G\) is cyclic.

\begin{proposition}\label{prop:SubgroupsOfCyclicGroupsAreCyclic}
Subgroups of cyclic groups are themselves cyclic.
\end{proposition}

\begin{proof}
Consider a cyclic group \(G\), generated by some \(x \in G\). Observe that \(G\) hs two cyclic subgroups: itself and \(\set{1}\). For this reason, let us focus on subgroups \(H\) that are neither \(\set 1\) nor \(G\). Hence
\[H = \set{x^n \mid n \in I} \text{ for some } I \subseteq \zz .\]
First of all, we note that if \(n \in I\), then also \(-n \in I\): indeed, if \(x^n \in H\), then also its inverse \(x^{-n}\) belongs to \(H\), being \(H\) a subgroup. Furthermore, some \(0 \ne n \in I\), since we have assumed \(H\) is not trivial. Recalling now that \(\nn\) is well-ordered, we can introduce the number
\[s := \min \set{i \in \nn^{\ge 1} \mid x^i \in H} .\]
Obviously, \(\gen{x^s} \subseteq H\), but the inverse inclusion is also true. For every \(n \in I\) we have \(q, r \in \zz\) such that \(0 \le r \le s\) and \(n =qs +r\). Consequently
\[x^r = x^{n-qs} = \underbrace{x^n}_{\in H} \underbrace{(x^s)^{-q}}_{\in H} .\]
If \(r \ge 1\), then \(r < s\) and \(x^r \in H\), which is in contrast with the definition of \(s\). Thus it must be necessarily \(r = 0\), that is \(n\) is a multiple of \(s\).
\end{proof}

In the particular case of \(\zz\), the subgroups of \(\zz\) those of the form \(n \zz\) for \(n \in \nn\), being \(n \zz = (-n)\zz\).

\begin{lemma}Let \(G\) be a group and \(x \in G\) such that \(\gen x\) is finite. Then
\[\set{i \in \nn^{\ge 1} \mid x^i = 1} \ne \nil.\]
\end{lemma}

\begin{proof}
Consider the function \(\nn \to \gen x\), \(i \to x^i\). Because \(\nn\) is infinite and \(\gen x\) is finite, this function cannot be injective. Thus there exists \(m, n \in \nn\) such that \(m \ne n\) and \(x^m = x^n\). One between \(m-n\) and \(n-m\) is positive, and in any case \(x^{m-n} = x^{n-m} = 1\).
\end{proof}

\(\nn\) is well ordered, and this associated with the previous lemma legitimate the following definition.

\begin{definition}[Order of elements]
Let \(G\) be a group and \(x \in G\) such that \(\gen x\) is finite. Then we call {\em order} of \(x\) the natural number
\[\ord x \coloneq \min \set{n \in \nn^{\ge 1} \mid x^n = 1}.\]
In that case \(x\) is said to be of \q{finite order}.
\end{definition}

\begin{exercise}
Let \(G\) be a finite group. Every subset of \(G\) closed under the operation of \(G\) is a subgroup.
\end{exercise}

\begin{proposition}\label{prop:OrdIsCardinality}
Let \(G\) be a group and \(x \in G\) of finite order. Then \(\ord x\) is the cardinality of \(\gen x\).
\end{proposition}

\begin{proof}
Consider \(I \coloneq \set{0, \dots{}, \ord x-1}\) and the function
\[f : I \to \gen x\,, \ f(n) \coloneq x^n.\]
Take \(f(j) = f(k)\), that is \(x^j = x^k\). Without loss of generality, let us assume \(j \le k\). Then \(x^{k-j} = 1\). It must be \(j = k\), because otherwise \(0 < k-j < \ord x\) while \(x^{k-j} = 1\), absurd. Hence \(f\) is injective.\newline
For every \(s \in \zz\) there exist \(q, r \in \zz\) such that \(0 \le r < \ord x\) and \(s = q \ord x +r\). Now
\[x^s = x^{q \ord x +r} = \left(x^{\ord x}\right)^q x^r = x^r.\]
\(f\) is surjective too.\newline
To put all in a nutshell: we have found a bijection from \(I\), which has \(\ord x\) elements, to \(\gen x\).
\end{proof}

\begin{proposition}\label{prop:CyclicIffOrdIsCard}
A finite group \(G\) is cyclic if and only if there exists \(x \in G\) such that \(\ord x = \lvert G \rvert\).
\end{proposition}

\begin{proof}
Half of the work is already done in Proposition~\ref{prop:OrdIsCardinality}. Suppose \(G\) has an element \(x\) such that \(\ord x = \lvert G \rvert\): then \(\gen x = \set{1, x, \dots{}, x^{n-1}} \subseteq G\); since they are both finite and have the same cardinality, they must be equal.
\end{proof}

\begin{proposition}\label{prop:OrdDividesN}
Let \(G\) be a group and \(x \in G\) of finite group. Then
\[x^n = 1 \lrarr \ord x \text{ divides } n.\]
\end{proposition}

\begin{proof}
One part is obvious. Now suppose \(x^n = 1\). There exist \(q, r \in \zz\) such that \(0 \le r < \ord x\) and \(n = q \ord x +r\). Then \(1 = x^n = x^r\). By the definition of order of element, \(r = 0\) and so \(n\) is a multiple of \(\ord x\).
\end{proof}

\begin{proposition}
Let \(G\) be a group and \(x \in G\) of finite order. Then
\[\ord\left(x^k\right) = \frac{\ord x}{\gcd(\ord x, k)} \quad \text{for every } k \in \zz.\]
\end{proposition}

\begin{proof} By definition of order of elements, we have find the minimum of the set \(\set{n \in \nn^{\ge 1} \mid \left(x^k\right)^n = 1}\). We have 
\begin{align*}
\set{ n \in \nn^{\ge 1} \mid x^{kn} = 1} & = \set{n \in \nn^{\ge 1} \mid \ord x \text{ divides } kn} = \\
& = \set{n \in \nn^{\ge 1} \mid \frac{\ord x}{\gcd(\ord x, k)} \text{ divides } n},
\end{align*}
whose minimum is \(\frac{\ord x}{\gcd(\ord x, k)}\).
\end{proof}

\begin{corollary}
Let \(G\) be a finite cyclic group of cardinality \(s\). Then there exist exactly \(\phi(s)\) elements \(x \in G\) such that \(G = \gen x\).
\end{corollary}

\begin{proof}
So \(s = \ord x\). We have to seek for which \(r \in \set{1, \dots{}, s-1}\) we have \(G = \gen{x^r}\): this occurs, by Proposition~\ref{prop:CyclicIffOrdIsCard}, if and only if \(\ord \left(x^r\right) = s\), which itself is equivalent to \(\gcd(s, r) = 1\).
\end{proof}

\begin{corollary}
For \(a, n \in \zz\), with \(n \ge 2\), we have
\[\ord [a]_n = \frac{n}{\gcd(a, n)}.\]
(Here, \([a]_n\) is an element of \(\zz/n\zz\).)
\end{corollary}

\begin{proposition}\label{prop:CyclicOneAndOnlySubgroup}
Let \(G\) be a finite cyclic group with cardinality \(s\). Then for every \(n \in \nn^{\ge 1}\) that divides \(s\) there exists one and only subgroup of \(G\) with cardinality \(n\).
\end{proposition}

\begin{proof}
Above all, \(G = \gen x\) for some \(x \in G\) with \(\ord x = s\). Then for every \(n \in \nn^{\ge 1}\) that divides \(s\) we have
\[\ord\left(x^{\frac s n}\right) = \frac{s}{\gcd\left(s, \frac s n\right)} = n,\]
that is the subgroup \(\gen{x^{\frac s n}}\) of \(G\) has \(n\) elements. Now, consider a subgroup \(K\) of \(G\) with cardinality \(n\). By Proposition~\ref{prop:SubgroupsOfCyclicGroupsAreCyclic}, \(K\) is cyclic and \(K = \gen{x^l}\) for a suitable \(l \in \zz\). Hence
\[n = \ord{\left(x^l\right)} = \frac{s}{\gcd(s, l)}.\]
We have that \(l\) is a multiple of \(\frac s n\), and so \(K \subseteq \gen{x^{\frac s n}}\). Since \(K\) and \(\gen{x^{\frac s n}}\) are both finite with the same cardinality, they are actually equal.
\end{proof}

\begin{corollary}\label{cor:CyclicGroupsHavePhiNElemsOfOrdN}
Let \(G\) be a finite cyclic group of cardinality \(s\). For every \(n \in \nn^{\ge 1}\) that divides \(s\) there are exactly \(\phi(n)\) elements of order \(n\).
\end{corollary}

\begin{exercise}
Prove that for \(G\) group, \(C_1, C_2 \subseteq G\) finite cyclic subgroups and \(p\) prime number if \(\abs{C_1} = \abs{C_2} = p\), then \(C_1 \cap C_2 = \set{1}\) or \(C_1 = C_2\).
\end{exercise}

