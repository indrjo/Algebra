% !TEX program = lualatex
% !TEX spellcheck = en_GB
% !TEX root = ../groups.tex

\section{Sylow Theorem}

\begin{proposition}[First Sylow Theorem]\label{prop:SylowTheorem1}
Let \(G\) be a finite group. For every prime number \(p\) and \(r \in \nn^{\ge 1}\) such that \(p^r\) divides \(\abs G\) there exists a subgroup of \(G\) of cardinality \(p^r\).
\end{proposition}

The proof of this lemma splits in two proofs: the first is for the case in which \(G\) is also abelian, while the second one drops this requirement. As you will observe, the first proof is just to avoid a nested proof by induction nested within another proof of induction. \NotaInterna{Is there another proof for this lemma?}

\begin{proof}[Proof, with \(G\) abelian] We use induction on the cardinality of \(G\). If \(G\) has \(2\) elements, the statement is true. Thanks to Proposition~\ref{prop:PreCauchysTheorem}, there exists a cyclic subgroup \(H\) of \(G\) with order \(p\). Since \(G\) is abelian, \(H\) is abelian (thus it is normal too), and so we have the abelian group \(G/H\) that has cardinality multiple of \(p^{r-1}\) (Proposition~\ref{prop:LagrangesTheorem}) and less then \(\abs G\). By inductive hypothesis, there is a subgroup \(K\) of \(G/H\) that has \(p^{r-1}\) elements; besides, \(K = K'/H\) for some \(K'\) subgroup of \(G\). We can conclude \(\abs {K'} = p^{r}\), again by Proposition~\ref{prop:LagrangesTheorem}.
\end{proof}

\begin{proof}[Proof of the general case]
Again by induction on \(\abs G\). The case in which \(G\) has \(2\) elements is trivial. We assume now the statement is true for all natural numbers less than \(n\), and prove it for any finite group \(G\) with \(n\) elements. We have considered the case in which \(G\) is abelian, hence we assume it is not. Moreover, let \(s, k \in \nn^{\ge 1}\) and a prime \(p \ge 2\) such that \(p \ndivides k\) and \(\abs G = p^s k\); in other words, \(p^s\) is the maximum power of \(p\). We show that \(G\) has a subgroup of cardinality \(p^r\) for \(1 \le r \le s\). Let \(R \subseteq G\) have exactly one element from every conjugacy class of \(G\) so that, by Proposition~\ref{prop:ClassFormulaForGroups}, we have
\[p^s k = \abs G = \abs{\mathcal Z G} + \sum_{x \in R \setminus \mathcal Z G} \frac{\abs G}{\abs{C_G(x)}} .\]
If \(p\) does not divide some \(\abs{G}/\abs{C_G(x)}\) with \(x \in R \setminus \mathcal Z G\), then \(\abs{C_G(x)}\) is a multiple of \(p^s\). Consequently, by induction, \(C_G(a)\) has a subgroup of order \(p^r\), so has \(G\). Otherwise, we must have \(\abs{\calz G}\) is a multiple of \(p\). Thanks to Proposition~\ref{prop:PreCauchysTheorem}, there exists a cyclic subgroup \(H\) of \(\calz G\) with order \(p\). We have then the quotient \(G/H\), since \(H\) is normal, which has cardinality \(p^{s-1}k < n\). So, by induction, there exists a subgroup \(K/H\) of \(G/H\) with \(p^{r-1}\) elements, thus so \(\abs K = p^r\).
\end{proof}

It follows the generalization of Proposition~\ref{prop:PreCauchysTheorem} to non-abelian groups to.

\begin{corollary}[Cauchy's Theorem]\label{coroll:CauchysTheorem}
Let \(G\) be a finite group. Then for every prime \(p \in \nn\) that divides \(\abs G\) there exists \(x \in G\) such that \(\ord x = p\).
\end{corollary}

At this point it is best we introduce some names. A \(p\)-group, for \(p \ge 2\) prime, is a group of cardinality \(p^n\) for some \(n \ge 1\). If \(G\) is a finite group, \(p\) a prime and \(s \ge 1\) such that \(p^s \divides \abs G\) and \(p^{s+1} \ndivides \abs G\), the subgroups of \(G\) of cardinality \(p^s\) are called {\em Sylow} \(p\)-subgroups.

The other Sylow Theorems can be derived by using some actions and the following lemma.

\begin{lemma}\label{lem:PreSylowTheorem2}
Let \(H\) be a group of order \(p^r\), for some prime \(p\) and \(r \in \nn^{\ge 1}\), and \(\phi\) an action of \(H\) on a set \(X\); consider \(X_0 \coloneq \set{x \in X \mid \stab_\phi x = H}\). Then
\[\abs X \equiv \abs{X_0} \bmod p .\]
\end{lemma}

What does this lemma say? \(\stab_\phi x\) is a subgroup of \(G\) and if \(H = \stab_\phi x\) then \(\orb_{\phi} x = \set{x}\). To put it in other words, \(X_0\) is the subset of the elements of \(X\) {\em fixed} by \(\phi\), that is the \(x \in X\) such that \(\phi_g x = x\) for every \(g \in H\). \NotaInterna{We may refer to \(X_0\) as \(\operatorname{fix} \phi\), for example\dots{}}

\begin{proof}[Proof of Lemma~\ref{lem:PreSylowTheorem2}]
The proof does not require any sophisticated tool other than those of previous section. Recall that \(X\) is chopped into orbits and the elements of \(X_0\) have banal orbits while, the non banal orbits have cardinality that divides \(p^r\), that is \(p^n\) with (it is important!) \(1 \le n \le r\).
\end{proof}

\begin{proposition}[Second Sylow Theorem]\label{prop:SylowTheorem2}
Let \(p\) be a prime number and  \(G\) a group with \(\abs G = p^s k\), for \(s, k \in \nn^{\ge 1}\) such that \(p \ndivides k\). Let \(S\) and \(H\) be subgroups of \(G\) with cardinality \(p^s\) and \(p^r\) respectively. Then \(\inv g H g \subseteq S\) for some \(g \in G\). In particular, two any subgroups of \(G\) with \(p^s\) elements are conjugated.
\end{proposition}

Recall that \(G\) is partitioned by its lateral classes --- say the left ones, for example ---, and we can make \(H\) act on the set of these classes as follows:
\begin{align*}
& \phi : H \to \cals(G/\call_S) \\
& \phi_h (sS) := h(sS) .
\end{align*}

\begin{proof} Thus, let us consider the action just introduced and write, thanks to Lemma~\ref{lem:PreSylowTheorem2}, the relation
\[\underbrace{[G:S]}_{\abs{G/\call_S}} \equiv \abs\Omega \bmod p ,\]
where
\[\Omega \coloneq \set{gS \in G/\call_S \mid \stab_\phi (gS) = H} .\]
By assumption, \(p\) does not divide \([G:S]\), hence it neither divides \(\abs\Omega\). In particular \(\abs\Omega \ne 0\), so there exists \(gS \in G/\call_S\) such that \(\phi_h(gS) = hgS = gS\) for every \(h \in H\). That is, \((\inv g h g) S = S\) for every \(h \in H\), and then \(\inv g H g \subseteq S\).
\end{proof}

\begin{proposition}[Third Sylow Theorem]\label{prop:SylowTheorem}
Let \(p\) be a prime number and  \(G\) a group with \(\abs G = p^s k\), for \(s, k \in \nn^{\ge 1}\) such that \(p \ndivides k\). Let \(s_p\) be the number of subgroups of \(G\) with \(p^s\) elements. Then
\[\begin{cases}
s_p \equiv 1 \bmod p \\
s_p \text{ divides } k .
\end{cases}\]
\end{proposition}

\begin{proof}
Let \(X\) be the family of the subgroups of \(G\) with cardinality \(p^s\), and consider the action of \(G\) on \(X\)
\begin{align*}
& \eta : G \to \cals X\\
& \eta_g (H) := \inv g H g .
\end{align*}
Because of the Second Sylow Theorem, the action \(\eta\) has a unique orbit, \(X\) itself, which has cardinality \(s_p\). Thus for every \(H \in X\),
\[s_p = \abs X = \underbrace{\abs{\orb_\eta H} = \frac{\abs G}{\abs{\stab_\eta H}}}_{\text{by Proposition~\ref{prop:OrbsStabsCosets}}}.\]
But \(\abs{\stab_\eta H} \ge \abs H = p^s\) and \(\abs{\stab_\eta H}\) divides \(\abs G = p^s k\), hence \(s_p\) does divide \(k\), being \(p \ndivides k\). For the remaining part, we need to consider the action \(\eta\) restricted to one the Sylow \(p\)-subgroups, call it \(S\):
\begin{align*}
& \theta : S \to \cals X\\
& \theta_g (H) := \inv g H g .
\end{align*}
If we consider
\[X_0 \coloneq \set{H \in X \mid \inv g H g = H \text{ for every } g \in S}\]
then we have by Lemma~\ref{lem:PreSylowTheorem2} we have
%is a singleton. It has at least one element because
\[s_p = \abs X \equiv \abs{X_0} \bmod p .\]
We show now that \(X_0\) is a singleton. In fact, if it were empty, \(p\) would divide \(s_p\), which itself divides \(k\), hence the absurd because \(p \ndivides k\) by assumption. On the other hand, take \(H \in X_0\): in particular, \(S\) is a subgroup of \(\stab_\theta H\) and consequently of \(\stab_\eta H\). \NotaInterna{what?}
\end{proof}

